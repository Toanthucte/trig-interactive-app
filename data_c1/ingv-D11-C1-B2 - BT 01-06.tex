\documentclass[12pt,a4paper]{article}
\usepackage[top=1.5cm, bottom=1.5cm, left=2.0cm, right=1.5cm] {geometry}
\usepackage{amsmath,amssymb,fontawesome}
\usepackage{tkz-euclide}
\usepackage{setspace}
\usepackage{lastpage}

\usepackage{tikz,tkz-tab}
%\usepackage[solcolor]{ex_test}
%\usepackage[dethi]{ex_test} % Chỉ hiển thị đề thi
\usepackage[loigiai]{ex_test} % Hiển thị lời giải
%\usepackage[color]{ex_test} % Khoanh các đáp án
\everymath{\displaystyle}

\def\colorEX{\color{purple}}
%\def\colorEX{}%Không tô màu đáp án đúng trong tùy chọn loigiai
\renewtheorem{ex}{\color{violet}Câu}
\renewcommand{\FalseEX}{\stepcounter{dapan}{{\bf \textcolor{blue}{\Alph{dapan}.}}}}
\renewcommand{\TrueEX}{\stepcounter{dapan}{{\bf \textcolor{blue}{\Alph{dapan}.}}}}

%---------- Khai báo viết tắt, in đáp án
\newcommand{\hoac}[1]{ %hệ hoặc
    \left[\begin{aligned}#1\end{aligned}\right.}
\newcommand{\heva}[1]{ %hệ và
    \left\{\begin{aligned}#1\end{aligned}\right.}

%Tiêu đề
\newcommand{\tenso}{THÀNH PHỐ HỒ CHÍ MINH}
\newcommand{\tentruong}{HD-NQQ}
\newcommand{\tenkythi}{GTLG CỦA MỘT GÓC LƯỢNG GIÁC}
\newcommand{\tenmonthi}{Môn học: D11-C1-B2 - BT 01-06}
\newcommand{\thoigian}{}
\newcommand{\tieude}[1]{
   \begin{tabular}{cm{1cm}cm{3cm}cm{3cm}}
    {\bf \tenso} & & {\bf \tenkythi} \\
    {\bf \tentruong} & & {\bf \tenmonthi}\\
    && {\bf Thời gian: \bf \thoigian \, phút}\\
    && { \fbox{\bf Mã đề: #1}}
   \end{tabular}\\\\
    
   {Họ tên HS: \dotfill Số báo danh \dotfill}\\
}
\newcommand{\chantrang}[2]{\rfoot{Trang \thepage $-$ Mã đề #2}}
\pagestyle{fancy}
\fancyhf{}
\renewcommand{\headrulewidth}{0pt} 
\renewcommand{\footrulewidth}{0pt}
\usetikzlibrary{shapes.geometric,arrows,calc,intersections,angles,quotes,patterns,snakes,positioning}
%\dotlinefull{ex}
\begin{document}
%Thiết lập giãn dọng 1.5cm 
%\setlength{\lineskip}{1.5em}
%Nội dung trắc nghiệm bắt đầu ở đây


\tieude{001}
%\chantrang{\pageref{LastPage}}{001}
\setcounter{page}{1}
{\bf PHẦN I. Câu trắc nghiệm nhiều phương án lựa chọn.}
\setcounter{ex}{0}
\Opensolutionfile{ans}[ans/ans001-1]
\begin{ex}[Tính giá trị đặc biệt của một góc lượng giác]
 Tính $\sin\frac{103 \pi}{3}$.\\ 
\choice
{ $ \sqrt{3} $ }
   { $ \frac{\sqrt{3}}{3} $ }
     { \True $ \frac{\sqrt{3}}{2} $ }
    { $ \frac{1}{2} $ }
\loigiai{ 
  
 }\end{ex}

\begin{ex}
 Tính $\cos\frac{25 \pi}{3}$.\\ 
\choice
{ \True $ \frac{1}{2} $ }
   { $ \frac{\sqrt{3}}{2} $ }
     { $ \sqrt{3} $ }
    { $ \frac{\sqrt{3}}{3} $ }
\loigiai{ 
  
 }\end{ex}

\begin{ex}
 Tính $\sin\frac{2 \pi}{3}$.\\ 
\choice
{ $ - \frac{\sqrt{3}}{3} $ }
   { \True $ \frac{\sqrt{3}}{2} $ }
     { $ - \sqrt{3} $ }
    { $ - \frac{1}{2} $ }
\loigiai{ 
  
 }\end{ex}

\begin{ex}
 Tính $\cot\frac{103 \pi}{3}$.\\ 
\choice
{ $ \sqrt{3} $ }
   { $ \frac{1}{2} $ }
     { \True $ \frac{\sqrt{3}}{3} $ }
    { $ \frac{\sqrt{3}}{2} $ }
\loigiai{ 
  
 }\end{ex}

\begin{ex}
 Tính $\cot\frac{103 \pi}{3}$.\\ 
\choice
{ $ \frac{1}{2} $ }
   { $ \frac{\sqrt{3}}{2} $ }
     { $ \sqrt{3} $ }
    { \True $ \frac{\sqrt{3}}{3} $ }
\loigiai{ 
  
 }\end{ex}

\begin{ex}
 Tính $\sin\frac{103 \pi}{3}$.\\ 
\choice
{ \True $ \frac{\sqrt{3}}{2} $ }
   { $ \sqrt{3} $ }
     { $ \frac{1}{2} $ }
    { $ \frac{\sqrt{3}}{3} $ }
\loigiai{ 
  
 }\end{ex}

\Closesolutionfile{ans}
{\bf PHẦN II. Câu trắc nghiệm đúng sai.}
\setcounter{ex}{0}
\Opensolutionfile{ans}[ans/ans001-2]
\begin{ex}[\text{Cho góc $x (a<x<b)$. Tìm khẳng định đúng về dấu của GTLG}]
 Cho góc lượng giác $\alpha \in \left( \pi;\frac{3 \pi}{2} \right)$. Xét tính đúng-sai của các khẳng định sau.
\choiceTFt
{ $\sin \alpha > 0$ }
   { $\cos \alpha > 0$  }
     { \True $\tan \alpha > 0$ }
    { \True $\cot \alpha > 0$ }
\loigiai{ 
 

 a) Khẳng định đã cho là khẳng định sai.

 Vì $\alpha \in \left( \pi;\frac{3 \pi}{2} \right)$ nên $\sin \alpha < 0$.

b) Khẳng định đã cho là khẳng định sai.

 Vì $\alpha \in \left( \pi;\frac{3 \pi}{2} \right)$ nên $\cos \alpha < 0$.

c) Khẳng định đã cho là khẳng định đúng.

 Vì $\alpha \in \left( \pi;\frac{3 \pi}{2} \right)$ nên $\sin \alpha < 0, \cos \alpha < 0 \Rightarrow \tan \alpha >0$.

d) Khẳng định đã cho là khẳng định đúng.

 Vì $\alpha \in \left( \pi;\frac{3 \pi}{2} \right)$ nên $\sin \alpha < 0, \cos \alpha < 0 \Rightarrow \cot \alpha >0$.

 
 }\end{ex}

\begin{ex}
 Cho góc lượng giác $\alpha \in \left( \frac{\pi}{2};\pi \right)$. Xét tính đúng-sai của các khẳng định sau.
\choiceTFt
{ $\sin \alpha < 0$ }
   { \True $\cos \alpha < 0$ }
     { \True $\tan \alpha < 0$ }
    { $\cot \alpha > 0$ }
\loigiai{ 
 

 a) Khẳng định đã cho là khẳng định sai.

 Vì $\alpha \in \left( \frac{\pi}{2};\pi \right)$ nên $\sin \alpha > 0$.

b) Khẳng định đã cho là khẳng định đúng.

 Vì $\alpha \in \left( \frac{\pi}{2};\pi \right)$ nên $\cos \alpha < 0$.

c) Khẳng định đã cho là khẳng định đúng.

 Vì $\alpha \in \left( \frac{\pi}{2};\pi \right)$ nên $\sin \alpha > 0, \cos \alpha < 0 \Rightarrow \tan \alpha <0$.

d) Khẳng định đã cho là khẳng định sai.

 Vì $\alpha \in \left( \frac{\pi}{2};\pi \right)$ nên $\sin \alpha > 0, \cos \alpha < 0 \Rightarrow \cot \alpha <0$.

 
 }\end{ex}

\begin{ex}
 Cho góc lượng giác $\alpha \in \left( \frac{\pi}{2};\pi \right)$. Xét tính đúng-sai của các khẳng định sau.
\choiceTFt
{ \True $\sin \alpha > 0$ }
   { \True $\cos \alpha < 0$ }
     { $\tan \alpha > 0$ }
    { \True $\cot \alpha < 0$ }
\loigiai{ 
 

 a) Khẳng định đã cho là khẳng định đúng.

 Vì $\alpha \in \left( \frac{\pi}{2};\pi \right)$ nên $\sin \alpha > 0$.

b) Khẳng định đã cho là khẳng định đúng.

 Vì $\alpha \in \left( \frac{\pi}{2};\pi \right)$ nên $\cos \alpha < 0$.

c) Khẳng định đã cho là khẳng định sai.

 Vì $\alpha \in \left( \frac{\pi}{2};\pi \right)$ nên $\sin \alpha > 0, \cos \alpha < 0 \Rightarrow \tan \alpha <0$.

d) Khẳng định đã cho là khẳng định đúng.

 Vì $\alpha \in \left( \frac{\pi}{2};\pi \right)$ nên $\sin \alpha > 0, \cos \alpha < 0 \Rightarrow \cot \alpha <0$.

 
 }\end{ex}

\begin{ex}
 Cho góc lượng giác $\alpha \in \left( \frac{5 \pi}{2};3\pi \right)$. Xét tính đúng-sai của các khẳng định sau.
\choiceTFt
{ $\sin \alpha < 0$ }
   { \True $\cos \alpha < 0$ }
     { $\tan \alpha > 0$ }
    { \True $\cot \alpha < 0$ }
\loigiai{ 
 

 a) Khẳng định đã cho là khẳng định sai.

 Vì $\alpha \in \left( \frac{5 \pi}{2};3\pi \right)$ nên $\sin \alpha > 0$.

b) Khẳng định đã cho là khẳng định đúng.

 Vì $\alpha \in \left( \frac{5 \pi}{2};3\pi \right)$ nên $\cos \alpha < 0$.

c) Khẳng định đã cho là khẳng định sai.

 Vì $\alpha \in \left( \frac{5 \pi}{2};3\pi \right)$ nên $\sin \alpha > 0, \cos \alpha < 0 \Rightarrow \tan \alpha <0$.

d) Khẳng định đã cho là khẳng định đúng.

 Vì $\alpha \in \left( \frac{5 \pi}{2};3\pi \right)$ nên $\sin \alpha > 0, \cos \alpha < 0 \Rightarrow \cot \alpha <0$.

 
 }\end{ex}

\begin{ex}
 Cho góc lượng giác $\alpha \in \left( \frac{\pi}{2};\pi \right)$. Xét tính đúng-sai của các khẳng định sau.
\choiceTFt
{ $\sin \alpha < 0$ }
   { \True $\cos \alpha < 0$ }
     { \True $\tan \alpha < 0$ }
    { $\cot \alpha > 0$ }
\loigiai{ 
 

 a) Khẳng định đã cho là khẳng định sai.

 Vì $\alpha \in \left( \frac{\pi}{2};\pi \right)$ nên $\sin \alpha > 0$.

b) Khẳng định đã cho là khẳng định đúng.

 Vì $\alpha \in \left( \frac{\pi}{2};\pi \right)$ nên $\cos \alpha < 0$.

c) Khẳng định đã cho là khẳng định đúng.

 Vì $\alpha \in \left( \frac{\pi}{2};\pi \right)$ nên $\sin \alpha > 0, \cos \alpha < 0 \Rightarrow \tan \alpha <0$.

d) Khẳng định đã cho là khẳng định sai.

 Vì $\alpha \in \left( \frac{\pi}{2};\pi \right)$ nên $\sin \alpha > 0, \cos \alpha < 0 \Rightarrow \cot \alpha <0$.

 
 }\end{ex}

\begin{ex}
 Cho góc lượng giác $\alpha \in \left( \frac{3 \pi}{2}; 2\pi \right)$. Xét tính đúng-sai của các khẳng định sau.
\choiceTFt
{ \True $\sin \alpha < 0$ }
   { \True $\cos \alpha > 0$ }
     { \True $\tan \alpha < 0$ }
    { $\cot \alpha > 0$ }
\loigiai{ 
 

 a) Khẳng định đã cho là khẳng định đúng.

 Vì $\alpha \in \left( \frac{3 \pi}{2}; 2\pi \right)$ nên $\sin \alpha < 0$.

b) Khẳng định đã cho là khẳng định đúng.

 Vì $\alpha \in \left( \frac{3 \pi}{2}; 2\pi \right)$ nên $\cos \alpha < 0$.

c) Khẳng định đã cho là khẳng định đúng.

 Vì $\alpha \in \left( \frac{3 \pi}{2}; 2\pi \right)$ nên $\sin \alpha < 0, \cos \alpha > 0 \Rightarrow \tan \alpha <0$.

d) Khẳng định đã cho là khẳng định sai.

 Vì $\alpha \in \left( \frac{3 \pi}{2}; 2\pi \right)$ nên $\sin \alpha < 0, \cos \alpha > 0 \Rightarrow \cot \alpha <0$.

 
 }\end{ex}

\Closesolutionfile{ans}
{\bf PHẦN III. Câu trắc nghiệm trả lời ngắn.}
\setcounter{ex}{0}
\Opensolutionfile{ans}[ans/ans001-3]
\begin{ex}[Cho sinx (hoặc cosx), x thuộc (a;b). Tìm cosx (hoặc sinx)]
 Cho góc lượng giác $\alpha$ thỏa mãn $\cos \alpha=\frac{5}{9}, \alpha \in \left( \frac{3 \pi}{2}; 2\pi \right)$. Tính $\sin\alpha$ (kết quả làm tròn đến hàng phần mười).\\ 
\shortans[4]{-0,83}

\loigiai{ 
 Vì $\alpha \in \left( \frac{3 \pi}{2}; 2\pi \right)$ nên $\sin\alpha < 0$.

$\sin\alpha =\sqrt{1-\frac{25}{81}}=- \frac{2 \sqrt{14}}{9}=-0,83$.Đáp án: -0,83 
 }\end{ex}

\begin{ex}
 Cho góc lượng giác $\alpha$ thỏa mãn $\sin \alpha=\frac{4}{5}, \alpha \in \left( 2\pi;\frac{5 \pi}{2} \right)$. Tính $\cos\alpha$ (kết quả làm tròn đến hàng phần mười).\\ 
\shortans[4]{0,6}

\loigiai{ 
 Vì $\alpha \in \left( 2\pi;\frac{5 \pi}{2} \right)$ nên $\cos\alpha > 0$.

$\cos\alpha =\sqrt{1-\frac{16}{25}}=\frac{3}{5}=0,6$.Đáp án: 0,6 
 }\end{ex}

\begin{ex}
 Cho góc lượng giác $\alpha$ thỏa mãn $\sin \alpha=- \frac{9}{10}, \alpha \in \left( \frac{3 \pi}{2}; 2\pi \right)$. Tính $\cos\alpha$ (kết quả làm tròn đến hàng phần mười).\\ 
\shortans[4]{0,4}

\loigiai{ 
 Vì $\alpha \in \left( \frac{3 \pi}{2}; 2\pi \right)$ nên $\cos\alpha > 0$.

$\cos\alpha =\sqrt{1-\frac{81}{100}}=\frac{\sqrt{19}}{10}=0,4$.Đáp án: 0,4 
 }\end{ex}

\begin{ex}
 Cho góc lượng giác $\alpha$ thỏa mãn $\sin \alpha=\frac{1}{3}, \alpha \in \left( 0;\frac{\pi}{2} \right)$. Tính $\cos\alpha$ (kết quả làm tròn đến hàng phần mười).\\ 
\shortans[4]{0,9}

\loigiai{ 
 Vì $\alpha \in \left( 0;\frac{\pi}{2} \right)$ nên $\cos\alpha > 0$.

$\cos\alpha =\sqrt{1-\frac{1}{9}}=\frac{2 \sqrt{2}}{3}=0,9$.Đáp án: 0,9 
 }\end{ex}

\begin{ex}
 Cho góc lượng giác $\alpha$ thỏa mãn $\sin \alpha=\frac{2}{3}, \alpha \in \left( 0;\frac{\pi}{2} \right)$. Tính $\cos\alpha$ (kết quả làm tròn đến hàng phần mười).\\ 
\shortans[4]{0,7}

\loigiai{ 
 Vì $\alpha \in \left( 0;\frac{\pi}{2} \right)$ nên $\cos\alpha > 0$.

$\cos\alpha =\sqrt{1-\frac{4}{9}}=\frac{\sqrt{5}}{3}=0,7$.Đáp án: 0,7 
 }\end{ex}

\begin{ex}
 Cho góc lượng giác $\alpha$ thỏa mãn $\cos \alpha=\frac{1}{2}, \alpha \in \left( \frac{7 \pi}{2}; 4\pi \right)$. Tính $\sin\alpha$ (kết quả làm tròn đến hàng phần mười).\\ 
\shortans[4]{-0,87}

\loigiai{ 
 Vì $\alpha \in \left( \frac{7 \pi}{2}; 4\pi \right)$ nên $\sin\alpha < 0$.

$\sin\alpha =\sqrt{1-\frac{1}{4}}=- \frac{\sqrt{3}}{2}=-0,87$.Đáp án: -0,87 
 }\end{ex}

\begin{ex}[Cho sinx (hoặc cosx), x thuộc (a;b). Tìm tanx (hoặc cotx)]
 Cho góc lượng giác $\alpha$ thỏa mãn $\sin \alpha=- \frac{1}{6}, \alpha \in \left( \frac{7 \pi}{2}; 4\pi \right)$. Tính $\tan\alpha$ (kết quả làm tròn đến hàng phần mười).\\ 
\shortans[4]{-0,2}

\loigiai{ 
 Vì $\alpha \in \left( \frac{7 \pi}{2}; 4\pi \right)$ nên $\cos\alpha > 0$.

$\cos\alpha =\sqrt{1-\frac{1}{36}}=\frac{\sqrt{35}}{6}$.

$\tan\alpha=- \frac{1}{6}:\frac{\sqrt{35}}{6}=-0,2$.

Đáp án: -0,2 
 }\end{ex}

\begin{ex}
 Cho góc lượng giác $\alpha$ thỏa mãn $\cos \alpha=\frac{9}{11}, \alpha \in \left( 0;\frac{\pi}{2} \right)$. Tính $\cot\alpha$ (kết quả làm tròn đến hàng phần mười).\\ 
\shortans[4]{1,4}

\loigiai{ 
 Vì $\alpha \in \left( 0;\frac{\pi}{2} \right)$ nên $\sin\alpha > 0$.

$\sin\alpha =\sqrt{1-\frac{81}{121}}=\frac{2 \sqrt{10}}{11}$.

$\cot\alpha=0.818181818181818:\frac{2 \sqrt{10}}{11}=1,4$.

Đáp án: 1,4 
 }\end{ex}

\begin{ex}
 Cho góc lượng giác $\alpha$ thỏa mãn $\sin \alpha=- \frac{3}{4}, \alpha \in \left( \pi;\frac{3 \pi}{2} \right)$. Tính $\tan\alpha$ (kết quả làm tròn đến hàng phần mười).\\ 
\shortans[4]{1,1}

\loigiai{ 
 Vì $\alpha \in \left( \pi;\frac{3 \pi}{2} \right)$ nên $\cos\alpha < 0$.

$\cos\alpha =-\sqrt{1-\frac{9}{16}}=- \frac{\sqrt{7}}{4}$.

$\tan\alpha=- \frac{3}{4}:- \frac{\sqrt{7}}{4}=1,1$.

Đáp án: 1,1 
 }\end{ex}

\begin{ex}
 Cho góc lượng giác $\alpha$ thỏa mãn $\sin \alpha=- \frac{5}{6}, \alpha \in \left( 3 \pi;\frac{7 \pi}{2} \right)$. Tính $\cot\alpha$ (kết quả làm tròn đến hàng phần mười).\\ 
\shortans[4]{0,7}

\loigiai{ 
 Vì $\alpha \in \left( 3 \pi;\frac{7 \pi}{2} \right)$ nên $\cos\alpha < 0$.

$\cos\alpha =-\sqrt{1-\frac{25}{36}}=- \frac{\sqrt{11}}{6}$.

$\cot\alpha=- \frac{\sqrt{11}}{6}:- \frac{5}{6}=0,7$.

Đáp án: 0,7 
 }\end{ex}

\begin{ex}
 Cho góc lượng giác $\alpha$ thỏa mãn $\sin \alpha=\frac{3}{4}, \alpha \in \left( \frac{5 \pi}{2};3\pi \right)$. Tính $\cot\alpha$ (kết quả làm tròn đến hàng phần mười).\\ 
\shortans[4]{-0,9}

\loigiai{ 
 Vì $\alpha \in \left( \frac{5 \pi}{2};3\pi \right)$ nên $\cos\alpha < 0$.

$\cos\alpha =-\sqrt{1-\frac{9}{16}}=- \frac{\sqrt{7}}{4}$.

$\cot\alpha=- \frac{\sqrt{7}}{4}:\frac{3}{4}=-0,9$.

Đáp án: -0,9 
 }\end{ex}

\begin{ex}
 Cho góc lượng giác $\alpha$ thỏa mãn $\cos \alpha=\frac{1}{6}, \alpha \in \left( 2\pi;\frac{5 \pi}{2} \right)$. Tính $\tan\alpha$ (kết quả làm tròn đến hàng phần mười).\\ 
\shortans[4]{5,9}

\loigiai{ 
 Vì $\alpha \in \left( 2\pi;\frac{5 \pi}{2} \right)$ nên $\sin\alpha > 0$.

$\sin\alpha =\sqrt{1-\frac{1}{36}}=\frac{\sqrt{35}}{6}$.

$\tan\alpha=\frac{\sqrt{35}}{6}:\frac{1}{6}=5,9$.

Đáp án: 5,9 
 }\end{ex}

\begin{ex}[Cho tanx (hoặc cotx). Tìm P=(asinx+bcosx)/(csinx+dcosx)]
 Cho góc lượng giác ${x}$ thỏa mãn $\tan x=6$. Tính giá trị biểu thức $P=\frac{5 \cos x + 6 \sin x}{- 2 \cos x + 6 \sin x}$ (kết quả làm tròn đến hàng phần mười).\\ 
\shortans[4]{1,2}

\loigiai{ 
 $P=\frac{5 \cos x + 6 \sin x}{- 2 \cos x + 6 \sin x}=\frac{6 \tan x + 5}{6 \tan x - 2}=\frac{41}{34}=1,2$.

Đáp án: 1,2 
 }\end{ex}

\begin{ex}
 Cho góc lượng giác ${x}$ thỏa mãn $\tan x=4$. Tính giá trị biểu thức $P=\frac{- 5 \cos x - 2 \sin x}{- \cos x + 4 \sin x}$ (kết quả làm tròn đến hàng phần mười).\\ 
\shortans[4]{-0,9}

\loigiai{ 
 $P=\frac{- 5 \cos x - 2 \sin x}{- \cos x + 4 \sin x}=\frac{- 2 \tan x - 5}{4 \tan x - 1}=- \frac{13}{15}=-0,9$.

Đáp án: -0,9 
 }\end{ex}

\begin{ex}
 Cho góc lượng giác ${x}$ thỏa mãn $\tan x=-2$. Tính giá trị biểu thức $P=\frac{5 \cos x + 4 \sin x}{- 5 \cos x + 5 \sin x}$ (kết quả làm tròn đến hàng phần mười).\\ 
\shortans[4]{0,2}

\loigiai{ 
 $P=\frac{5 \cos x + 4 \sin x}{- 5 \cos x + 5 \sin x}=\frac{4 \tan x + 5}{5 \tan x - 5}=\frac{1}{5}=0,2$.

Đáp án: 0,2 
 }\end{ex}

\begin{ex}
 Cho góc lượng giác ${x}$ thỏa mãn $\tan x=-2$. Tính giá trị biểu thức $P=\frac{4 \cos x - \sin x}{- 3 \cos x + 6 \sin x}$ (kết quả làm tròn đến hàng phần mười).\\ 
\shortans[4]{-0,4}

\loigiai{ 
 $P=\frac{4 \cos x - \sin x}{- 3 \cos x + 6 \sin x}=\frac{4 - \tan x}{6 \tan x - 3}=- \frac{2}{5}=-0,4$.

Đáp án: -0,4 
 }\end{ex}

\begin{ex}
 Cho góc lượng giác ${x}$ thỏa mãn $\tan x=-3$. Tính giá trị biểu thức $P=\frac{5 \cos x + \sin x}{- 4 \cos x - \sin x}$ (kết quả làm tròn đến hàng phần mười).\\ 
\shortans[4]{-2,0}

\loigiai{ 
 $P=\frac{5 \cos x + \sin x}{- 4 \cos x - \sin x}=\frac{\tan x + 5}{- \tan x - 4}=-2=-2,0$.

Đáp án: -2,0 
 }\end{ex}

\begin{ex}
 Cho góc lượng giác ${x}$ thỏa mãn $\tan x=-3$. Tính giá trị biểu thức $P=\frac{3 \cos x - 3 \sin x}{- 4 \cos x - \sin x}$ (kết quả làm tròn đến hàng phần mười).\\ 
\shortans[4]{-12,0}

\loigiai{ 
 $P=\frac{3 \cos x - 3 \sin x}{- 4 \cos x - \sin x}=\frac{3 - 3 \tan x}{- \tan x - 4}=-12=-12,0$.

Đáp án: -12,0 
 }\end{ex}

\begin{ex}[Cho tanx (hoặc cotx). Tìm $P=\dfrac{a\sin^2 x+b\sin x\cos x}{c\sin^2 x+d\cos^2 x}$]
 Cho góc lượng giác ${x}$ thỏa mãn $\tan x=-2$. Tính giá trị biểu thức $P=\dfrac{4\sin^2 x +4\sin x \cos x}{2\sin^2x-1\cos^2x}$ (kết quả làm tròn đến hàng phần mười).\\ 
\shortans[4]{1,1}

\loigiai{ 
 $P=\dfrac{4\sin^2 x +4\sin x \cos x}{2\sin^2x-1\cos^2x}=\dfrac{4\tan^2x+4\tan x}{2\tan^2x -1}=\frac{8}{7}=1,1$.

Đáp án: 1,1 
 }\end{ex}

\begin{ex}
 Cho góc lượng giác ${x}$ thỏa mãn $\tan x=-1$. Tính giá trị biểu thức $P=\dfrac{-1\sin^2 x -5\sin x \cos x}{6\sin^2x-2\cos^2x}$ (kết quả làm tròn đến hàng phần mười).\\ 
\shortans[4]{1,0}

\loigiai{ 
 $P=\dfrac{-1\sin^2 x -5\sin x \cos x}{6\sin^2x-2\cos^2x}=\dfrac{-1\tan^2x-5\tan x}{6\tan^2x -2}=1=1,0$.

Đáp án: 1,0 
 }\end{ex}

\begin{ex}
 Cho góc lượng giác ${x}$ thỏa mãn $\tan x=5$. Tính giá trị biểu thức $P=\dfrac{-2\sin^2 x -5\sin x \cos x}{-1\sin^2x+4\cos^2x}$ (kết quả làm tròn đến hàng phần mười).\\ 
\shortans[4]{3,6}

\loigiai{ 
 $P=\dfrac{-2\sin^2 x -5\sin x \cos x}{-1\sin^2x+4\cos^2x}=\dfrac{-2\tan^2x-5\tan x}{-1\tan^2x +4}=\frac{25}{7}=3,6$.

Đáp án: 3,6 
 }\end{ex}

\begin{ex}
 Cho góc lượng giác ${x}$ thỏa mãn $\tan x=3$. Tính giá trị biểu thức $P=\dfrac{-2\sin^2 x +1\sin x \cos x}{-6\sin^2x-3\cos^2x}$ (kết quả làm tròn đến hàng phần mười).\\ 
\shortans[4]{0,3}

\loigiai{ 
 $P=\dfrac{-2\sin^2 x +1\sin x \cos x}{-6\sin^2x-3\cos^2x}=\dfrac{-2\tan^2x+1\tan x}{-6\tan^2x -3}=\frac{5}{19}=0,3$.

Đáp án: 0,3 
 }\end{ex}

\begin{ex}
 Cho góc lượng giác ${x}$ thỏa mãn $\tan x=-4$. Tính giá trị biểu thức $P=\dfrac{-4\sin^2 x -1\sin x \cos x}{-6\sin^2x+2\cos^2x}$ (kết quả làm tròn đến hàng phần mười).\\ 
\shortans[4]{0,6}

\loigiai{ 
 $P=\dfrac{-4\sin^2 x -1\sin x \cos x}{-6\sin^2x+2\cos^2x}=\dfrac{-4\tan^2x-1\tan x}{-6\tan^2x +2}=\frac{30}{47}=0,6$.

Đáp án: 0,6 
 }\end{ex}

\begin{ex}
 Cho góc lượng giác ${x}$ thỏa mãn $\tan x=-2$. Tính giá trị biểu thức $P=\dfrac{2\sin^2 x -6\sin x \cos x}{-6\sin^2x+3\cos^2x}$ (kết quả làm tròn đến hàng phần mười).\\ 
\shortans[4]{-1,0}

\loigiai{ 
 $P=\dfrac{2\sin^2 x -6\sin x \cos x}{-6\sin^2x+3\cos^2x}=\dfrac{2\tan^2x-6\tan x}{-6\tan^2x +3}=- \frac{20}{21}=-1,0$.

Đáp án: -1,0 
 }\end{ex}

\Closesolutionfile{ans}

 \begin{center}
-----HẾT-----
\end{center}

 %\newpage 
%\begin{center}
%{\bf BẢNG ĐÁP ÁN MÃ ĐỀ 1 }
%\end{center}
%{\bf Phần 1 }
% \inputansbox{6}{ans001-1}
%{\bf Phần 2 }
% \inputansbox{2}{ans001-2}
%{\bf Phần 3 }
% \inputansbox{6}{ans001-3}
\newpage 



\end{document}