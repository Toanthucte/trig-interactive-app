\documentclass[12pt,a4paper]{article}
\usepackage[top=1.5cm, bottom=1.5cm, left=2.0cm, right=1.5cm] {geometry}
\usepackage{amsmath,amssymb,fontawesome}
\usepackage{tkz-euclide}
\usepackage{setspace}
\usepackage{lastpage}

\usepackage{tikz,tkz-tab}
%\usepackage[solcolor]{ex_test}
%\usepackage[dethi]{ex_test} % Chỉ hiển thị đề thi
\usepackage[loigiai]{ex_test} % Hiển thị lời giải
%\usepackage[color]{ex_test} % Khoanh các đáp án
\everymath{\displaystyle}

\def\colorEX{\color{purple}}
%\def\colorEX{}%Không tô màu đáp án đúng trong tùy chọn loigiai
\renewtheorem{ex}{\color{violet}Câu}
\renewcommand{\FalseEX}{\stepcounter{dapan}{{\bf \textcolor{blue}{\Alph{dapan}.}}}}
\renewcommand{\TrueEX}{\stepcounter{dapan}{{\bf \textcolor{blue}{\Alph{dapan}.}}}}

%---------- Khai báo viết tắt, in đáp án
\newcommand{\hoac}[1]{ %hệ hoặc
    \left[\begin{aligned}#1\end{aligned}\right.}
\newcommand{\heva}[1]{ %hệ và
    \left\{\begin{aligned}#1\end{aligned}\right.}

%Tiêu đề
\newcommand{\tenso}{THÀNH PHỐ HỒ CHÍ MINH}
\newcommand{\tentruong}{HD-NQQ}
\newcommand{\tenkythi}{Bài 1 Góc Lượng Giác }
\newcommand{\tenmonthi}{Môn thi: Mức độ 1}
\newcommand{\thoigian}{}
\newcommand{\tieude}[1]{
   \begin{tabular}{cm{3cm}cm{3cm}cm{3cm}}
    {\bf \tenso} & & {\bf \tenkythi} \\
    {\bf \tentruong} & & {\bf \tenmonthi}\\
    && {\bf Thời gian: \bf \thoigian \, phút}\\
    && { \fbox{\bf Mã đề: #1}}
   \end{tabular}\\\\
    
   {Họ tên HS: \dotfill Số báo danh \dotfill}\\
}
\newcommand{\chantrang}[2]{\rfoot{Trang \thepage $-$ Mã đề #2}}
\pagestyle{fancy}
\fancyhf{}
\renewcommand{\headrulewidth}{0pt} 
\renewcommand{\footrulewidth}{0pt}
\usetikzlibrary{shapes.geometric,arrows,calc,intersections,angles,quotes,patterns,snakes,positioning}

\begin{document}
%Thiết lập giãn dọng 1.5cm 
%\setlength{\lineskip}{1.5em}
%Nội dung trắc nghiệm bắt đầu ở đây


\tieude{001}
\chantrang{\pageref{LastPage}}{001}
\setcounter{page}{1}
{\bf PHẦN I. Câu trắc nghiệm nhiều phương án lựa chọn.}
\setcounter{ex}{0}
\Opensolutionfile{ans}[ans/ans001-1]
\begin{ex}
 Đổi số đo của góc $900^\circ$ sang radian ta được kết quả bằng\\ 
\choice
{ \True $5 \pi$ }
   { $\frac{91 \pi}{18}$ }
     { $\frac{31 \pi}{6}$ }
    { $\frac{44 \pi}{9}$ }
\loigiai{ 
 Áp dụng công thức chuyển đổi: $900^\circ=\dfrac{900.\pi}{180}=5 \pi$. 
 }\end{ex}

\begin{ex}
 Đổi số đo của góc $405^\circ$ sang radian ta được kết quả bằng\\ 
\choice
{ $\frac{29 \pi}{12}$ }
   { $\frac{83 \pi}{36}$ }
     { $\frac{77 \pi}{36}$ }
    { \True $\frac{9 \pi}{4}$ }
\loigiai{ 
 Áp dụng công thức chuyển đổi: $405^\circ=\dfrac{405.\pi}{180}=\frac{9 \pi}{4}$. 
 }\end{ex}

\begin{ex}
 Đổi số đo của góc $-270^\circ$ sang radian ta được kết quả bằng\\ 
\choice
{ $- \frac{4 \pi}{3}$ }
   { \True $- \frac{3 \pi}{2}$ }
     { $- \frac{13 \pi}{9}$ }
    { $- \frac{29 \pi}{18}$ }
\loigiai{ 
 Áp dụng công thức chuyển đổi: $-270^\circ=\dfrac{-270.\pi}{180}=- \frac{3 \pi}{2}$. 
 }\end{ex}

\begin{ex}
 Đổi số đo của góc $945^\circ$ sang radian ta được kết quả bằng\\ 
\choice
{ $\frac{185 \pi}{36}$ }
   { $\frac{191 \pi}{36}$ }
     { $\frac{65 \pi}{12}$ }
    { \True $\frac{21 \pi}{4}$ }
\loigiai{ 
 Áp dụng công thức chuyển đổi: $945^\circ=\dfrac{945.\pi}{180}=\frac{21 \pi}{4}$. 
 }\end{ex}

\begin{ex}
 Đổi số đo của góc $-810^\circ$ sang radian ta được kết quả bằng\\ 
\choice
{ $- \frac{83 \pi}{18}$ }
   { $- \frac{13 \pi}{3}$ }
     { \True $- \frac{9 \pi}{2}$ }
    { $- \frac{40 \pi}{9}$ }
\loigiai{ 
 Áp dụng công thức chuyển đổi: $-810^\circ=\dfrac{-810.\pi}{180}=- \frac{9 \pi}{2}$. 
 }\end{ex}

\begin{ex}
 Đổi số đo của góc $330^\circ$ sang radian ta được kết quả bằng\\ 
\choice
{ \True $\frac{11 \pi}{6}$ }
   { $\frac{31 \pi}{18}$ }
     { $\frac{17 \pi}{9}$ }
    { $2 \pi$ }
\loigiai{ 
 Áp dụng công thức chuyển đổi: $330^\circ=\dfrac{330.\pi}{180}=\frac{11 \pi}{6}$. 
 }\end{ex}

\begin{ex}
 Đổi số đo của góc $-720^\circ$ sang radian ta được kết quả bằng\\ 
\choice
{ $- \frac{23 \pi}{6}$ }
   { $- \frac{37 \pi}{9}$ }
     { \True $- 4 \pi$ }
    { $- \frac{71 \pi}{18}$ }
\loigiai{ 
 Áp dụng công thức chuyển đổi: $-720^\circ=\dfrac{-720.\pi}{180}=- 4 \pi$. 
 }\end{ex}

\begin{ex}
 Đổi số đo của góc $840^\circ$ sang radian ta được kết quả bằng\\ 
\choice
{ $\frac{29 \pi}{6}$ }
   { \True $\frac{14 \pi}{3}$ }
     { $\frac{85 \pi}{18}$ }
    { $\frac{41 \pi}{9}$ }
\loigiai{ 
 Áp dụng công thức chuyển đổi: $840^\circ=\dfrac{840.\pi}{180}=\frac{14 \pi}{3}$. 
 }\end{ex}

\begin{ex}
 Đổi số đo của góc $435^\circ$ sang radian ta được kết quả bằng\\ 
\choice
{ $\frac{89 \pi}{36}$ }
   { $\frac{31 \pi}{12}$ }
     { \True $\frac{29 \pi}{12}$ }
    { $\frac{83 \pi}{36}$ }
\loigiai{ 
 Áp dụng công thức chuyển đổi: $435^\circ=\dfrac{435.\pi}{180}=\frac{29 \pi}{12}$. 
 }\end{ex}

\begin{ex}
 Đổi số đo của góc $-780^\circ$ sang radian ta được kết quả bằng\\ 
\choice
{ \True $- \frac{13 \pi}{3}$ }
   { $- \frac{25 \pi}{6}$ }
     { $- \frac{40 \pi}{9}$ }
    { $- \frac{77 \pi}{18}$ }
\loigiai{ 
 Áp dụng công thức chuyển đổi: $-780^\circ=\dfrac{-780.\pi}{180}=- \frac{13 \pi}{3}$. 
 }\end{ex}

\begin{ex}
 Đổi số đo của góc $- \frac{13 \pi}{3}$ sang độ ta được kết quả bằng\\ 
\choice
{ $-750^\circ$ }
   { \True $-780^\circ$ }
     { $-803^\circ$ }
    { $-770^\circ$ }
\loigiai{ 
 Áp dụng công thức chuyển đổi: $- \frac{13 \pi}{3}=\left(\dfrac{- \frac{13 \pi}{3}.180}{\pi}\right)^\circ=-780^\circ$. 
 }\end{ex}

\begin{ex}
 Đổi số đo của góc $\frac{91 \pi}{18}$ sang độ ta được kết quả bằng\\ 
\choice
{ $940^\circ$ }
   { \True $910^\circ$ }
     { $920^\circ$ }
    { $865^\circ$ }
\loigiai{ 
 Áp dụng công thức chuyển đổi: $\frac{91 \pi}{18}=\left(\dfrac{\frac{91 \pi}{18}.180}{\pi}\right)^\circ=910^\circ$. 
 }\end{ex}

\begin{ex}
 Đổi số đo của góc $\frac{37 \pi}{12}$ sang độ ta được kết quả bằng\\ 
\choice
{ \True $555^\circ$ }
   { $565^\circ$ }
     { $585^\circ$ }
    { $517^\circ$ }
\loigiai{ 
 Áp dụng công thức chuyển đổi: $\frac{37 \pi}{12}=\left(\dfrac{\frac{37 \pi}{12}.180}{\pi}\right)^\circ=555^\circ$. 
 }\end{ex}

\begin{ex}
 Đổi số đo của góc $- \frac{43 \pi}{12}$ sang độ ta được kết quả bằng\\ 
\choice
{ $-635^\circ$ }
   { \True $-645^\circ$ }
     { $-615^\circ$ }
    { $-695^\circ$ }
\loigiai{ 
 Áp dụng công thức chuyển đổi: $- \frac{43 \pi}{12}=\left(\dfrac{- \frac{43 \pi}{12}.180}{\pi}\right)^\circ=-645^\circ$. 
 }\end{ex}

\begin{ex}
 Đổi số đo của góc $\frac{8 \pi}{3}$ sang độ ta được kết quả bằng\\ 
\choice
{ $490^\circ$ }
   { $431^\circ$ }
     { \True $480^\circ$ }
    { $510^\circ$ }
\loigiai{ 
 Áp dụng công thức chuyển đổi: $\frac{8 \pi}{3}=\left(\dfrac{\frac{8 \pi}{3}.180}{\pi}\right)^\circ=480^\circ$. 
 }\end{ex}

\begin{ex}
 Đổi số đo của góc $\frac{13 \pi}{4}$ sang độ ta được kết quả bằng\\ 
\choice
{ $595^\circ$ }
   { \True $585^\circ$ }
     { $615^\circ$ }
    { $551^\circ$ }
\loigiai{ 
 Áp dụng công thức chuyển đổi: $\frac{13 \pi}{4}=\left(\dfrac{\frac{13 \pi}{4}.180}{\pi}\right)^\circ=585^\circ$. 
 }\end{ex}

\begin{ex}
 Đổi số đo của góc $\frac{17 \pi}{3}$ sang độ ta được kết quả bằng\\ 
\choice
{ $974^\circ$ }
   { $1050^\circ$ }
     { $1030^\circ$ }
    { \True $1020^\circ$ }
\loigiai{ 
 Áp dụng công thức chuyển đổi: $\frac{17 \pi}{3}=\left(\dfrac{\frac{17 \pi}{3}.180}{\pi}\right)^\circ=1020^\circ$. 
 }\end{ex}

\begin{ex}
 Đổi số đo của góc $- \frac{53 \pi}{18}$ sang độ ta được kết quả bằng\\ 
\choice
{ $-500^\circ$ }
   { \True $-530^\circ$ }
     { $-573^\circ$ }
    { $-520^\circ$ }
\loigiai{ 
 Áp dụng công thức chuyển đổi: $- \frac{53 \pi}{18}=\left(\dfrac{- \frac{53 \pi}{18}.180}{\pi}\right)^\circ=-530^\circ$. 
 }\end{ex}

\begin{ex}
 Đổi số đo của góc $- 2 \pi$ sang độ ta được kết quả bằng\\ 
\choice
{ $-330^\circ$ }
   { $-350^\circ$ }
     { \True $-360^\circ$ }
    { $-401^\circ$ }
\loigiai{ 
 Áp dụng công thức chuyển đổi: $- 2 \pi=\left(\dfrac{- 2 \pi.180}{\pi}\right)^\circ=-360^\circ$. 
 }\end{ex}

\begin{ex}
 Đổi số đo của góc $\frac{\pi}{6}$ sang độ ta được kết quả bằng\\ 
\choice
{ $60^\circ$ }
   { $-9^\circ$ }
     { \True $30^\circ$ }
    { $40^\circ$ }
\loigiai{ 
 Áp dụng công thức chuyển đổi: $\frac{\pi}{6}=\left(\dfrac{\frac{\pi}{6}.180}{\pi}\right)^\circ=30^\circ$. 
 }\end{ex}

\begin{ex}
 Trên đường tròn lượng giác, cho góc lượng giác có số đo ${\frac{4 \pi}{3}}$ thì mọi góc lượng giác có cùng tia đầu và tia cuối với góc lượng giác trên đều có số đo dạng nào trong các dạng sau? 
\choice
{ $\frac{4 \pi}{3}+k3\pi, k\in \mathbb{Z}$ }
   { \True $\frac{4 \pi}{3}+k2\pi, k\in \mathbb{Z}$ }
     { $\frac{4 \pi}{3}+k4\pi, k\in \mathbb{Z}$ }
    { $\frac{4 \pi}{3}+k\pi, k\in \mathbb{Z}$ }
\loigiai{ 
 $\frac{4 \pi}{3}+k2\pi, k\in \mathbb{Z}$ là khẳng định đúng. 
 }\end{ex}

\begin{ex}
 Trên đường tròn lượng giác, cho góc lượng giác có số đo ${\frac{8 \pi}{9}}$ thì mọi góc lượng giác có cùng tia đầu và tia cuối với góc lượng giác trên đều có số đo dạng nào trong các dạng sau? 
\choice
{ $\frac{8 \pi}{9}+k4\pi, k\in \mathbb{Z}$ }
   { $\frac{8 \pi}{9}+\dfrac{k\pi }{2}, k\in \mathbb{Z}$ }
     { $\frac{8 \pi}{9}+k3\pi, k\in \mathbb{Z}$ }
    { \True $\frac{8 \pi}{9}+k2\pi, k\in \mathbb{Z}$ }
\loigiai{ 
 $\frac{8 \pi}{9}+k2\pi, k\in \mathbb{Z}$ là khẳng định đúng. 
 }\end{ex}

\begin{ex}
 Trên đường tròn lượng giác, cho góc lượng giác có số đo ${8 \pi}$ thì mọi góc lượng giác có cùng tia đầu và tia cuối với góc lượng giác trên đều có số đo dạng nào trong các dạng sau? 
\choice
{ \True $8 \pi+k2\pi, k\in \mathbb{Z}$ }
   { $8 \pi+k3\pi, k\in \mathbb{Z}$ }
     { $8 \pi+k\pi, k\in \mathbb{Z}$ }
    { $8 \pi+\dfrac{k\pi }{3}, k\in \mathbb{Z}$ }
\loigiai{ 
 $8 \pi+k2\pi, k\in \mathbb{Z}$ là khẳng định đúng. 
 }\end{ex}

\begin{ex}
 Trên đường tròn lượng giác, cho góc lượng giác có số đo ${\frac{7 \pi}{2}}$ thì mọi góc lượng giác có cùng tia đầu và tia cuối với góc lượng giác trên đều có số đo dạng nào trong các dạng sau? 
\choice
{ $\frac{7 \pi}{2}+\dfrac{k\pi }{3}, k\in \mathbb{Z}$ }
   { $\frac{7 \pi}{2}+\dfrac{k\pi }{4}, k\in \mathbb{Z}$ }
     { $\frac{7 \pi}{2}+k3\pi, k\in \mathbb{Z}$ }
    { \True $\frac{7 \pi}{2}+k2\pi, k\in \mathbb{Z}$ }
\loigiai{ 
 $\frac{7 \pi}{2}+k2\pi, k\in \mathbb{Z}$ là khẳng định đúng. 
 }\end{ex}

\begin{ex}
 Trên đường tròn lượng giác, cho góc lượng giác có số đo ${\pi}$ thì mọi góc lượng giác có cùng tia đầu và tia cuối với góc lượng giác trên đều có số đo dạng nào trong các dạng sau? 
\choice
{ \True $\pi+k2\pi, k\in \mathbb{Z}$ }
   { $\pi+k4\pi, k\in \mathbb{Z}$ }
     { $\pi+k\pi, k\in \mathbb{Z}$ }
    { $\pi+\dfrac{k\pi }{2}, k\in \mathbb{Z}$ }
\loigiai{ 
 $\pi+k2\pi, k\in \mathbb{Z}$ là khẳng định đúng. 
 }\end{ex}

\begin{ex}
 Trên đường tròn lượng giác, cho góc lượng giác có số đo ${\frac{\pi}{8}}$ thì mọi góc lượng giác có cùng tia đầu và tia cuối với góc lượng giác trên đều có số đo dạng nào trong các dạng sau? 
\choice
{ \True $\frac{\pi}{8}+k2\pi, k\in \mathbb{Z}$ }
   { $\frac{\pi}{8}+k3\pi, k\in \mathbb{Z}$ }
     { $\frac{\pi}{8}+k\pi, k\in \mathbb{Z}$ }
    { $\frac{\pi}{8}+\dfrac{k\pi }{2}, k\in \mathbb{Z}$ }
\loigiai{ 
 $\frac{\pi}{8}+k2\pi, k\in \mathbb{Z}$ là khẳng định đúng. 
 }\end{ex}

\begin{ex}
 Trên đường tròn lượng giác, cho góc lượng giác có số đo ${\frac{\pi}{6}}$ thì mọi góc lượng giác có cùng tia đầu và tia cuối với góc lượng giác trên đều có số đo dạng nào trong các dạng sau? 
\choice
{ $\frac{\pi}{6}+k4\pi, k\in \mathbb{Z}$ }
   { $\frac{\pi}{6}+k\pi, k\in \mathbb{Z}$ }
     { \True $\frac{\pi}{6}+k2\pi, k\in \mathbb{Z}$ }
    { $\frac{\pi}{6}+k3\pi, k\in \mathbb{Z}$ }
\loigiai{ 
 $\frac{\pi}{6}+k2\pi, k\in \mathbb{Z}$ là khẳng định đúng. 
 }\end{ex}

\begin{ex}
 Trên đường tròn lượng giác, cho góc lượng giác có số đo ${\frac{3 \pi}{8}}$ thì mọi góc lượng giác có cùng tia đầu và tia cuối với góc lượng giác trên đều có số đo dạng nào trong các dạng sau? 
\choice
{ $\frac{3 \pi}{8}+\dfrac{k\pi }{4}, k\in \mathbb{Z}$ }
   { $\frac{3 \pi}{8}+k4\pi, k\in \mathbb{Z}$ }
     { $\frac{3 \pi}{8}+\dfrac{k\pi }{3}, k\in \mathbb{Z}$ }
    { \True $\frac{3 \pi}{8}+k2\pi, k\in \mathbb{Z}$ }
\loigiai{ 
 $\frac{3 \pi}{8}+k2\pi, k\in \mathbb{Z}$ là khẳng định đúng. 
 }\end{ex}

\begin{ex}
 Trên đường tròn lượng giác, cho góc lượng giác có số đo ${\frac{7 \pi}{6}}$ thì mọi góc lượng giác có cùng tia đầu và tia cuối với góc lượng giác trên đều có số đo dạng nào trong các dạng sau? 
\choice
{ $\frac{7 \pi}{6}+k\pi, k\in \mathbb{Z}$ }
   { $\frac{7 \pi}{6}+\dfrac{k\pi }{2}, k\in \mathbb{Z}$ }
     { \True $\frac{7 \pi}{6}+k2\pi, k\in \mathbb{Z}$ }
    { $\frac{7 \pi}{6}+k3\pi, k\in \mathbb{Z}$ }
\loigiai{ 
 $\frac{7 \pi}{6}+k2\pi, k\in \mathbb{Z}$ là khẳng định đúng. 
 }\end{ex}

\begin{ex}
 Trên đường tròn lượng giác, cho góc lượng giác có số đo ${\pi}$ thì mọi góc lượng giác có cùng tia đầu và tia cuối với góc lượng giác trên đều có số đo dạng nào trong các dạng sau? 
\choice
{ $\pi+\dfrac{k\pi }{2}, k\in \mathbb{Z}$ }
   { $\pi+k3\pi, k\in \mathbb{Z}$ }
     { $\pi+\dfrac{k\pi }{3}, k\in \mathbb{Z}$ }
    { \True $\pi+k2\pi, k\in \mathbb{Z}$ }
\loigiai{ 
 $\pi+k2\pi, k\in \mathbb{Z}$ là khẳng định đúng. 
 }\end{ex}

\Closesolutionfile{ans}

 \begin{center}
-----HẾT-----
\end{center}

 %\newpage 
%\begin{center}
%{\bf BẢNG ĐÁP ÁN MÃ ĐỀ 1 }
%\end{center}
%{\bf Phần 1 }
% \inputansbox{6}{ans001-1}



\end{document}