\documentclass[12pt,a4paper]{article}
\usepackage[top=1.5cm, bottom=1.5cm, left=2.0cm, right=1.5cm] {geometry}
\usepackage{amsmath,amssymb,fontawesome}
\usepackage{tkz-euclide}
\usepackage{setspace}
\usepackage{lastpage}

\usepackage{tikz,tkz-tab}
%\usepackage[solcolor]{ex_test}
%\usepackage[dethi]{ex_test} % Chỉ hiển thị đề thi
\usepackage[loigiai]{ex_test} % Hiển thị lời giải
%\usepackage[color]{ex_test} % Khoanh các đáp án
%\usepackage[book]{ex_test} % Lam sach 
\everymath{\displaystyle}

\def\colorEX{\color{purple}}
%\def\colorEX{}%Không tô màu đáp án đúng trong tùy chọn loigiai
\renewtheorem{ex}{\color{violet}Câu}
\renewcommand{\FalseEX}{\stepcounter{dapan}{{\bf \textcolor{blue}{\Alph{dapan}.}}}}
\renewcommand{\TrueEX}{\stepcounter{dapan}{{\bf \textcolor{blue}{\Alph{dapan}.}}}}

%---------- Khai báo viết tắt, in đáp án
\newcommand{\hoac}[1]{ %hệ hoặc
    \left[\begin{aligned}#1\end{aligned}\right.}
\newcommand{\heva}[1]{ %hệ và
    \left\{\begin{aligned}#1\end{aligned}\right.}

%Tiêu đề
\newcommand{\tenso}{THÀNH PHỐ HỒ CHÍ MINH}
\newcommand{\tentruong}{BÀI GIẢI MẪU - HD-NQQ}
\newcommand{\tenkythi}{GTLG CỦA MỘT GÓC LƯỢNG GIÁC}
\newcommand{\tenmonthi}{Môn: D11-C1-B2 - 12 DẠNG}
\newcommand{\thoigian}{}
\newcommand{\tieude}[1]{
   \begin{tabular}{cm{0,2cm}cm{3cm}cm{3cm}}
    {\bf \tenso} & & {\bf \tenkythi} \\
    {\bf \tentruong} & & {\bf \tenmonthi}\\
    && {\bf Thời gian: \bf \thoigian \, phút}\\
    && { \fbox{\bf Mã đề: #1}}
   \end{tabular}\\\\
    
   {Họ tên HS: \dotfill Số báo danh \dotfill}\\
}
\newcommand{\chantrang}[2]{\rfoot{Trang \thepage $-$ Mã đề #2}}
\pagestyle{fancy}
\fancyhf{}
\renewcommand{\headrulewidth}{0pt} 
\renewcommand{\footrulewidth}{0pt}
\usetikzlibrary{shapes.geometric,arrows,calc,intersections,angles,quotes,patterns,snakes,positioning}
%\dotlinefull{ex} % Đổi lời giải thành các dòng chấm bằng số dòng của lời giải.
\begin{document}
%Thiết lập giãn dọng 1.5cm 
%\setlength{\lineskip}{1.5em}
%Nội dung trắc nghiệm bắt đầu ở đây


\tieude{001}
\chantrang{\pageref{LastPage}}{001}
\setcounter{page}{1}
{\bf PHẦN I. Câu trắc nghiệm nhiều phương án lựa chọn.}
\setcounter{ex}{0}
\Opensolutionfile{ans}[ans/ans001-1]
\begin{ex}[Tính giá trị đặc biệt của một góc lượng giác]
 Tính $\tan\frac{25 \pi}{3}$.\\ 
\choice
{ $ \frac{1}{2} $ }
   { $ \frac{\sqrt{3}}{3} $ }
     { \True $ \sqrt{3} $ }
    { $ \frac{\sqrt{3}}{2} $ }
\loigiai{ 
  
 }\end{ex}

\begin{ex}[\text{Cho góc $x (a<x<b)$. Tìm khẳng định đúng về dấu của GTLG}]
 Cho góc lượng giác ${\alpha}$ thỏa mãn $\alpha\in \left( \frac{5 \pi}{2};3\pi \right)$. Khẳng định nào sau đây là khẳng định đúng. 
\choice
{ $\cot \alpha >0$  }
   { $\tan \alpha >0$  }
     { \True $\sin \alpha >0$  }
    { $\cos \alpha >0$  }
\loigiai{ 
 $\sin \alpha >0$  là khẳng định đúng 
 }\end{ex}

\begin{ex}[\text{Cho $\sin{x}$ (hoặc $\cos{x}$), $x$ thuộc $(a;b)$. Tìm $\cos{x}$ (hoặc $\sin{x}$)}]
 Cho góc lượng giác $\alpha$ thỏa mãn $\cos \alpha=- \frac{1}{3}, \alpha \in \left( - \frac{3 \pi}{2};- \pi \right)$. Tính $\sin\alpha$ (kết quả làm tròn đến hàng phần mười). 
\choice
{ ${\frac{\sqrt{13}}{17}}$ }
   { ${\frac{\sqrt{6}}{13}}$ }
     { \True ${\frac{2 \sqrt{2}}{3}}$ }
    { ${\frac{2 \sqrt{3}}{5}}$ }
\loigiai{ 
 Vì $\alpha \in \left( - \frac{3 \pi}{2};- \pi \right)$ nên $\sin\alpha > 0$.

$\sin\alpha =\sqrt{1-\frac{1}{9}}=\frac{2 \sqrt{2}}{3}$. 
 }\end{ex}

\begin{ex}[\text{Cho $\sin{x}$ (hoặc $\cos{x}$), $x$ thuộc $(a;b)$. Tìm $\tan{x}$ (hoặc $\cot{x}$)}]
 Cho góc lượng giác $\alpha$ thỏa mãn $\sin \alpha=- \frac{5}{6}, \alpha \in \left( 3 \pi;\frac{7 \pi}{2} \right)$. Tính $\cot\alpha$. 
\choice
{ $\frac{\sqrt{13}}{9}$ }
   { \True $\frac{\sqrt{11}}{5}$ }
     { $\frac{\sqrt{2}}{8}$ }
    { $\frac{\sqrt{2}}{20}$ }
\loigiai{ 
 Vì $\alpha \in \left( 3 \pi;\frac{7 \pi}{2} \right)$ nên $\cos\alpha < 0$.

$\cos\alpha =-\sqrt{1-\frac{25}{36}}=- \frac{\sqrt{11}}{6}$.

$\cot\alpha=- \frac{\sqrt{11}}{6}:- \frac{5}{6}=\frac{\sqrt{11}}{5}$.

 
 }\end{ex}

\Closesolutionfile{ans}
{\bf PHẦN II. Câu trắc nghiệm đúng sai.}
\setcounter{ex}{0}
\Opensolutionfile{ans}[ans/ans001-2]
\begin{ex}[\text{Cho góc lượng giác thuộc cung phần tư. Xét Đ-S dấu của sin, cos, tan, cot}]
 Cho góc lượng giác $\alpha \in \left( 0;\frac{\pi}{2} \right)$. Xét tính đúng-sai của các khẳng định sau.
\choiceTFt
{ $\sin \alpha < 0$ }
   { \True $\cos \alpha > 0$ }
     { \True $\tan \alpha > 0$ }
    { \True $\cot \alpha > 0$ }
\loigiai{ 
 

 a) Khẳng định đã cho là khẳng định sai.

 Vì $\alpha \in \left( 0;\frac{\pi}{2} \right)$ nên $\sin \alpha > 0$.

b) Khẳng định đã cho là khẳng định đúng.

 Vì $\alpha \in \left( 0;\frac{\pi}{2} \right)$ nên $\cos \alpha > 0$.

c) Khẳng định đã cho là khẳng định đúng.

 Vì $\alpha \in \left( 0;\frac{\pi}{2} \right)$ nên $\sin \alpha > 0, \cos \alpha > 0 \Rightarrow \tan \alpha >0$.

d) Khẳng định đã cho là khẳng định đúng.

 Vì $\alpha \in \left( 0;\frac{\pi}{2} \right)$ nên $\sin \alpha > 0, \cos \alpha > 0 \Rightarrow \cot \alpha >0$.

 
 }\end{ex}

\begin{ex}[\text{Cho $\tan{x}$. Xét Đ-S: $\cot{x}$, $\sin^2 {x}$, $\cos^2{x}$, $\dfrac{asinx+bcosx}{csinx+dcosx}$}]
 Cho góc lượng giác ${x}$ thỏa mãn $\tan x=-4$. Xét tính đúng-sai của các khẳng định sau. 
\choiceTFt
{ $\cot x= \frac{3}{4}$  }
   { \True $\cos^2x=\frac{1}{17}$ }
     { \True $\sin^2x=\frac{16}{17}$ }
    { \True $P=\frac{- \cos x + 2 \sin x}{- 5 \cos x + 4 \sin x}=\frac{3}{7}$ }
\loigiai{ 
 

 a) Khẳng định đã cho là khẳng định sai.

 $\cot x=\dfrac{1}{\tan x}=1:-4=- \frac{1}{4}$

b) Khẳng định đã cho là khẳng định đúng.

 $1+\tan^2x=\dfrac{1}{\cos^2x} \Rightarrow \cos^2x=\dfrac{1}{1+\tan^2x}=\frac{1}{17}$

c) Khẳng định đã cho là khẳng định đúng.

 $\sin^2x=1-\cos^2x=1-\dfrac{1}{1+\tan^2x}=\frac{16}{17}$

d) Khẳng định đã cho là khẳng định đúng.

 $P=\frac{- \cos x + 2 \sin x}{- 5 \cos x + 4 \sin x}=\frac{2 \tan x - 1}{4 \tan x - 5}=\frac{3}{7}$.

 
 }\end{ex}

\begin{ex}[\text{Cho sinx (a<x<b). Xét Đ-S: dấu của cosx, cosx, sin(x+kpi/2), P=f(tanx)}]
 Cho $\sin x=- \frac{8}{11}, x\in \left( - \pi; - \frac{\pi}{2} \right)$. Xét tính đúng-sai của các khẳng định sau.
\choiceTFt
{ $\cos x >0$ }
   { \True $\cos x=- \frac{\sqrt{57}}{11}$ }
     { $\cos\left(x+ \frac{13 \pi}{2} \right)=- \frac{8}{11}$  }
    { $P=\frac{2 \tan x}{3 \tan^2 x - 1}=- \frac{16 \sqrt{57}}{135}$ }
\loigiai{ 
 

 a) Khẳng định đã cho là khẳng định sai.

 Với $x \in \left( - \pi; - \frac{\pi}{2} \right) $ thì $\cos x <0$

b) Khẳng định đã cho là khẳng định đúng.

 Vì $x \in \left( - \pi; - \frac{\pi}{2} \right)$ nên $\cos x < 0$.

$\cos x=-\sqrt{1-\frac{64}{121}}=- \frac{\sqrt{57}}{11}$.

c) Khẳng định đã cho là khẳng định sai.

 $\cos\left(x+ \frac{13 \pi}{2} \right)=\cos \left( x+\frac{\pi}{2}+3.2\pi \right)=\cos \left(x+\frac{\pi}{2}\right)=\cos \left[ \frac{\pi}{2}-(-x) \right]=\sin (-x)=-\sin x=0.727272727272727$.

d) Khẳng định đã cho là khẳng định sai.

 $\tan x=- \frac{8}{11}:- \frac{\sqrt{57}}{11}=\frac{8 \sqrt{57}}{57}$.

$\Rightarrow P=\frac{2 \tan x}{3 \tan^2 x - 1}=\frac{16 \sqrt{57}}{135}$.

 
 }\end{ex}

\begin{ex}[\text{Cho cosx (a<x<b). Xét Đ-S: dấu của sinx, sinx, sin(x+kpi/2), P=f(tanx)}]
 Cho $\cos x=- \frac{\sqrt{21}}{11}, x\in \left( - \pi; - \frac{\pi}{2} \right)$. Xét tính đúng-sai của các khẳng định sau.
\choiceTFt
{ \True $\sin x <0$ }
   { \True $\sin x=- \frac{10}{11}$ }
     { \True $\cos\left(x+ \frac{17 \pi}{2} \right)= \frac{10}{11}$ }
    { \True $P=\frac{\sqrt{5} \tan x}{- \tan^2 x - 4}=- \frac{5 \sqrt{105}}{92}$ }
\loigiai{ 
 

 a) Khẳng định đã cho là khẳng định đúng.

 Với $x \in \left( - \pi; - \frac{\pi}{2} \right) $ thì $\sin x >0$

b) Khẳng định đã cho là khẳng định đúng.

 Vì $x \in \left( - \pi; - \frac{\pi}{2} \right)$ nên $\sin x > 0$.

$\sin x =-\sqrt{1-\frac{21}{121}}=- \frac{10}{11}$.

c) Khẳng định đã cho là khẳng định đúng.

 $\cos\left(x+ \frac{17 \pi}{2} \right)=\cos \left( x+\frac{\pi}{2}+4.2\pi \right)=\cos \left(x+\frac{\pi}{2}\right)=\cos \left[ \frac{\pi}{2}-(-x) \right]=\sin (-x)=-\sin x=0.909090909090909$.

d) Khẳng định đã cho là khẳng định đúng.

 $\tan x=- \frac{10}{11}:- \frac{\sqrt{21}}{11}=\frac{10 \sqrt{21}}{21}$.

$\Rightarrow P=\frac{\sqrt{5} \tan x}{- \tan^2 x - 4}=- \frac{5 \sqrt{105}}{92}$.

 
 }\end{ex}

\Closesolutionfile{ans}
{\bf PHẦN III. Câu trắc nghiệm trả lời ngắn.}
\setcounter{ex}{0}
\Opensolutionfile{ans}[ans/ans001-3]
\begin{ex}[Cho sinx (hoặc cosx), x thuộc (a;b). Tìm cosx (hoặc sinx)]
 Cho góc lượng giác $\alpha$ thỏa mãn $\cos \alpha=\frac{7}{11}, \alpha \in \left( \frac{7 \pi}{2}; 4\pi \right)$. Tính $\sin\alpha$ (kết quả làm tròn đến hàng phần mười).\\ 
\shortans[4]{-0,77}

\loigiai{ 
 Vì $\alpha \in \left( \frac{7 \pi}{2}; 4\pi \right)$ nên $\sin\alpha < 0$.

$\sin\alpha =\sqrt{1-\frac{49}{121}}=- \frac{6 \sqrt{2}}{11}=-0,77$.Đáp án: -0,77 
 }\end{ex}

\begin{ex}[Cho sinx (hoặc cosx), x thuộc (a;b). Tìm tanx (hoặc cotx)]
 Cho góc lượng giác $\alpha$ thỏa mãn $\cos \alpha=- \frac{5}{7}, \alpha \in \left( \pi;\frac{3 \pi}{2} \right)$. Tính $\cot\alpha$ (kết quả làm tròn đến hàng phần mười).\\ 
\shortans[4]{1,00000000000000}

\loigiai{ 
 Vì $\alpha \in \left( \pi;\frac{3 \pi}{2} \right)$ nên $\sin\alpha < 0$.

$\sin\alpha =\sqrt{1-\frac{25}{49}}=- \frac{2 \sqrt{6}}{7}$.

$\cot\alpha=-0.714285714285714:- \frac{2 \sqrt{6}}{7}=1,00000000000000$.

Đáp án: 1,00000000000000 
 }\end{ex}

\begin{ex}[Cho tanx (hoặc cotx). Tìm P=(asinx+bcosx)/(csinx+dcosx)]
 Cho góc lượng giác ${x}$ thỏa mãn $\tan x=4$. Tính giá trị biểu thức $P=\frac{3 \cos x + 2 \sin x}{4 \cos x + 4 \sin x}$ (kết quả làm tròn đến hàng phần mười).\\ 
\shortans[4]{0,6}

\loigiai{ 
 $P=\frac{3 \cos x + 2 \sin x}{4 \cos x + 4 \sin x}=\frac{2 \tan x + 3}{4 \tan x + 4}=\frac{11}{20}=0,6$.

Đáp án: 0,6 
 }\end{ex}

\begin{ex}[Cho $\tan{x}$ (hoặc $\cot{x}$). Tìm $P=\dfrac{a\sin^2 x+b\sin x \cos x}{c\sin^2 x+d\cos^2 x}$]
 Cho góc lượng giác ${x}$ thỏa mãn $\tan x=-4$. Tính giá trị biểu thức $P=\dfrac{3\sin^2 x -5\sin x \cos x}{-1\sin^2x+5\cos^2x}$ (kết quả làm tròn đến hàng phần mười).\\ 
\shortans[4]{-6,2}

\loigiai{ 
 $P=\dfrac{3\sin^2 x -5\sin x \cos x}{-1\sin^2x+5\cos^2x}=\dfrac{3\tan^2x-5\tan x}{-1\tan^2x +5}=- \frac{68}{11}=-6,2$.

Đáp án: -6,2 
 }\end{ex}

\Closesolutionfile{ans}

 \begin{center}
-----HẾT-----
\end{center}

 %\newpage 
%\begin{center}
%{\bf BẢNG ĐÁP ÁN MÃ ĐỀ 1 }
%\end{center}
%{\bf Phần 1 }
% \inputansbox{6}{ans001-1}
%{\bf Phần 2 }
% \inputansbox{2}{ans001-2}
%{\bf Phần 3 }
% \inputansbox{6}{ans001-3}
\newpage 



\end{document}