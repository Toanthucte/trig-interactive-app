\documentclass[12pt,a4paper]{article}
\usepackage[top=1.5cm, bottom=1.5cm, left=2.0cm, right=1.5cm] {geometry}
\usepackage{amsmath,amssymb,fontawesome}
\usepackage{tkz-euclide}
\usepackage{setspace}
\usepackage{lastpage}

\usepackage{tikz,tkz-tab}
%\usepackage[solcolor]{ex_test}
%\usepackage[dethi]{ex_test} % Chỉ hiển thị đề thi
\usepackage[loigiai]{ex_test} % Hiển thị lời giải
%\usepackage[color]{ex_test} % Khoanh các đáp án
\everymath{\displaystyle}

\def\colorEX{\color{purple}}
%\def\colorEX{}%Không tô màu đáp án đúng trong tùy chọn loigiai
\renewtheorem{ex}{\color{violet}Câu}
\renewcommand{\FalseEX}{\stepcounter{dapan}{{\bf \textcolor{blue}{\Alph{dapan}.}}}}
\renewcommand{\TrueEX}{\stepcounter{dapan}{{\bf \textcolor{blue}{\Alph{dapan}.}}}}

%---------- Khai báo viết tắt, in đáp án
\newcommand{\hoac}[1]{ %hệ hoặc
    \left[\begin{aligned}#1\end{aligned}\right.}
\newcommand{\heva}[1]{ %hệ và
    \left\{\begin{aligned}#1\end{aligned}\right.}

%Tiêu đề
\newcommand{\tenso}{THÀNH PHỐ HỒ CHÍ MINH}
\newcommand{\tentruong}{HD-NQQ}
\newcommand{\tenkythi}{GÓC LƯỢNG GIÁC }
\newcommand{\tenmonthi}{Môn: D11-C1-B1 - BT 08-13}
\newcommand{\thoigian}{}
\newcommand{\tieude}[1]{
   \begin{tabular}{cm{1cm}cm{3cm}cm{3cm}}
    {\bf \tenso} & & {\bf \tenkythi} \\
    {\bf \tentruong} & & {\bf \tenmonthi}\\
    && {\bf Thời gian: \bf \thoigian \, phút}\\
    && { \fbox{\bf Mã đề: #1}}
   \end{tabular}\\\\
    
   {Họ tên HS: \dotfill Số báo danh \dotfill}\\
}
\newcommand{\chantrang}[2]{\rfoot{Trang \thepage $-$ Mã đề #2}}
\pagestyle{fancy}
\fancyhf{}
\renewcommand{\headrulewidth}{0pt} 
\renewcommand{\footrulewidth}{0pt}
\usetikzlibrary{shapes.geometric,arrows,calc,intersections,angles,quotes,patterns,snakes,positioning}

\begin{document}
%Thiết lập giãn dọng 1.5cm 
%\setlength{\lineskip}{1.5em}
%Nội dung trắc nghiệm bắt đầu ở đây


\tieude{001}
%\chantrang{\pageref{LastPage}}{001}
\setcounter{page}{1}
{\bf PHẦN I. Câu trắc nghiệm nhiều phương án lựa chọn.}
\setcounter{ex}{0}
\Opensolutionfile{ans}[ans/ans001-1]
\begin{ex}[13-M1. Cho góc x. Tìm tất cả các góc có cùng điểm biểu diễn với x]
 Trên đường tròn lượng giác, cho góc lượng giác có số đo ${\frac{\pi}{3}}$ thì mọi góc lượng giác có cùng tia đầu và tia cuối với góc lượng giác trên đều có số đo dạng nào trong các dạng sau? 
\choice
{ $\frac{\pi}{3}+k4\pi, k\in \mathbb{Z}$ }
   { $\frac{\pi}{3}+\dfrac{k\pi }{3}, k\in \mathbb{Z}$ }
     { $\frac{\pi}{3}+k\pi, k\in \mathbb{Z}$ }
    { \True $\frac{\pi}{3}+k2\pi, k\in \mathbb{Z}$ }
\loigiai{ 
 $\frac{\pi}{3}+k2\pi, k\in \mathbb{Z}$ là khẳng định đúng. 
 }\end{ex}

\begin{ex}
 Trên đường tròn lượng giác, cho góc lượng giác có số đo ${3 \pi}$ thì mọi góc lượng giác có cùng tia đầu và tia cuối với góc lượng giác trên đều có số đo dạng nào trong các dạng sau? 
\choice
{ $3 \pi+\dfrac{k\pi }{3}, k\in \mathbb{Z}$ }
   { \True $3 \pi+k2\pi, k\in \mathbb{Z}$ }
     { $3 \pi+k3\pi, k\in \mathbb{Z}$ }
    { $3 \pi+k4\pi, k\in \mathbb{Z}$ }
\loigiai{ 
 $3 \pi+k2\pi, k\in \mathbb{Z}$ là khẳng định đúng. 
 }\end{ex}

\begin{ex}
 Trên đường tròn lượng giác, cho góc lượng giác có số đo ${8 \pi}$ thì mọi góc lượng giác có cùng tia đầu và tia cuối với góc lượng giác trên đều có số đo dạng nào trong các dạng sau? 
\choice
{ $8 \pi+k4\pi, k\in \mathbb{Z}$ }
   { $8 \pi+\dfrac{k\pi }{4}, k\in \mathbb{Z}$ }
     { $8 \pi+\dfrac{k\pi }{2}, k\in \mathbb{Z}$ }
    { \True $8 \pi+k2\pi, k\in \mathbb{Z}$ }
\loigiai{ 
 $8 \pi+k2\pi, k\in \mathbb{Z}$ là khẳng định đúng. 
 }\end{ex}

\begin{ex}
 Trên đường tròn lượng giác, cho góc lượng giác có số đo ${\frac{2 \pi}{5}}$ thì mọi góc lượng giác có cùng tia đầu và tia cuối với góc lượng giác trên đều có số đo dạng nào trong các dạng sau? 
\choice
{ $\frac{2 \pi}{5}+k3\pi, k\in \mathbb{Z}$ }
   { $\frac{2 \pi}{5}+\dfrac{k\pi }{2}, k\in \mathbb{Z}$ }
     { $\frac{2 \pi}{5}+k4\pi, k\in \mathbb{Z}$ }
    { \True $\frac{2 \pi}{5}+k2\pi, k\in \mathbb{Z}$ }
\loigiai{ 
 $\frac{2 \pi}{5}+k2\pi, k\in \mathbb{Z}$ là khẳng định đúng. 
 }\end{ex}

\begin{ex}
 Trên đường tròn lượng giác, cho góc lượng giác có số đo ${3 \pi}$ thì mọi góc lượng giác có cùng tia đầu và tia cuối với góc lượng giác trên đều có số đo dạng nào trong các dạng sau? 
\choice
{ \True $3 \pi+k2\pi, k\in \mathbb{Z}$ }
   { $3 \pi+k3\pi, k\in \mathbb{Z}$ }
     { $3 \pi+\dfrac{k\pi }{3}, k\in \mathbb{Z}$ }
    { $3 \pi+\dfrac{k\pi }{4}, k\in \mathbb{Z}$ }
\loigiai{ 
 $3 \pi+k2\pi, k\in \mathbb{Z}$ là khẳng định đúng. 
 }\end{ex}

\begin{ex}
 Trên đường tròn lượng giác, cho góc lượng giác có số đo ${\frac{2 \pi}{7}}$ thì mọi góc lượng giác có cùng tia đầu và tia cuối với góc lượng giác trên đều có số đo dạng nào trong các dạng sau? 
\choice
{ $\frac{2 \pi}{7}+k\pi, k\in \mathbb{Z}$ }
   { $\frac{2 \pi}{7}+\dfrac{k\pi }{4}, k\in \mathbb{Z}$ }
     { \True $\frac{2 \pi}{7}+k2\pi, k\in \mathbb{Z}$ }
    { $\frac{2 \pi}{7}+\dfrac{k\pi }{3}, k\in \mathbb{Z}$ }
\loigiai{ 
 $\frac{2 \pi}{7}+k2\pi, k\in \mathbb{Z}$ là khẳng định đúng. 
 }\end{ex}

\Closesolutionfile{ans}
{\bf PHẦN II. Câu trắc nghiệm đúng sai.}
\setcounter{ex}{0}
\Opensolutionfile{ans}[ans/ans001-2]
\begin{ex}[08-M2. Cho góc radian. Xét Đ-S: Đổi sang độ, Điểm biểu diễn thuộc phần tư][Góc cùng điểm biểu diễn, Đếm số điểm biểu diễn]\\
 Cho góc lượng giác $\frac{37 \pi}{9}$. Xét tính đúng-sai của các khẳng định sau
\choiceTFt
{ $\frac{37 \pi}{9}=743^\circ$  }
   { Góc lượng giác đã cho có cùng điểm biểu diễn trên đường tròn lượng giác với góc $- \frac{\pi}{9}$  }
     { \True Điểm biểu diễn trên đường tròn lượng giác của góc đã cho thuộc phần tư thứ I }
    { \True Số điểm biểu diễn trên đường tròn lượng giác của góc $\frac{37 \pi}{9}+k\frac{2 \pi}{7},k \in \mathbb{Z}$ là ${7}$ }
\loigiai{ 
 

 a) Khẳng định đã cho là khẳng định sai.

 Ta có: $\frac{37 \pi}{9}=\left(\dfrac{\frac{37 \pi}{9}.180}{\pi}\right)^\circ=740^\circ$.

b) Khẳng định đã cho là khẳng định sai.

 Vì: $\frac{37 \pi}{9}=\frac{\pi}{9} +2.2\pi$

c) Khẳng định đã cho là khẳng định đúng.

 Vì: $\frac{37 \pi}{9}=\frac{\pi}{9} +2.2\pi$ nên điểm biểu diễn của góc đã cho thuộc phần tư thứ I.

d) Khẳng định đã cho là khẳng định đúng.

 $\frac{37 \pi}{9}+k\frac{2 \pi}{7}=\frac{37 \pi}{9}+\dfrac{k2\pi}{7}$ nên có ${7}$ điểm biểu diễn trên đường tròn lượng giác.

 
 }\end{ex}

\begin{ex}
 Cho góc lượng giác $\frac{71 \pi}{18}$. Xét tính đúng-sai của các khẳng định sau
\choiceTFt
{ $\frac{71 \pi}{18}=720^\circ$  }
   { \True Góc lượng giác đã cho có cùng điểm biểu diễn trên đường tròn lượng giác với góc $- \frac{\pi}{18}$ }
     { Điểm biểu diễn trên đường tròn lượng giác của góc đã cho thuộc phần tư thứ I }
    { \True Số điểm biểu diễn trên đường tròn lượng giác của góc $\frac{71 \pi}{18}+k\pi,k \in \mathbb{Z}$ là ${2}$ }
\loigiai{ 
 

 a) Khẳng định đã cho là khẳng định sai.

 Ta có: $\frac{71 \pi}{18}=\left(\dfrac{\frac{71 \pi}{18}.180}{\pi}\right)^\circ=710^\circ$.

b) Khẳng định đã cho là khẳng định đúng.

 Vì: $\frac{71 \pi}{18}=- \frac{\pi}{18} +2.2\pi$

c) Khẳng định đã cho là khẳng định sai.

 Vì: $\frac{71 \pi}{18}=- \frac{\pi}{18} +2.2\pi$ nên điểm biểu diễn của góc đã cho thuộc phần tư thứ IV.

d) Khẳng định đã cho là khẳng định đúng.

 $\frac{71 \pi}{18}+k\pi=\frac{71 \pi}{18}+\dfrac{k2\pi}{2}$ nên có ${2}$ điểm biểu diễn trên đường tròn lượng giác.

 
 }\end{ex}

\begin{ex}
 Cho góc lượng giác $\frac{19 \pi}{3}$. Xét tính đúng-sai của các khẳng định sau
\choiceTFt
{ \True $\frac{19 \pi}{3}=1140^\circ$ }
   { \True Góc lượng giác đã cho có cùng điểm biểu diễn trên đường tròn lượng giác với góc $\frac{\pi}{3}$ }
     { Điểm biểu diễn trên đường tròn lượng giác của góc đã cho thuộc phần tư thứ IV }
    { Số điểm biểu diễn trên đường tròn lượng giác của góc $\frac{19 \pi}{3}+k\frac{\pi}{4},k \in \mathbb{Z}$ là ${9}$ }
\loigiai{ 
 

 a) Khẳng định đã cho là khẳng định đúng.

 Ta có: $\frac{19 \pi}{3}=\left(\dfrac{\frac{19 \pi}{3}.180}{\pi}\right)^\circ=1140^\circ$.

b) Khẳng định đã cho là khẳng định đúng.

 Vì: $\frac{19 \pi}{3}=\frac{\pi}{3} +3.2\pi$

c) Khẳng định đã cho là khẳng định sai.

 Vì: $\frac{19 \pi}{3}=\frac{\pi}{3} +3.2\pi$ nên điểm biểu diễn của góc đã cho thuộc phần tư thứ I.

d) Khẳng định đã cho là khẳng định sai.

 $\frac{19 \pi}{3}+k\frac{\pi}{4}=\frac{19 \pi}{3}+\dfrac{k2\pi}{8}$ nên có ${8}$ điểm biểu diễn trên đường tròn lượng giác.

 
 }\end{ex}

\begin{ex}
 Cho góc lượng giác $- \frac{43 \pi}{12}$. Xét tính đúng-sai của các khẳng định sau
\choiceTFt
{ $- \frac{43 \pi}{12}=-628^\circ$  }
   { \True Góc lượng giác đã cho có cùng điểm biểu diễn trên đường tròn lượng giác với góc $\frac{5 \pi}{12}$ }
     { Điểm biểu diễn trên đường tròn lượng giác của góc đã cho thuộc phần tư thứ IV }
    { \True Số điểm biểu diễn trên đường tròn lượng giác của góc $- \frac{43 \pi}{12}+k\frac{\pi}{3},k \in \mathbb{Z}$ là ${6}$ }
\loigiai{ 
 

 a) Khẳng định đã cho là khẳng định sai.

 Ta có: $- \frac{43 \pi}{12}=\left(\dfrac{- \frac{43 \pi}{12}.180}{\pi}\right)^\circ=-645^\circ$.

b) Khẳng định đã cho là khẳng định đúng.

 Vì: $- \frac{43 \pi}{12}=\frac{5 \pi}{12} -2.2\pi$

c) Khẳng định đã cho là khẳng định sai.

 Vì: $- \frac{43 \pi}{12}=\frac{5 \pi}{12} -2.2\pi$ nên điểm biểu diễn của góc đã cho thuộc phần tư thứ I.

d) Khẳng định đã cho là khẳng định đúng.

 $- \frac{43 \pi}{12}+k\frac{\pi}{3}=- \frac{43 \pi}{12}+\dfrac{k2\pi}{6}$ nên có ${6}$ điểm biểu diễn trên đường tròn lượng giác.

 
 }\end{ex}

\begin{ex}
 Cho góc lượng giác $\frac{73 \pi}{12}$. Xét tính đúng-sai của các khẳng định sau
\choiceTFt
{ $\frac{73 \pi}{12}=1106^\circ$  }
   { Góc lượng giác đã cho có cùng điểm biểu diễn trên đường tròn lượng giác với góc $- \frac{\pi}{12}$  }
     { Điểm biểu diễn trên đường tròn lượng giác của góc đã cho thuộc phần tư thứ IV }
    { \True Số điểm biểu diễn trên đường tròn lượng giác của góc $\frac{73 \pi}{12}+k\pi,k \in \mathbb{Z}$ là ${2}$ }
\loigiai{ 
 

 a) Khẳng định đã cho là khẳng định sai.

 Ta có: $\frac{73 \pi}{12}=\left(\dfrac{\frac{73 \pi}{12}.180}{\pi}\right)^\circ=1095^\circ$.

b) Khẳng định đã cho là khẳng định sai.

 Vì: $\frac{73 \pi}{12}=\frac{\pi}{12} +3.2\pi$

c) Khẳng định đã cho là khẳng định sai.

 Vì: $\frac{73 \pi}{12}=\frac{\pi}{12} +3.2\pi$ nên điểm biểu diễn của góc đã cho thuộc phần tư thứ I.

d) Khẳng định đã cho là khẳng định đúng.

 $\frac{73 \pi}{12}+k\pi=\frac{73 \pi}{12}+\dfrac{k2\pi}{2}$ nên có ${2}$ điểm biểu diễn trên đường tròn lượng giác.

 
 }\end{ex}

\begin{ex}
 Cho góc lượng giác $\frac{7 \pi}{4}$. Xét tính đúng-sai của các khẳng định sau
\choiceTFt
{ \True $\frac{7 \pi}{4}=315^\circ$ }
   { Góc lượng giác đã cho có cùng điểm biểu diễn trên đường tròn lượng giác với góc $\frac{\pi}{4}$  }
     { \True Điểm biểu diễn trên đường tròn lượng giác của góc đã cho thuộc phần tư thứ IV }
    { Số điểm biểu diễn trên đường tròn lượng giác của góc $\frac{7 \pi}{4}+k\pi,k \in \mathbb{Z}$ là ${5}$ }
\loigiai{ 
 

 a) Khẳng định đã cho là khẳng định đúng.

 Ta có: $\frac{7 \pi}{4}=\left(\dfrac{\frac{7 \pi}{4}.180}{\pi}\right)^\circ=315^\circ$.

b) Khẳng định đã cho là khẳng định sai.

 Vì: $\frac{7 \pi}{4}=- \frac{\pi}{4} +2\pi$

c) Khẳng định đã cho là khẳng định đúng.

 Vì: $\frac{7 \pi}{4}=- \frac{\pi}{4} +1.2\pi$ nên điểm biểu diễn của góc đã cho thuộc phần tư thứ IV.

d) Khẳng định đã cho là khẳng định sai.

 $\frac{7 \pi}{4}+k\pi=\frac{7 \pi}{4}+\dfrac{k2\pi}{2}$ nên có ${2}$ điểm biểu diễn trên đường tròn lượng giác.

 
 }\end{ex}

\begin{ex}[09-M2. Cho góc độ. Xét Đ-S: Đổi sang radian, Điểm biểu diễn thuộc phần tư][Góc cùng điểm biểu diễn, Đếm số điểm biểu diễn] \\
 Cho góc lượng giác $-350^\circ$. Xét tính đúng-sai của các khẳng định sau
\choiceTFt
{ $-350^\circ=- \frac{86 \pi}{45}$  }
   { Góc lượng giác đã cho có cùng điểm biểu diễn trên đường tròn lượng giác với góc $-10^\circ$  }
     { \True Điểm biểu diễn trên đường tròn lượng giác của góc đã cho thuộc phần tư thứ I }
    { \True Số điểm biểu diễn trên đường tròn lượng giác của góc $-350^\circ +k36^\circ,k \in \mathbb{Z}$ là ${10}$ }
\loigiai{ 
 

 a) Khẳng định đã cho là khẳng định sai.

 Ta có: $-350^\circ=\dfrac{-350.\pi}{180}=- \frac{35 \pi}{18}$.

b) Khẳng định đã cho là khẳng định sai.

 Vì: $-350^\circ=10^\circ -1.360^\circ$

c) Khẳng định đã cho là khẳng định đúng.

 Vì: $-350^\circ=10^\circ -1.360^\circ$ nên điểm biểu diễn của góc đã cho thuộc phần tư thứ I.

d) Khẳng định đã cho là khẳng định đúng.

 $-350^\circ +k36^\circ=-350^\circ +\dfrac{k360^\circ}{10}$ nên có ${10}$ điểm biểu diễn trên đường tròn lượng giác.

 
 }\end{ex}

\begin{ex}
 Cho góc lượng giác $-675^\circ$. Xét tính đúng-sai của các khẳng định sau
\choiceTFt
{ $-675^\circ=- \frac{11 \pi}{3}$  }
   { Góc lượng giác đã cho có cùng điểm biểu diễn trên đường tròn lượng giác với góc $-45^\circ$  }
     { Điểm biểu diễn trên đường tròn lượng giác của góc đã cho thuộc phần tư thứ II }
    { Số điểm biểu diễn trên đường tròn lượng giác của góc $-675^\circ +k40^\circ,k \in \mathbb{Z}$ là ${11}$ }
\loigiai{ 
 

 a) Khẳng định đã cho là khẳng định sai.

 Ta có: $-675^\circ=\dfrac{-675.\pi}{180}=- \frac{15 \pi}{4}$.

b) Khẳng định đã cho là khẳng định sai.

 Vì: $-675^\circ=45^\circ -2.360^\circ$

c) Khẳng định đã cho là khẳng định sai.

 Vì: $-675^\circ=45^\circ -2.360^\circ$ nên điểm biểu diễn của góc đã cho thuộc phần tư thứ I.

d) Khẳng định đã cho là khẳng định sai.

 $-675^\circ +k40^\circ=-675^\circ +\dfrac{k360^\circ}{9}$ nên có ${9}$ điểm biểu diễn trên đường tròn lượng giác.

 
 }\end{ex}

\begin{ex}
 Cho góc lượng giác $930^\circ$. Xét tính đúng-sai của các khẳng định sau
\choiceTFt
{ $930^\circ=\frac{467 \pi}{90}$  }
   { Góc lượng giác đã cho có cùng điểm biểu diễn trên đường tròn lượng giác với góc $150^\circ$  }
     { Điểm biểu diễn trên đường tròn lượng giác của góc đã cho thuộc phần tư thứ II }
    { \True Số điểm biểu diễn trên đường tròn lượng giác của góc $930^\circ +k180^\circ,k \in \mathbb{Z}$ là ${2}$ }
\loigiai{ 
 

 a) Khẳng định đã cho là khẳng định sai.

 Ta có: $930^\circ=\dfrac{930.\pi}{180}=\frac{31 \pi}{6}$.

b) Khẳng định đã cho là khẳng định sai.

 Vì: $930^\circ=-150^\circ +3.360^\circ$

c) Khẳng định đã cho là khẳng định sai.

 Vì: $930^\circ=-150^\circ +3.360^\circ$ nên điểm biểu diễn của góc đã cho thuộc phần tư thứ III.

d) Khẳng định đã cho là khẳng định đúng.

 $930^\circ +k180^\circ=930^\circ +\dfrac{k360^\circ}{2}$ nên có ${2}$ điểm biểu diễn trên đường tròn lượng giác.

 
 }\end{ex}

\begin{ex}
 Cho góc lượng giác $-375^\circ$. Xét tính đúng-sai của các khẳng định sau
\choiceTFt
{ \True $-375^\circ=- \frac{25 \pi}{12}$ }
   { Góc lượng giác đã cho có cùng điểm biểu diễn trên đường tròn lượng giác với góc $15^\circ$  }
     { Điểm biểu diễn trên đường tròn lượng giác của góc đã cho thuộc phần tư thứ I }
    { \True Số điểm biểu diễn trên đường tròn lượng giác của góc $-375^\circ +k36^\circ,k \in \mathbb{Z}$ là ${10}$ }
\loigiai{ 
 

 a) Khẳng định đã cho là khẳng định đúng.

 Ta có: $-375^\circ=\dfrac{-375.\pi}{180}=- \frac{25 \pi}{12}$.

b) Khẳng định đã cho là khẳng định sai.

 Vì: $-375^\circ=-15^\circ -1.360^\circ$

c) Khẳng định đã cho là khẳng định sai.

 Vì: $-375^\circ=-15^\circ -1.360^\circ$ nên điểm biểu diễn của góc đã cho thuộc phần tư thứ IV.

d) Khẳng định đã cho là khẳng định đúng.

 $-375^\circ +k36^\circ=-375^\circ +\dfrac{k360^\circ}{10}$ nên có ${10}$ điểm biểu diễn trên đường tròn lượng giác.

 
 }\end{ex}

\begin{ex}
 Cho góc lượng giác $-585^\circ$. Xét tính đúng-sai của các khẳng định sau
\choiceTFt
{ \True $-585^\circ=- \frac{13 \pi}{4}$ }
   { Góc lượng giác đã cho có cùng điểm biểu diễn trên đường tròn lượng giác với góc $-135^\circ$  }
     { \True Điểm biểu diễn trên đường tròn lượng giác của góc đã cho thuộc phần tư thứ II }
    { Số điểm biểu diễn trên đường tròn lượng giác của góc $-585^\circ +k72^\circ,k \in \mathbb{Z}$ là ${6}$ }
\loigiai{ 
 

 a) Khẳng định đã cho là khẳng định đúng.

 Ta có: $-585^\circ=\dfrac{-585.\pi}{180}=- \frac{13 \pi}{4}$.

b) Khẳng định đã cho là khẳng định sai.

 Vì: $-585^\circ=135^\circ -2.360^\circ$

c) Khẳng định đã cho là khẳng định đúng.

 Vì: $-585^\circ=135^\circ -2.360^\circ$ nên điểm biểu diễn của góc đã cho thuộc phần tư thứ II.

d) Khẳng định đã cho là khẳng định sai.

 $-585^\circ +k72^\circ=-585^\circ +\dfrac{k360^\circ}{5}$ nên có ${5}$ điểm biểu diễn trên đường tròn lượng giác.

 
 }\end{ex}

\begin{ex}
 Cho góc lượng giác $-855^\circ$. Xét tính đúng-sai của các khẳng định sau
\choiceTFt
{ $-855^\circ=- \frac{169 \pi}{36}$  }
   { \True Góc lượng giác đã cho có cùng điểm biểu diễn trên đường tròn lượng giác với góc $-135^\circ$ }
     { Điểm biểu diễn trên đường tròn lượng giác của góc đã cho thuộc phần tư thứ I }
    { Số điểm biểu diễn trên đường tròn lượng giác của góc $-855^\circ +k36^\circ,k \in \mathbb{Z}$ là ${13}$ }
\loigiai{ 
 

 a) Khẳng định đã cho là khẳng định sai.

 Ta có: $-855^\circ=\dfrac{-855.\pi}{180}=- \frac{19 \pi}{4}$.

b) Khẳng định đã cho là khẳng định đúng.

 Vì: $-855^\circ=-135^\circ -2.360^\circ$

c) Khẳng định đã cho là khẳng định sai.

 Vì: $-855^\circ=-135^\circ -2.360^\circ$ nên điểm biểu diễn của góc đã cho thuộc phần tư thứ III.

d) Khẳng định đã cho là khẳng định sai.

 $-855^\circ +k36^\circ=-855^\circ +\dfrac{k360^\circ}{10}$ nên có ${10}$ điểm biểu diễn trên đường tròn lượng giác.

 
 }\end{ex}

\Closesolutionfile{ans}
{\bf PHẦN III. Câu trắc nghiệm trả lời ngắn.}
\setcounter{ex}{0}
\Opensolutionfile{ans}[ans/ans001-3]
\begin{ex}[10-M2. Cho số vòng quay bánh xe sau t1 giây][Tính góc radian sau khi quay trong t2 giây] \\
 Một bánh xe của một loại xe quay được 8 vòng trong 3 giây. Tính góc theo rađian mà bánh xe quay được trong 10 giây (kết quả làm tròn đến hàng phần chục).\ 
\shortans[4]{167,6}

\loigiai{ 
 Một giây bánh xe quay được số vòng là: $\frac{8}{3}$.

Một vòng quay ứng với góc $2\pi$. Sau ${10}$ giây có $\frac{8}{3}.10=\frac{80}{3}$ vòng quay ứng với góc:

$\frac{80}{3}.2\pi=\frac{160}{3}\pi=167,6$. 
 }\end{ex}

\begin{ex}
 Một bánh xe của một loại xe quay được 11 vòng trong 4 giây. Tính góc theo rađian mà bánh xe quay được trong 6 giây (kết quả làm tròn đến hàng phần chục).\ 
\shortans[4]{103,7}

\loigiai{ 
 Một giây bánh xe quay được số vòng là: $\frac{11}{4}$.

Một vòng quay ứng với góc $2\pi$. Sau ${6}$ giây có $\frac{11}{4}.6=\frac{33}{2}$ vòng quay ứng với góc:

$\frac{33}{2}.2\pi=33\pi=103,7$. 
 }\end{ex}

\begin{ex}
 Một bánh xe của một loại xe quay được 8 vòng trong 5 giây. Tính góc theo rađian mà bánh xe quay được trong 4 giây (kết quả làm tròn đến hàng phần chục).\ 
\shortans[4]{40,2}

\loigiai{ 
 Một giây bánh xe quay được số vòng là: $\frac{8}{5}$.

Một vòng quay ứng với góc $2\pi$. Sau ${4}$ giây có $\frac{8}{5}.4=\frac{32}{5}$ vòng quay ứng với góc:

$\frac{32}{5}.2\pi=\frac{64}{5}\pi=40,2$. 
 }\end{ex}

\begin{ex}
 Một bánh xe của một loại xe quay được 12 vòng trong 3 giây. Tính góc theo rađian mà bánh xe quay được trong 6 giây (kết quả làm tròn đến hàng phần chục).\ 
\shortans[4]{150,8}

\loigiai{ 
 Một giây bánh xe quay được số vòng là: $4$.

Một vòng quay ứng với góc $2\pi$. Sau ${6}$ giây có $4.6=24$ vòng quay ứng với góc:

$24.2\pi=48\pi=150,8$. 
 }\end{ex}

\begin{ex}
 Một bánh xe của một loại xe quay được 5 vòng trong 5 giây. Tính góc theo rađian mà bánh xe quay được trong 7 giây (kết quả làm tròn đến hàng phần chục).\ 
\shortans[4]{44}

\loigiai{ 
 Một giây bánh xe quay được số vòng là: $1$.

Một vòng quay ứng với góc $2\pi$. Sau ${7}$ giây có $1.7=7$ vòng quay ứng với góc:

$7.2\pi=14\pi=44$. 
 }\end{ex}

\begin{ex}
 Một bánh xe của một loại xe quay được 9 vòng trong 5 giây. Tính góc theo rađian mà bánh xe quay được trong 6 giây (kết quả làm tròn đến hàng phần chục).\ 
\shortans[4]{67,9}

\loigiai{ 
 Một giây bánh xe quay được số vòng là: $\frac{9}{5}$.

Một vòng quay ứng với góc $2\pi$. Sau ${6}$ giây có $\frac{9}{5}.6=\frac{54}{5}$ vòng quay ứng với góc:

$\frac{54}{5}.2\pi=\frac{108}{5}\pi=67,9$. 
 }\end{ex}

\begin{ex}[11-M3. Cho số vòng quay bánh xe sau t1 giây]\textbf{[Tính quãng đường đi được sau t2 giây]} \\
 Một bánh xe của một loại xe có bán kính ${57}$ cm và quay được 6 vòng trong 4 giây. Tính độ dài quãng đường (theo đơn vị mét) xe đi được trong 8 giây (kết quả làm tròn đến hàng phần mười). 
\shortans[4]{43}

\loigiai{ 
 Một giây bánh xe quay được số vòng là: $\frac{3}{2}$.

Một vòng quay ứng với quãng đường là $2\pi.\dfrac{57}{100}=2\pi.\frac{57}{100}=\frac{57}{50}\pi$.

Sau ${8}$ giây quãng đường đi được là: $\frac{3}{2}.8.\frac{57}{50}\pi=\frac{342}{25}\pi=43$.

 
 }\end{ex}

\begin{ex}
 Một bánh xe của một loại xe có bán kính ${59}$ cm và quay được 6 vòng trong 5 giây. Tính độ dài quãng đường (theo đơn vị mét) xe đi được trong 9 giây (kết quả làm tròn đến hàng phần mười). 
\shortans[4]{40}

\loigiai{ 
 Một giây bánh xe quay được số vòng là: $\frac{6}{5}$.

Một vòng quay ứng với quãng đường là $2\pi.\dfrac{59}{100}=2\pi.\frac{59}{100}=\frac{59}{50}\pi$.

Sau ${9}$ giây quãng đường đi được là: $\frac{6}{5}.9.\frac{59}{50}\pi=\frac{1593}{125}\pi=40$.

 
 }\end{ex}

\begin{ex}
 Một bánh xe của một loại xe có bán kính ${52}$ cm và quay được 8 vòng trong 6 giây. Tính độ dài quãng đường (theo đơn vị mét) xe đi được trong 5 giây (kết quả làm tròn đến hàng phần mười). 
\shortans[4]{21,8}

\loigiai{ 
 Một giây bánh xe quay được số vòng là: $\frac{4}{3}$.

Một vòng quay ứng với quãng đường là $2\pi.\dfrac{52}{100}=2\pi.\frac{13}{25}=\frac{26}{25}\pi$.

Sau ${5}$ giây quãng đường đi được là: $\frac{4}{3}.5.\frac{26}{25}\pi=\frac{104}{15}\pi=21,8$.

 
 }\end{ex}

\begin{ex}
 Một bánh xe của một loại xe có bán kính ${54}$ cm và quay được 5 vòng trong 4 giây. Tính độ dài quãng đường (theo đơn vị mét) xe đi được trong 8 giây (kết quả làm tròn đến hàng phần mười). 
\shortans[4]{33,9}

\loigiai{ 
 Một giây bánh xe quay được số vòng là: $\frac{5}{4}$.

Một vòng quay ứng với quãng đường là $2\pi.\dfrac{54}{100}=2\pi.\frac{27}{50}=\frac{27}{25}\pi$.

Sau ${8}$ giây quãng đường đi được là: $\frac{5}{4}.8.\frac{27}{25}\pi=\frac{54}{5}\pi=33,9$.

 
 }\end{ex}

\begin{ex}
 Một bánh xe của một loại xe có bán kính ${50}$ cm và quay được 8 vòng trong 4 giây. Tính độ dài quãng đường (theo đơn vị mét) xe đi được trong 6 giây (kết quả làm tròn đến hàng phần mười). 
\shortans[4]{37,7}

\loigiai{ 
 Một giây bánh xe quay được số vòng là: $2$.

Một vòng quay ứng với quãng đường là $2\pi.\dfrac{50}{100}=2\pi.\frac{1}{2}=1\pi$.

Sau ${6}$ giây quãng đường đi được là: $2.6.1\pi=12\pi=37,7$.

 
 }\end{ex}

\begin{ex}
 Một bánh xe của một loại xe có bán kính ${44}$ cm và quay được 11 vòng trong 5 giây. Tính độ dài quãng đường (theo đơn vị mét) xe đi được trong 9 giây (kết quả làm tròn đến hàng phần mười). 
\shortans[4]{54,7}

\loigiai{ 
 Một giây bánh xe quay được số vòng là: $\frac{11}{5}$.

Một vòng quay ứng với quãng đường là $2\pi.\dfrac{44}{100}=2\pi.\frac{11}{25}=\frac{22}{25}\pi$.

Sau ${9}$ giây quãng đường đi được là: $\frac{11}{5}.9.\frac{22}{25}\pi=\frac{2178}{125}\pi=54,7$.

 
 }\end{ex}

\begin{ex}
 \textbf{[12-M3. Cho đường kính của bánh trước và bánh sau, vận tốc n vòng/phút. Tính quãng đường xe đi.]} \\ Một máy kéo nông nghiệp với bánh xe trước có đường kính là ${72}$ cm, bánh xe sau có đường kính là ${178}$ cm,  xe chuyển động với vận tốc không đổi trên một đoạn đường thẳng. Biết rằng vận tốc của bánh xe sau trong chuyển động này là ${12}$ vòng/ phút. Tính quãng đường mà máy kéo đi được (bằng km) trong ${10}$ phút (làm tròn đến hàng phần mười).\\ 
\shortans[4]{0,7}

\loigiai{ 
 Chu vi của bánh sau là: $2\pi.89=178\pi$.

Tức là bánh xe sau đi mỗi vòng được quãng đường có độ dài là $178\pi$.

Vận tốc của bánh xe sau là ${12}$ vòng/ phút nên trong ${10}$ phút bánh xe sau chuyển động được:\[12.10=120\text{(vòng)}.\]

Quãng đường đi được của máy kéo trong ${10}$ phút là quãng đường đi được khi bánh xe sau lăn ${120}$ vòng:

$120.178\pi=21360\pi$ (cm) $=0,2136\pi$ (km) $=0,7$ (km). 
 }\end{ex}

\begin{ex}
 Một máy kéo nông nghiệp với bánh xe trước có đường kính là ${96}$ cm, bánh xe sau có đường kính là ${174}$ cm,  xe chuyển động với vận tốc không đổi trên một đoạn đường thẳng. Biết rằng vận tốc của bánh xe trước trong chuyển động này là ${14}$ vòng/ phút. Tính quãng đường mà máy kéo đi được (bằng km) trong ${11}$ phút (làm tròn đến hàng phần mười).\\ 
\shortans[4]{0,5}

\loigiai{ 
 Chu vi của bánh trước là: $2\pi.48=96\pi$.

Tức là bánh xe trước đi mỗi vòng được quãng đường có độ dài là $96\pi$.

Vận tốc của bánh xe trước là ${14}$ vòng/ phút nên trong ${11}$ phút bánh xe trước chuyển động được:\[14.11=154\text{(vòng)}.\]

Quãng đường đi được của máy kéo trong ${11}$ phút là quãng đường đi được khi bánh xe trước lăn ${154}$ vòng:

$154.96\pi=14784\pi$ (cm) $=0,14784\pi$ (km) $=0,5$ (km). 
 }\end{ex}

\begin{ex}
 Một máy kéo nông nghiệp với bánh xe trước có đường kính là ${94}$ cm, bánh xe sau có đường kính là ${178}$ cm,  xe chuyển động với vận tốc không đổi trên một đoạn đường thẳng. Biết rằng vận tốc của bánh xe sau trong chuyển động này là ${7}$ vòng/ phút. Tính quãng đường mà máy kéo đi được (bằng km) trong ${13}$ phút (làm tròn đến hàng phần mười).\\ 
\shortans[4]{0,5}

\loigiai{ 
 Chu vi của bánh sau là: $2\pi.89=178\pi$.

Tức là bánh xe sau đi mỗi vòng được quãng đường có độ dài là $178\pi$.

Vận tốc của bánh xe sau là ${7}$ vòng/ phút nên trong ${13}$ phút bánh xe sau chuyển động được:\[7.13=91\text{(vòng)}.\]

Quãng đường đi được của máy kéo trong ${13}$ phút là quãng đường đi được khi bánh xe sau lăn ${91}$ vòng:

$91.178\pi=16198\pi$ (cm) $=0,16198\pi$ (km) $=0,5$ (km). 
 }\end{ex}

\begin{ex}
 Một máy kéo nông nghiệp với bánh xe trước có đường kính là ${72}$ cm, bánh xe sau có đường kính là ${162}$ cm,  xe chuyển động với vận tốc không đổi trên một đoạn đường thẳng. Biết rằng vận tốc của bánh xe sau trong chuyển động này là ${5}$ vòng/ phút. Tính quãng đường mà máy kéo đi được (bằng km) trong ${10}$ phút (làm tròn đến hàng phần mười).\\ 
\shortans[4]{0,3}

\loigiai{ 
 Chu vi của bánh sau là: $2\pi.81=162\pi$.

Tức là bánh xe sau đi mỗi vòng được quãng đường có độ dài là $162\pi$.

Vận tốc của bánh xe sau là ${5}$ vòng/ phút nên trong ${10}$ phút bánh xe sau chuyển động được:\[5.10=50\text{(vòng)}.\]

Quãng đường đi được của máy kéo trong ${10}$ phút là quãng đường đi được khi bánh xe sau lăn ${50}$ vòng:

$50.162\pi=8100\pi$ (cm) $=0,081\pi$ (km) $=0,3$ (km). 
 }\end{ex}

\begin{ex}
 Một máy kéo nông nghiệp với bánh xe trước có đường kính là ${74}$ cm, bánh xe sau có đường kính là ${182}$ cm,  xe chuyển động với vận tốc không đổi trên một đoạn đường thẳng. Biết rằng vận tốc của bánh xe sau trong chuyển động này là ${4}$ vòng/ phút. Tính quãng đường mà máy kéo đi được (bằng km) trong ${14}$ phút (làm tròn đến hàng phần mười).\\ 
\shortans[4]{0,3}

\loigiai{ 
 Chu vi của bánh sau là: $2\pi.91=182\pi$.

Tức là bánh xe sau đi mỗi vòng được quãng đường có độ dài là $182\pi$.

Vận tốc của bánh xe sau là ${4}$ vòng/ phút nên trong ${14}$ phút bánh xe sau chuyển động được:\[4.14=56\text{(vòng)}.\]

Quãng đường đi được của máy kéo trong ${14}$ phút là quãng đường đi được khi bánh xe sau lăn ${56}$ vòng:

$56.182\pi=10192\pi$ (cm) $=0,10192\pi$ (km) $=0,3$ (km). 
 }\end{ex}

\begin{ex}
 Một máy kéo nông nghiệp với bánh xe trước có đường kính là ${100}$ cm, bánh xe sau có đường kính là ${176}$ cm,  xe chuyển động với vận tốc không đổi trên một đoạn đường thẳng. Biết rằng vận tốc của bánh xe sau trong chuyển động này là ${6}$ vòng/ phút. Tính quãng đường mà máy kéo đi được (bằng km) trong ${11}$ phút (làm tròn đến hàng phần mười).\\ 
\shortans[4]{0,4}

\loigiai{ 
 Chu vi của bánh sau là: $2\pi.88=176\pi$.

Tức là bánh xe sau đi mỗi vòng được quãng đường có độ dài là $176\pi$.

Vận tốc của bánh xe sau là ${6}$ vòng/ phút nên trong ${11}$ phút bánh xe sau chuyển động được:\[6.11=66\text{(vòng)}.\]

Quãng đường đi được của máy kéo trong ${11}$ phút là quãng đường đi được khi bánh xe sau lăn ${66}$ vòng:

$66.176\pi=11616\pi$ (cm) $=0,11616\pi$ (km) $=0,4$ (km). 
 }\end{ex}

\Closesolutionfile{ans}

 \begin{center}
-----HẾT-----
\end{center}

 %\newpage 
%\begin{center}
%{\bf BẢNG ĐÁP ÁN MÃ ĐỀ 1 }
%\end{center}
%{\bf Phần 1 }
% \inputansbox{6}{ans001-1}
%{\bf Phần 2 }
% \inputansbox{2}{ans001-2}
%{\bf Phần 3 }
% \inputansbox{6}{ans001-3}
\newpage 



\end{document}