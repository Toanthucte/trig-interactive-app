\documentclass[12pt,a4paper]{article}
\usepackage[top=1.5cm, bottom=1.5cm, left=2.0cm, right=1.5cm] {geometry}
\usepackage{amsmath,amssymb,fontawesome}
\usepackage{tkz-euclide}
\usepackage{setspace}
\usepackage{lastpage}

\usepackage{tikz,tkz-tab}
%\usepackage[solcolor]{ex_test}
%\usepackage[dethi]{ex_test} % Chỉ hiển thị đề thi
\usepackage[loigiai]{ex_test} % Hiển thị lời giải
%\usepackage[color]{ex_test} % Khoanh các đáp án
\everymath{\displaystyle}

\def\colorEX{\color{purple}}
%\def\colorEX{}%Không tô màu đáp án đúng trong tùy chọn loigiai
\renewtheorem{ex}{\color{violet}Câu}
\renewcommand{\FalseEX}{\stepcounter{dapan}{{\bf \textcolor{blue}{\Alph{dapan}.}}}}
\renewcommand{\TrueEX}{\stepcounter{dapan}{{\bf \textcolor{blue}{\Alph{dapan}.}}}}

%---------- Khai báo viết tắt, in đáp án
\newcommand{\hoac}[1]{ %hệ hoặc
    \left[\begin{aligned}#1\end{aligned}\right.}
\newcommand{\heva}[1]{ %hệ và
    \left\{\begin{aligned}#1\end{aligned}\right.}

%Tiêu đề
\newcommand{\tenso}{THÀNH PHỐ HỒ CHÍ MINH}
\newcommand{\tentruong}{HD-NQQ}
\newcommand{\tenkythi}{CÔNG THỨC LƯỢNG GIÁC}
\newcommand{\tenmonthi}{Môn thi: D11-C1-B3 - BT 01-07}
\newcommand{\thoigian}{}
\newcommand{\tieude}[1]{
   \begin{tabular}{cm{1cm}cm{3cm}cm{3cm}}
    {\bf \tenso} & & {\bf \tenkythi} \\
    {\bf \tentruong} & & {\bf \tenmonthi}\\
    && {\bf Thời gian: \bf \thoigian \, phút}\\
    && { \fbox{\bf Mã đề: #1}}
   \end{tabular}\\\\
    
   {Họ tên HS: \dotfill Số báo danh \dotfill}\\
}
\newcommand{\chantrang}[2]{\rfoot{Trang \thepage $-$ Mã đề #2}}
\pagestyle{fancy}
\fancyhf{}
\renewcommand{\headrulewidth}{0pt} 
\renewcommand{\footrulewidth}{0pt}
\usetikzlibrary{shapes.geometric,arrows,calc,intersections,angles,quotes,patterns,snakes,positioning}

\begin{document}
%Thiết lập giãn dọng 1.5cm 
%\setlength{\lineskip}{1.5em}
%Nội dung trắc nghiệm bắt đầu ở đây


\tieude{001}
%\chantrang{\pageref{LastPage}}{001}
\setcounter{page}{1}
{\bf PHẦN I. Câu trắc nghiệm nhiều phương án lựa chọn.}
\setcounter{ex}{0}
\Opensolutionfile{ans}[ans/ans001-1]
\begin{ex}[M2. Tìm khẳng định đúng về hai góc đối nhau]
 Cho ${3\beta}$ là góc lượng giác. Tìm khẳng định đúng trong các khẳng định sau.\ 
\choice
{ $\sin (-3\beta)=\sin 3\beta$ }
   { \True $\cos (-3\beta)=\cos 3\beta$ }
     { $\tan (-3\beta)=\cot 3\beta$ }
    { $\cos (-3\beta)=\sin 3\beta$ }
\loigiai{ 
 $\cos (-3\beta)=\cos 3\beta$ là khẳng định đúng. 
 }\end{ex}

\begin{ex}
 Cho ${4a}$ là góc lượng giác. Tìm khẳng định đúng trong các khẳng định sau.\ 
\choice
{ $\cot (-4a)=\tan 4a$ }
   { $\sin (-4a)=\sin 4a$ }
     { $\cos (-4a)=-\cos 4a$ }
    { \True $\sin (-4a)=-\sin 4a$ }
\loigiai{ 
 $\sin (-4a)=-\sin 4a$ là khẳng định đúng. 
 }\end{ex}

\begin{ex}
 Cho ${\alpha}$ là góc lượng giác. Tìm khẳng định đúng trong các khẳng định sau.\ 
\choice
{ $\cos (-\alpha)=\sin \alpha$ }
   { $\sin (-\alpha)=\cos \alpha$ }
     { $\tan (-\alpha)=\tan \alpha$ }
    { \True $\tan (-\alpha)=-\tan \alpha$ }
\loigiai{ 
 $\tan (-\alpha)=-\tan \alpha$ là khẳng định đúng. 
 }\end{ex}

\begin{ex}
 Cho ${\beta}$ là góc lượng giác. Tìm khẳng định đúng trong các khẳng định sau.\ 
\choice
{ $\cos (-\beta)=-\cos \beta$ }
   { $\sin (-\beta)=\cos \beta$ }
     { $\cot (-\beta)=\tan \beta$ }
    { \True $\cot (-\beta)=-\cot \beta$ }
\loigiai{ 
 $\cot (-\beta)=-\cot \beta$ là khẳng định đúng. 
 }\end{ex}

\begin{ex}
 Cho ${\beta}$ là góc lượng giác. Tìm khẳng định đúng trong các khẳng định sau.\ 
\choice
{ $\tan (-\beta)=\cot \beta$ }
   { $\cos (-\beta)=\sin \beta$ }
     { $\sin (-\beta)=\sin \beta$ }
    { \True $\tan (-\beta)=-\tan \beta$ }
\loigiai{ 
 $\tan (-\beta)=-\tan \beta$ là khẳng định đúng. 
 }\end{ex}

\begin{ex}
 Cho ${x}$ là góc lượng giác. Tìm khẳng định đúng trong các khẳng định sau.\ 
\choice
{ \True $\cot (-x)=-\cot x$ }
   { $\cos (-x)=-\cos x$ }
     { $\tan (-x)=\tan x$ }
    { $\sin (-x)=\sin x$ }
\loigiai{ 
 $\cot (-x)=-\cot x$ là khẳng định đúng. 
 }\end{ex}

\begin{ex}[M2. Tìm khẳng định đúng về hai góc bù nhau]
 Cho ${x}$ là góc lượng giác. Tìm khẳng định đúng trong các khẳng định sau.\ 
\choice
{ $\sin (\pi-x)=\cos x$ }
   { \True $\tan (\pi-x)=-\tan x$ }
     { $\cos (\pi-x)=\cos x$ }
    { $\tan (\pi-x)=\tan x$ }
\loigiai{ 
 $\tan (\pi-x)=-\tan x$ là khẳng định đúng. 
 }\end{ex}

\begin{ex}
 Cho ${4a}$ là góc lượng giác. Tìm khẳng định đúng trong các khẳng định sau.\ 
\choice
{ $\tan (\pi-4a)=\cot 4a$ }
   { $\cos (\pi-4a)=\cos 4a$ }
     { \True $\cos (\pi-4a)=-\cos 4a$ }
    { $\sin (\pi-4a)=\cos 4a$ }
\loigiai{ 
 $\cos (\pi-4a)=-\cos 4a$ là khẳng định đúng. 
 }\end{ex}

\begin{ex}
 Cho ${\beta}$ là góc lượng giác. Tìm khẳng định đúng trong các khẳng định sau.\ 
\choice
{ $\cot (\pi-\beta)=\tan \beta$ }
   { $\cos (\pi-\beta)=\cos \beta$ }
     { \True $\tan (\pi-\beta)=-\tan \beta$ }
    { $\sin (\pi-\beta)=\cos \beta$ }
\loigiai{ 
 $\tan (\pi-\beta)=-\tan \beta$ là khẳng định đúng. 
 }\end{ex}

\begin{ex}
 Cho ${3\beta}$ là góc lượng giác. Tìm khẳng định đúng trong các khẳng định sau.\ 
\choice
{ $\cot (\pi-3\beta)=\cot 3\beta$ }
   { \True $\cos (\pi-3\beta)=-\cos 3\beta$ }
     { $\sin (\pi-3\beta)=\cos 3\beta$ }
    { $\cos (\pi-3\beta)=\cos 3\beta$ }
\loigiai{ 
 $\cos (\pi-3\beta)=-\cos 3\beta$ là khẳng định đúng. 
 }\end{ex}

\begin{ex}
 Cho ${3a}$ là góc lượng giác. Tìm khẳng định đúng trong các khẳng định sau.\ 
\choice
{ \True $\tan (\pi-3a)=-\tan 3a$ }
   { $\cot (\pi-3a)=\tan 3a$ }
     { $\cos (\pi-3a)=\cos 3a$ }
    { $\sin (\pi-3a)=-\sin 3a$ }
\loigiai{ 
 $\tan (\pi-3a)=-\tan 3a$ là khẳng định đúng. 
 }\end{ex}

\begin{ex}
 Cho ${x}$ là góc lượng giác. Tìm khẳng định đúng trong các khẳng định sau.\ 
\choice
{ \True $\cot (\pi-x)=-\cot x$ }
   { $\cot (\pi-x)=\cot x$ }
     { $\sin (\pi-x)=-\sin x$ }
    { $\cos (\pi-x)=\sin x$ }
\loigiai{ 
 $\cot (\pi-x)=-\cot x$ là khẳng định đúng. 
 }\end{ex}

\begin{ex}[M2. Tìm khẳng định đúng về hai góc phụ nhau]
 Cho ${a}$ là góc lượng giác. Tìm khẳng định đúng trong các khẳng định sau.\ 
\choice
{ \True $\cos \left(\frac{\pi}{2}-a\right)=\sin a$ }
   { $\cot \left(\frac{\pi}{2}-a\right)=\cot a$ }
     { $\sin \left(\frac{\pi}{2}-a\right)=-\sin a$ }
    { $\sin \left(\frac{\pi}{2}-a\right)=\sin a$ }
\loigiai{ 
 $\cos \left(\frac{\pi}{2}-a\right)=\sin a$ là khẳng định đúng. 
 }\end{ex}

\begin{ex}
 Cho ${a}$ là góc lượng giác. Tìm khẳng định đúng trong các khẳng định sau.\ 
\choice
{ $\sin \left(\frac{\pi}{2}-a\right)=-\sin a$ }
   { \True $\cot \left(\frac{\pi}{2}-a\right)=\tan a$ }
     { $\cos \left(\frac{\pi}{2}-a\right)=\cos a$ }
    { $\cot \left(\frac{\pi}{2}-a\right)=-\cot a$ }
\loigiai{ 
 $\cot \left(\frac{\pi}{2}-a\right)=\tan a$ là khẳng định đúng. 
 }\end{ex}

\begin{ex}
 Cho ${x}$ là góc lượng giác. Tìm khẳng định đúng trong các khẳng định sau.\ 
\choice
{ $\sin \left(\frac{\pi}{2}-x\right)=-\sin x$ }
   { $\sin \left(\frac{\pi}{2}-x\right)=\sin x$ }
     { $\tan \left(\frac{\pi}{2}-x\right)=-\tan x$ }
    { \True $\cot \left(\frac{\pi}{2}-x\right)=\tan x$ }
\loigiai{ 
 $\cot \left(\frac{\pi}{2}-x\right)=\tan x$ là khẳng định đúng. 
 }\end{ex}

\begin{ex}
 Cho ${a}$ là góc lượng giác. Tìm khẳng định đúng trong các khẳng định sau.\ 
\choice
{ $\sin \left(\frac{\pi}{2}-a\right)=-\sin a$ }
   { $\cot \left(\frac{\pi}{2}-a\right)=\cot a$ }
     { \True $\cot \left(\frac{\pi}{2}-a\right)=\tan a$ }
    { $\sin \left(\frac{\pi}{2}-a\right)=\sin a$ }
\loigiai{ 
 $\cot \left(\frac{\pi}{2}-a\right)=\tan a$ là khẳng định đúng. 
 }\end{ex}

\begin{ex}
 Cho ${x}$ là góc lượng giác. Tìm khẳng định đúng trong các khẳng định sau.\ 
\choice
{ $\sin \left(\frac{\pi}{2}-x\right)=-\sin x$ }
   { $\tan \left(\frac{\pi}{2}-x\right)=\tan x$ }
     { \True $\sin \left(\frac{\pi}{2}-x\right)=\cos x$ }
    { $\sin \left(\frac{\pi}{2}-x\right)=\sin x$ }
\loigiai{ 
 $\sin \left(\frac{\pi}{2}-x\right)=\cos x$ là khẳng định đúng. 
 }\end{ex}

\begin{ex}
 Cho ${a}$ là góc lượng giác. Tìm khẳng định đúng trong các khẳng định sau.\ 
\choice
{ $\cos \left(\frac{\pi}{2}-a\right)=-\cos a$ }
   { $\cot \left(\frac{\pi}{2}-a\right)=-\cot a$ }
     { $\cos \left(\frac{\pi}{2}-a\right)=\cos a$ }
    { \True $\sin \left(\frac{\pi}{2}-a\right)=\cos a$ }
\loigiai{ 
 $\sin \left(\frac{\pi}{2}-a\right)=\cos a$ là khẳng định đúng. 
 }\end{ex}

\begin{ex}[M2. Tìm khẳng định đúng về hai góc hơn kém pi]
 Cho ${5\beta}$ là góc lượng giác. Tìm khẳng định đúng trong các khẳng định sau.\ 
\choice
{ $\tan (\pi+5\beta)=\cot 5\beta$ }
   { \True $\sin (\pi+5\beta)=-\sin 5\beta$ }
     { $\sin (\pi+5\beta)=\cos 5\beta$ }
    { $\cos (\pi+5\beta)=\sin 5\beta$ }
\loigiai{ 
 $\sin (\pi+5\beta)=-\sin 5\beta$ là khẳng định đúng. 
 }\end{ex}

\begin{ex}
 Cho ${\beta}$ là góc lượng giác. Tìm khẳng định đúng trong các khẳng định sau.\ 
\choice
{ $\cos (\pi+\beta)=\sin \beta$ }
   { $\sin (\pi+\beta)=\sin \beta$ }
     { \True $\sin (\pi+\beta)=-\sin \beta$ }
    { $\tan (\pi+\beta)=\cot \beta$ }
\loigiai{ 
 $\sin (\pi+\beta)=-\sin \beta$ là khẳng định đúng. 
 }\end{ex}

\begin{ex}
 Cho ${3x}$ là góc lượng giác. Tìm khẳng định đúng trong các khẳng định sau.\ 
\choice
{ $\sin (\pi+3x)=\cos 3x$ }
   { $\cos (\pi+3x)=\cos 3x$ }
     { \True $\cos (\pi+3x)=-\cos 3x$ }
    { $\cot (\pi+3x)=-\cot 3x$ }
\loigiai{ 
 $\cos (\pi+3x)=-\cos 3x$ là khẳng định đúng. 
 }\end{ex}

\begin{ex}
 Cho ${\alpha}$ là góc lượng giác. Tìm khẳng định đúng trong các khẳng định sau.\ 
\choice
{ $\sin (\pi+\alpha)=\sin \alpha$ }
   { $\tan (\pi+\alpha)=-\tan \alpha$ }
     { $\cos (\pi+\alpha)=\sin \alpha$ }
    { \True $\tan (\pi+\alpha)=\tan \alpha$ }
\loigiai{ 
 $\tan (\pi+\alpha)=\tan \alpha$ là khẳng định đúng. 
 }\end{ex}

\begin{ex}
 Cho ${2x}$ là góc lượng giác. Tìm khẳng định đúng trong các khẳng định sau.\ 
\choice
{ $\tan (\pi+2x)=-\tan 2x$ }
   { $\cos (\pi+2x)=\cos 2x$ }
     { \True $\cot (\pi+2x)=\cot 2x$ }
    { $\sin (\pi+2x)=\cos 2x$ }
\loigiai{ 
 $\cot (\pi+2x)=\cot 2x$ là khẳng định đúng. 
 }\end{ex}

\begin{ex}
 Cho ${\alpha}$ là góc lượng giác. Tìm khẳng định đúng trong các khẳng định sau.\ 
\choice
{ $\cos (\pi+\alpha)=\sin \alpha$ }
   { \True $\tan (\pi+\alpha)=\tan \alpha$ }
     { $\cot (\pi+\alpha)=-\cot \alpha$ }
    { $\sin (\pi+\alpha)=\sin \alpha$ }
\loigiai{ 
 $\tan (\pi+\alpha)=\tan \alpha$ là khẳng định đúng. 
 }\end{ex}

\begin{ex}[M2. Tìm khẳng định đúng về hai góc hơn kém pi]
 Cho ${x}$ là góc lượng giác. Tìm khẳng định đúng trong các khẳng định sau.\ 
\choice
{ $\cos (\pi+x)=\sin x$ }
   { $\sin (-x)=\cos x$ }
     { \True $\cos \left(\frac{\pi}{2}-x\right)=\sin x$ }
    { $\cot (\pi-x)=\tan x$ }
\loigiai{ 
 $\cos \left(\frac{\pi}{2}-x\right)=\sin x$ là khẳng định đúng. 
 }\end{ex}

\begin{ex}
 Cho ${\gamma}$ là góc lượng giác. Tìm khẳng định đúng trong các khẳng định sau.\ 
\choice
{ $\cos \left(\frac{\pi}{2}-\gamma\right)=-\cos \gamma$ }
   { \True $\cot (\pi+\gamma)=\cot \gamma$ }
     { $\sin \left(\frac{\pi}{2}-\gamma\right)=\sin \gamma$ }
    { $\cot (\pi-\gamma)=\tan \gamma$ }
\loigiai{ 
 $\cot (\pi+\gamma)=\cot \gamma$ là khẳng định đúng. 
 }\end{ex}

\begin{ex}
 Cho ${a}$ là góc lượng giác. Tìm khẳng định đúng trong các khẳng định sau.\ 
\choice
{ \True $\cot (-a)=-\cot a$ }
   { $\sin (\pi-a)=-\sin a$ }
     { $\cos (-a)=\sin a$ }
    { $\cot (\pi-a)=\tan a$ }
\loigiai{ 
 $\cot (-a)=-\cot a$ là khẳng định đúng. 
 }\end{ex}

\begin{ex}
 Cho ${b}$ là góc lượng giác. Tìm khẳng định đúng trong các khẳng định sau.\ 
\choice
{ \True $\tan (\pi-b)=-\tan b$ }
   { $\sin \left(\frac{\pi}{2}-b\right)=-\sin b$ }
     { $\cos (-b)=\sin b$ }
    { $\cot (\pi-b)=\cot b$ }
\loigiai{ 
 $\tan (\pi-b)=-\tan b$ là khẳng định đúng. 
 }\end{ex}

\begin{ex}
 Cho ${\alpha}$ là góc lượng giác. Tìm khẳng định đúng trong các khẳng định sau.\ 
\choice
{ \True $\tan (\pi-\alpha)=-\tan \alpha$ }
   { $\cos (\pi-\alpha)=\sin \alpha$ }
     { $\sin (\pi+\alpha)=\sin \alpha$ }
    { $\cot (\pi-\alpha)=\cot \alpha$ }
\loigiai{ 
 $\tan (\pi-\alpha)=-\tan \alpha$ là khẳng định đúng. 
 }\end{ex}

\begin{ex}
 Cho ${\gamma}$ là góc lượng giác. Tìm khẳng định đúng trong các khẳng định sau.\ 
\choice
{ $\cos \left(\frac{\pi}{2}-\gamma\right)=-\cos \gamma$ }
   { $\cot \left(\frac{\pi}{2}-\gamma\right)=-\cot \gamma$ }
     { \True $\cot (-\gamma)=-\cot \gamma$ }
    { $\cos (-\gamma)=\sin \gamma$ }
\loigiai{ 
 $\cot (-\gamma)=-\cot \gamma$ là khẳng định đúng. 
 }\end{ex}

\begin{ex}[M2. Tìm khẳng định đúng về công thức nhân đôi]
 Cho ${\gamma}$ là góc lượng giác. Tìm khẳng định đúng trong các khẳng định sau.\ 
\choice
{ \True $\cos 2\gamma=2\cos^2 \gamma-1$ }
   { $\cos 2\gamma=2\sin^2 \gamma-1$ }
     { $\tan 2\gamma=\dfrac{2\tan \gamma}{1+\tan^2 \gamma}$ }
    { $\sin 2\gamma=2\sin \gamma$ }
\loigiai{ 
 $\cos 2\gamma=2\cos^2 \gamma-1$ là khẳng định đúng. 
 }\end{ex}

\begin{ex}
 Cho ${x}$ là góc lượng giác. Tìm khẳng định đúng trong các khẳng định sau.\ 
\choice
{ \True $\tan 2x=\dfrac{2\tan x}{1-\tan^2 x}$ }
   { $\sin 2x=2\sin x$ }
     { $\cos 2x=2\sin^2 x-1$ }
    { $\tan 2x=\dfrac{2\tan x}{1+\tan^2 x}$ }
\loigiai{ 
 $\tan 2x=\dfrac{2\tan x}{1-\tan^2 x}$ là khẳng định đúng. 
 }\end{ex}

\begin{ex}
 Cho ${\gamma}$ là góc lượng giác. Tìm khẳng định đúng trong các khẳng định sau.\ 
\choice
{ $\sin 2\gamma=\sin \gamma\cos \gamma$ }
   { \True $\tan 2\gamma=\dfrac{2\tan \gamma}{1-\tan^2 \gamma}$ }
     { $\tan 2\gamma=\dfrac{\tan \gamma}{1-\tan^2 \gamma}$ }
    { $\cos 2\gamma=2\sin^2 \gamma-1$ }
\loigiai{ 
 $\tan 2\gamma=\dfrac{2\tan \gamma}{1-\tan^2 \gamma}$ là khẳng định đúng. 
 }\end{ex}

\begin{ex}
 Cho ${\alpha}$ là góc lượng giác. Tìm khẳng định đúng trong các khẳng định sau.\ 
\choice
{ $\tan 2\alpha=\dfrac{\tan \alpha}{1-\tan^2 \alpha}$ }
   { \True $\cos 2\alpha=1-2\sin^2 \alpha$ }
     { $\cos 2\alpha=2\sin \alpha\cos \alpha$ }
    { $\sin 2\alpha=\sin \alpha+\cos \alpha$ }
\loigiai{ 
 $\cos 2\alpha=1-2\sin^2 \alpha$ là khẳng định đúng. 
 }\end{ex}

\begin{ex}
 Cho ${\beta}$ là góc lượng giác. Tìm khẳng định đúng trong các khẳng định sau.\ 
\choice
{ $\tan 2\beta=\dfrac{\tan \beta}{1-2\tan^2 \beta}$ }
   { $\cos 2\beta=\sin^2 \beta-\cos^2 \beta$ }
     { \True $\cos 2\beta=1-2\sin^2 \beta$ }
    { $\sin 2\beta=\sin \beta\cos \beta$ }
\loigiai{ 
 $\cos 2\beta=1-2\sin^2 \beta$ là khẳng định đúng. 
 }\end{ex}

\begin{ex}
 Cho ${\beta}$ là góc lượng giác. Tìm khẳng định đúng trong các khẳng định sau.\ 
\choice
{ $\cos 2\beta=2\sin^2 \beta-1$ }
   { $\sin 2\beta=2\sin \beta$ }
     { \True $\sin 2\beta=2\sin \beta\cos \beta$ }
    { $\tan 2\beta=\dfrac{2\tan \beta}{1+\tan^2 \beta}$ }
\loigiai{ 
 $\sin 2\beta=2\sin \beta\cos \beta$ là khẳng định đúng. 
 }\end{ex}

\begin{ex}[M2. Tìm khẳng định đúng về công thức cộng]
 Cho ${x,y}$ là các góc lượng giác. Tìm khẳng định đúng trong các khẳng định sau.\ 
\choice
{ \True $\cos (x-y)=\cos x \cos y+\sin x \sin y$ }
   { $\tan (x-y)=\dfrac{\tan x - \tan y } {1-\tan x \tan y }$ }
     { $\cos (x-y)=\cos x -\cos y$ }
    { $\sin (x+y)=\sin x +\sin y$ }
\loigiai{ 
 $\cos (x-y)=\cos x \cos y+\sin x \sin y$ là khẳng định đúng. 
 }\end{ex}

\begin{ex}
 Cho ${a,b}$ là các góc lượng giác. Tìm khẳng định đúng trong các khẳng định sau.\ 
\choice
{ $\tan (a+b)=\dfrac{\tan a + \tan b } {1+\tan a \tan b }$ }
   { \True $\cos (a+b)=\cos a \cos b-\sin a \sin b$ }
     { $\cos (a+b)=\cos a +\cos b$ }
    { $\sin (a+b)=\sin a +\sin b$ }
\loigiai{ 
 $\cos (a+b)=\cos a \cos b-\sin a \sin b$ là khẳng định đúng. 
 }\end{ex}

\begin{ex}
 Cho ${u,v}$ là các góc lượng giác. Tìm khẳng định đúng trong các khẳng định sau.\ 
\choice
{ $\sin (u+v)=\sin u +\sin v$ }
   { \True $\cos (u-v)=\cos u \cos v+\sin u \sin v$ }
     { $\cos (u+v)=\cos u \cos v+\sin u \sin v$ }
    { $\tan (u-v)=\dfrac{\tan u - \tan v } {1-\tan u \tan v }$ }
\loigiai{ 
 $\cos (u-v)=\cos u \cos v+\sin u \sin v$ là khẳng định đúng. 
 }\end{ex}

\begin{ex}
 Cho ${a,b}$ là các góc lượng giác. Tìm khẳng định đúng trong các khẳng định sau.\ 
\choice
{ $\sin (a+b)=\sin a +\sin b$ }
   { \True $\sin (a-b)=\sin a \cos b-\cos a \sin b $ }
     { $\tan (a+b)=\tan a + \tan b$ }
    { $\cos (a+b)=\cos a +\cos b$ }
\loigiai{ 
 $\sin (a-b)=\sin a \cos b-\cos a \sin b $ là khẳng định đúng. 
 }\end{ex}

\begin{ex}
 Cho ${x,y}$ là các góc lượng giác. Tìm khẳng định đúng trong các khẳng định sau.\ 
\choice
{ $\cos (x+y)=\cos x +\cos y$ }
   { \True $\cos (x+y)=\cos x \cos y-\sin x \sin y$ }
     { $\tan (x+y)=\tan x + \tan y$ }
    { $\sin (x-y)=\sin x -\sin y$ }
\loigiai{ 
 $\cos (x+y)=\cos x \cos y-\sin x \sin y$ là khẳng định đúng. 
 }\end{ex}

\begin{ex}
 Cho ${u,v}$ là các góc lượng giác. Tìm khẳng định đúng trong các khẳng định sau.\ 
\choice
{ $\sin (u+v)=\sin u \cos v- \cos u \sin v$ }
   { $\tan (u+v)=\tan u + \tan v$ }
     { \True $\tan (u+v)=\dfrac{\tan u + \tan v } {1-\tan u \tan v }$ }
    { $\cos (u+v)=\cos u +\cos v$ }
\loigiai{ 
 $\tan (u+v)=\dfrac{\tan u + \tan v } {1-\tan u \tan v }$ là khẳng định đúng. 
 }\end{ex}

\Closesolutionfile{ans}

 \begin{center}
-----HẾT-----
\end{center}

 %\newpage 
%\begin{center}
%{\bf BẢNG ĐÁP ÁN MÃ ĐỀ 1 }
%\end{center}
%{\bf Phần 1 }
% \inputansbox{6}{ans001-1}



\end{document}