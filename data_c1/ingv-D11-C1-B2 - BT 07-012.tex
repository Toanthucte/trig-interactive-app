\documentclass[12pt,a4paper]{article}
\usepackage[top=1.5cm, bottom=1.5cm, left=2.0cm, right=1.5cm] {geometry}
\usepackage{amsmath,amssymb,fontawesome}
\usepackage{tkz-euclide}
\usepackage{setspace}
\usepackage{lastpage}

\usepackage{tikz,tkz-tab}
%\usepackage[solcolor]{ex_test}
%\usepackage[dethi]{ex_test} % Chỉ hiển thị đề thi
\usepackage[loigiai]{ex_test} % Hiển thị lời giải
%\usepackage[color]{ex_test} % Khoanh các đáp án
\everymath{\displaystyle}

\def\colorEX{\color{purple}}
%\def\colorEX{}%Không tô màu đáp án đúng trong tùy chọn loigiai
\renewtheorem{ex}{\color{violet}Câu}
\renewcommand{\FalseEX}{\stepcounter{dapan}{{\bf \textcolor{blue}{\Alph{dapan}.}}}}
\renewcommand{\TrueEX}{\stepcounter{dapan}{{\bf \textcolor{blue}{\Alph{dapan}.}}}}

%---------- Khai báo viết tắt, in đáp án
\newcommand{\hoac}[1]{ %hệ hoặc
    \left[\begin{aligned}#1\end{aligned}\right.}
\newcommand{\heva}[1]{ %hệ và
    \left\{\begin{aligned}#1\end{aligned}\right.}

%Tiêu đề
\newcommand{\tenso}{THÀNH PHỐ HỒ CHÍ MINH}
\newcommand{\tentruong}{HD-NQQ}
\newcommand{\tenkythi}{GTLG CỦA MỘT GÓC LƯỢNG GIÁ}
\newcommand{\tenmonthi}{Môn học: D11-C1-B2 - BT 07-12}
\newcommand{\thoigian}{}
\newcommand{\tieude}[1]{
   \begin{tabular}{cm{1cm}cm{3cm}cm{3cm}}
    {\bf \tenso} & & {\bf \tenkythi} \\
    {\bf \tentruong} & & {\bf \tenmonthi}\\
    && {\bf Thời gian: \bf \thoigian \, phút}\\
    && { \fbox{\bf Mã đề: #1}}
   \end{tabular}\\\\
    
   {Họ tên HS: \dotfill Số báo danh \dotfill}\\
}
\newcommand{\chantrang}[2]{\rfoot{Trang \thepage $-$ Mã đề #2}}
\pagestyle{fancy}
\fancyhf{}
\renewcommand{\headrulewidth}{0pt} 
\renewcommand{\footrulewidth}{0pt}
\usetikzlibrary{shapes.geometric,arrows,calc,intersections,angles,quotes,patterns,snakes,positioning}
%\dotlinefull{ex}
\begin{document}
%Thiết lập giãn dọng 1.5cm 
%\setlength{\lineskip}{1.5em}
%Nội dung trắc nghiệm bắt đầu ở đây


\tieude{001}
%\chantrang{\pageref{LastPage}}{001}
\setcounter{page}{1}
{\bf PHẦN I. Câu trắc nghiệm nhiều phương án lựa chọn.}
\setcounter{ex}{0}
\Opensolutionfile{ans}[ans/ans001-1]
\begin{ex}[11-M1. Cho góc x (a<x<b). Tìm khẳng định đúng về dấu của GTLG]
 Cho góc lượng giác ${\beta}$ thỏa mãn $\beta\in \left( \frac{5 \pi}{2};3\pi \right)$. Khẳng định nào sau đây là khẳng định đúng. 
\choice
{ $\cos \beta >0$  }
   { $\sin \beta <0$  }
     { \True $\cos \beta <0$  }
    { $\cot \beta >0$  }
\loigiai{ 
 $\cos \beta <0$  là khẳng định đúng 
 }\end{ex}

\begin{ex}
 Cho góc lượng giác ${\beta}$ thỏa mãn $\beta\in \left( \frac{5 \pi}{2};3\pi \right)$. Khẳng định nào sau đây là khẳng định đúng. 
\choice
{ $\cot \beta >0$  }
   { $\sin \beta <0$  }
     { $\cos \beta >0$  }
    { \True $\cot \beta <0$  }
\loigiai{ 
 $\cot \beta <0$  là khẳng định đúng 
 }\end{ex}

\begin{ex}
 Cho góc lượng giác ${b}$ thỏa mãn $b\in \left( \frac{\pi}{2};\pi \right)$. Khẳng định nào sau đây là khẳng định đúng. 
\choice
{ $\cos b >0$  }
   { \True $\tan b <0$  }
     { $\tan b >0$  }
    { $\sin b <0$  }
\loigiai{ 
 $\tan b <0$  là khẳng định đúng 
 }\end{ex}

\begin{ex}
 Cho góc lượng giác ${x}$ thỏa mãn $x\in \left( - \frac{3 \pi}{2};- \pi \right)$. Khẳng định nào sau đây là khẳng định đúng. 
\choice
{ $\tan x >0$  }
   { $\sin x <0$  }
     { \True $\tan x <0$  }
    { $\cos x >0$  }
\loigiai{ 
 $\tan x <0$  là khẳng định đúng 
 }\end{ex}

\begin{ex}
 Cho góc lượng giác ${\alpha}$ thỏa mãn $\alpha\in \left( 2\pi;\frac{5 \pi}{2} \right)$. Khẳng định nào sau đây là khẳng định đúng. 
\choice
{ $\tan \alpha <0$  }
   { $\cot \alpha <0$  }
     { \True $\tan \alpha >0$  }
    { $\cos \alpha <0$  }
\loigiai{ 
 $\tan \alpha >0$  là khẳng định đúng 
 }\end{ex}

\begin{ex}
 Cho góc lượng giác ${\beta}$ thỏa mãn $\beta\in \left( 0;\frac{\pi}{2} \right)$. Khẳng định nào sau đây là khẳng định đúng. 
\choice
{ $\cot \beta <0$  }
   { $\sin \beta <0$  }
     { \True $\sin \beta >0$  }
    { $\cos \beta <0$  }
\loigiai{ 
 $\sin \beta >0$  là khẳng định đúng 
 }\end{ex}

\begin{ex}[12-M2. Cho sinx (hoặc cosx), x thuộc (a;b). Tìm cosx (hoặc sinx)]
 Cho góc lượng giác $\alpha$ thỏa mãn $\sin \alpha=- \frac{2}{3}, \alpha \in \left( \pi;\frac{3 \pi}{2} \right)$. Tính $\cos\alpha$. 
\choice
{ ${\frac{\sqrt{3}}{4}}$ }
   { ${\frac{\sqrt{15}}{9}}$ }
     { ${\frac{\sqrt{2}}{18}}$ }
    { \True ${- \frac{\sqrt{5}}{3}}$ }
\loigiai{ 
 Vì $\alpha \in \left( \pi;\frac{3 \pi}{2} \right)$ nên $\cos\alpha < 0$.

$\cos\alpha =-\sqrt{1-\frac{4}{9}}=- \frac{\sqrt{5}}{3}$. 
 }\end{ex}

\begin{ex}
 Cho góc lượng giác $\alpha$ thỏa mãn $\cos \alpha=\frac{13}{14}, \alpha \in \left( \frac{7 \pi}{2}; 4\pi \right)$. Tính $\sin\alpha$ (kết quả làm tròn đến hàng phần mười). 
\choice
{ ${\frac{\sqrt{3}}{3}}$ }
   { \True ${- \frac{3 \sqrt{3}}{14}}$ }
     { ${\frac{\sqrt{13}}{11}}$ }
    { ${\frac{2}{13}}$ }
\loigiai{ 
 Vì $\alpha \in \left( \frac{7 \pi}{2}; 4\pi \right)$ nên $\sin\alpha < 0$.

$\sin\alpha =\sqrt{1-\frac{169}{196}}=- \frac{3 \sqrt{3}}{14}$. 
 }\end{ex}

\begin{ex}
 Cho góc lượng giác $\alpha$ thỏa mãn $\cos \alpha=- \frac{3}{4}, \alpha \in \left( - \pi; - \frac{\pi}{2} \right)$. Tính $\sin\alpha$ (kết quả làm tròn đến hàng phần mười). 
\choice
{ ${\frac{\sqrt{14}}{11}}$ }
   { ${\frac{\sqrt{10}}{19}}$ }
     { \True ${- \frac{\sqrt{7}}{4}}$ }
    { ${\frac{\sqrt{7}}{8}}$ }
\loigiai{ 
 Vì $\alpha \in \left( - \pi; - \frac{\pi}{2} \right)$ nên $\sin\alpha < 0$.

$\sin\alpha =\sqrt{1-\frac{9}{16}}=- \frac{\sqrt{7}}{4}$. 
 }\end{ex}

\begin{ex}
 Cho góc lượng giác $\alpha$ thỏa mãn $\cos \alpha=- \frac{10}{13}, \alpha \in \left( - \pi; - \frac{\pi}{2} \right)$. Tính $\sin\alpha$ (kết quả làm tròn đến hàng phần mười). 
\choice
{ ${\frac{3}{11}}$ }
   { \True ${- \frac{\sqrt{69}}{13}}$ }
     { ${\frac{\sqrt{13}}{13}}$ }
    { ${\frac{\sqrt{10}}{7}}$ }
\loigiai{ 
 Vì $\alpha \in \left( - \pi; - \frac{\pi}{2} \right)$ nên $\sin\alpha < 0$.

$\sin\alpha =\sqrt{1-\frac{100}{169}}=- \frac{\sqrt{69}}{13}$. 
 }\end{ex}

\begin{ex}
 Cho góc lượng giác $\alpha$ thỏa mãn $\sin \alpha=\frac{5}{7}, \alpha \in \left( 2\pi;\frac{5 \pi}{2} \right)$. Tính $\cos\alpha$. 
\choice
{ ${\frac{\sqrt{14}}{6}}$ }
   { ${\frac{\sqrt{10}}{15}}$ }
     { \True ${\frac{2 \sqrt{6}}{7}}$ }
    { ${\frac{\sqrt{10}}{20}}$ }
\loigiai{ 
 Vì $\alpha \in \left( 2\pi;\frac{5 \pi}{2} \right)$ nên $\cos\alpha > 0$.

$\cos\alpha =\sqrt{1-\frac{25}{49}}=\frac{2 \sqrt{6}}{7}$. 
 }\end{ex}

\begin{ex}
 Cho góc lượng giác $\alpha$ thỏa mãn $\cos \alpha=\frac{1}{2}, \alpha \in \left( \frac{7 \pi}{2}; 4\pi \right)$. Tính $\sin\alpha$ (kết quả làm tròn đến hàng phần mười). 
\choice
{ ${\frac{\sqrt{13}}{18}}$ }
   { ${\frac{\sqrt{3}}{19}}$ }
     { ${\frac{1}{9}}$ }
    { \True ${- \frac{\sqrt{3}}{2}}$ }
\loigiai{ 
 Vì $\alpha \in \left( \frac{7 \pi}{2}; 4\pi \right)$ nên $\sin\alpha < 0$.

$\sin\alpha =\sqrt{1-\frac{1}{4}}=- \frac{\sqrt{3}}{2}$. 
 }\end{ex}

\begin{ex}[13-M2.Cho sinx (hoặc cosx), x thuộc (a;b). Tìm tanx (hoặc cotx)]
 Cho góc lượng giác $\alpha$ thỏa mãn $\sin \alpha=\frac{6}{7}, \alpha \in \left( - \frac{3 \pi}{2};- \pi \right)$. Tính $\tan\alpha$. 
\choice
{ \True $- \frac{6 \sqrt{13}}{13}$ }
   { $\frac{\sqrt{5}}{10}$ }
     { $\frac{2}{5}$ }
    { $\frac{\sqrt{10}}{9}$ }
\loigiai{ 
 Vì $\alpha \in \left( - \frac{3 \pi}{2};- \pi \right)$ nên $\cos\alpha < 0$.

$\cos\alpha =-\sqrt{1-\frac{36}{49}}=- \frac{\sqrt{13}}{7}$.

$\tan\alpha=\frac{6}{7}:- \frac{\sqrt{13}}{7}=- \frac{6 \sqrt{13}}{13}$.

 
 }\end{ex}

\begin{ex}
 Cho góc lượng giác $\alpha$ thỏa mãn $\cos \alpha=\frac{4}{5}, \alpha \in \left( \frac{7 \pi}{2}; 4\pi \right)$. Tính $\tan\alpha$ (kết quả làm tròn đến hàng phần mười). 
\choice
{ $\frac{2 \sqrt{3}}{13}$ }
   { $\frac{\sqrt{2}}{14}$ }
     { \True $- \frac{3}{4}$ }
    { $\frac{\sqrt{10}}{8}$ }
\loigiai{ 
 Vì $\alpha \in \left( \frac{7 \pi}{2}; 4\pi \right)$ nên $\sin\alpha < 0$.

$\sin\alpha =\sqrt{1-\frac{16}{25}}=- \frac{3}{5}$.

$\tan\alpha=- \frac{3}{5}:0.8=- \frac{3}{4}$.

 
 }\end{ex}

\begin{ex}
 Cho góc lượng giác $\alpha$ thỏa mãn $\sin \alpha=\frac{13}{15}, \alpha \in \left( \frac{5 \pi}{2};3\pi \right)$. Tính $\tan\alpha$. 
\choice
{ $\frac{\sqrt{11}}{7}$ }
   { \True $- \frac{13 \sqrt{14}}{28}$ }
     { $\frac{\sqrt{6}}{7}$ }
    { $\frac{2}{9}$ }
\loigiai{ 
 Vì $\alpha \in \left( \frac{5 \pi}{2};3\pi \right)$ nên $\cos\alpha < 0$.

$\cos\alpha =-\sqrt{1-\frac{169}{225}}=- \frac{2 \sqrt{14}}{15}$.

$\tan\alpha=\frac{13}{15}:- \frac{2 \sqrt{14}}{15}=- \frac{13 \sqrt{14}}{28}$.

 
 }\end{ex}

\begin{ex}
 Cho góc lượng giác $\alpha$ thỏa mãn $\cos \alpha=\frac{13}{16}, \alpha \in \left( \frac{7 \pi}{2}; 4\pi \right)$. Tính $\cot\alpha$ (kết quả làm tròn đến hàng phần mười). 
\choice
{ $\frac{1}{16}$ }
   { $\frac{3}{11}$ }
     { \True $- \frac{13 \sqrt{87}}{87}$ }
    { $\frac{\sqrt{11}}{15}$ }
\loigiai{ 
 Vì $\alpha \in \left( \frac{7 \pi}{2}; 4\pi \right)$ nên $\sin\alpha < 0$.

$\sin\alpha =\sqrt{1-\frac{169}{256}}=- \frac{\sqrt{87}}{16}$.

$\cot\alpha=0.8125:- \frac{\sqrt{87}}{16}=- \frac{13 \sqrt{87}}{87}$.

 
 }\end{ex}

\begin{ex}
 Cho góc lượng giác $\alpha$ thỏa mãn $\sin \alpha=\frac{13}{14}, \alpha \in \left( - \frac{3 \pi}{2};- \pi \right)$. Tính $\tan\alpha$. 
\choice
{ $\frac{\sqrt{3}}{15}$ }
   { \True $- \frac{13 \sqrt{3}}{9}$ }
     { $\frac{\sqrt{3}}{2}$ }
    { $\frac{\sqrt{7}}{14}$ }
\loigiai{ 
 Vì $\alpha \in \left( - \frac{3 \pi}{2};- \pi \right)$ nên $\cos\alpha < 0$.

$\cos\alpha =-\sqrt{1-\frac{169}{196}}=- \frac{3 \sqrt{3}}{14}$.

$\tan\alpha=\frac{13}{14}:- \frac{3 \sqrt{3}}{14}=- \frac{13 \sqrt{3}}{9}$.

 
 }\end{ex}

\begin{ex}
 Cho góc lượng giác $\alpha$ thỏa mãn $\cos \alpha=- \frac{1}{2}, \alpha \in \left( 3 \pi;\frac{7 \pi}{2} \right)$. Tính $\cot\alpha$ (kết quả làm tròn đến hàng phần mười). 
\choice
{ $\frac{\sqrt{2}}{9}$ }
   { \True $\frac{\sqrt{3}}{3}$ }
     { $\frac{\sqrt{3}}{7}$ }
    { $\frac{2}{7}$ }
\loigiai{ 
 Vì $\alpha \in \left( 3 \pi;\frac{7 \pi}{2} \right)$ nên $\sin\alpha < 0$.

$\sin\alpha =\sqrt{1-\frac{1}{4}}=- \frac{\sqrt{3}}{2}$.

$\cot\alpha=-0.5:- \frac{\sqrt{3}}{2}=\frac{\sqrt{3}}{3}$.

 
 }\end{ex}

\Closesolutionfile{ans}
{\bf PHẦN II. Câu trắc nghiệm đúng sai.}
\setcounter{ex}{0}
\Opensolutionfile{ans}[ans/ans001-2]
\begin{ex}[07-M3. Cho tanx. Xét Đ-S: cotx, $\sin^2 x$, $\cos^2 x$, (asinx+bcosx)/(csinx+dcosx)]
 Cho góc lượng giác ${x}$ thỏa mãn $\tan x=2$. Xét tính đúng-sai của các khẳng định sau. 
\choiceTFt
{ $\cot x= \frac{7}{2}$  }
   { \True $\cos^2x=\frac{1}{5}$ }
     { $\sin^2x=\frac{6}{5}$ }
    { $P=\frac{- \cos x + 4 \sin x}{- \cos x + \sin x}=10$ }
\loigiai{ 
 

 a) Khẳng định đã cho là khẳng định sai.

 $\cot x=\dfrac{1}{\tan x}=1:2=\frac{1}{2}$

b) Khẳng định đã cho là khẳng định đúng.

 $1+\tan^2x=\dfrac{1}{\cos^2x} \Rightarrow \cos^2x=\dfrac{1}{1+\tan^2x}=\frac{1}{5}$

c) Khẳng định đã cho là khẳng định sai.

 $\sin^2x=1-\cos^2x=1-\dfrac{1}{1+\tan^2x}=\frac{4}{5}$

d) Khẳng định đã cho là khẳng định sai.

 $P=\frac{- \cos x + 4 \sin x}{- \cos x + \sin x}=\frac{4 \tan x - 1}{\tan x - 1}=7$.

 
 }\end{ex}

\begin{ex}
 Cho góc lượng giác ${x}$ thỏa mãn $\tan x=6$. Xét tính đúng-sai của các khẳng định sau. 
\choiceTFt
{ \True $\cot x= \frac{1}{6}$ }
   { $\cos^2x=\frac{1}{7}$  }
     { \True $\sin^2x=\frac{36}{37}$ }
    { $P=\frac{- 6 \cos x + 5 \sin x}{- 3 \cos x - 6 \sin x}=\frac{5}{13}$ }
\loigiai{ 
 

 a) Khẳng định đã cho là khẳng định đúng.

 $\cot x=\dfrac{1}{\tan x}=1:6=\frac{1}{6}$

b) Khẳng định đã cho là khẳng định sai.

 $1+\tan^2x=\dfrac{1}{\cos^2x} \Rightarrow \cos^2x=\dfrac{1}{1+\tan^2x}=\frac{1}{37}$

c) Khẳng định đã cho là khẳng định đúng.

 $\sin^2x=1-\cos^2x=1-\dfrac{1}{1+\tan^2x}=\frac{36}{37}$

d) Khẳng định đã cho là khẳng định sai.

 $P=\frac{- 6 \cos x + 5 \sin x}{- 3 \cos x - 6 \sin x}=\frac{5 \tan x - 6}{- 6 \tan x - 3}=- \frac{8}{13}$.

 
 }\end{ex}

\begin{ex}
 Cho góc lượng giác ${x}$ thỏa mãn $\tan x=3$. Xét tính đúng-sai của các khẳng định sau. 
\choiceTFt
{ \True $\cot x= \frac{1}{3}$ }
   { $\cos^2x=\frac{1}{4}$  }
     { \True $\sin^2x=\frac{9}{10}$ }
    { $P=\frac{2 \cos x + 3 \sin x}{- \cos x + 3 \sin x}=\frac{19}{8}$ }
\loigiai{ 
 

 a) Khẳng định đã cho là khẳng định đúng.

 $\cot x=\dfrac{1}{\tan x}=1:3=\frac{1}{3}$

b) Khẳng định đã cho là khẳng định sai.

 $1+\tan^2x=\dfrac{1}{\cos^2x} \Rightarrow \cos^2x=\dfrac{1}{1+\tan^2x}=\frac{1}{10}$

c) Khẳng định đã cho là khẳng định đúng.

 $\sin^2x=1-\cos^2x=1-\dfrac{1}{1+\tan^2x}=\frac{9}{10}$

d) Khẳng định đã cho là khẳng định sai.

 $P=\frac{2 \cos x + 3 \sin x}{- \cos x + 3 \sin x}=\frac{3 \tan x + 2}{3 \tan x - 1}=\frac{11}{8}$.

 
 }\end{ex}

\begin{ex}
 Cho góc lượng giác ${x}$ thỏa mãn $\tan x=-5$. Xét tính đúng-sai của các khẳng định sau. 
\choiceTFt
{ $\cot x= \frac{14}{5}$  }
   { $\cos^2x=- \frac{1}{4}$  }
     { \True $\sin^2x=\frac{25}{26}$ }
    { $P=\frac{6 \cos x + \sin x}{4 \cos x + \sin x}=0$ }
\loigiai{ 
 

 a) Khẳng định đã cho là khẳng định sai.

 $\cot x=\dfrac{1}{\tan x}=1:-5=- \frac{1}{5}$

b) Khẳng định đã cho là khẳng định sai.

 $1+\tan^2x=\dfrac{1}{\cos^2x} \Rightarrow \cos^2x=\dfrac{1}{1+\tan^2x}=\frac{1}{26}$

c) Khẳng định đã cho là khẳng định đúng.

 $\sin^2x=1-\cos^2x=1-\dfrac{1}{1+\tan^2x}=\frac{25}{26}$

d) Khẳng định đã cho là khẳng định sai.

 $P=\frac{6 \cos x + \sin x}{4 \cos x + \sin x}=\frac{\tan x + 6}{\tan x + 4}=-1$.

 
 }\end{ex}

\begin{ex}
 Cho góc lượng giác ${x}$ thỏa mãn $\tan x=-6$. Xét tính đúng-sai của các khẳng định sau. 
\choiceTFt
{ \True $\cot x= - \frac{1}{6}$ }
   { \True $\cos^2x=\frac{1}{37}$ }
     { \True $\sin^2x=\frac{36}{37}$ }
    { $P=\frac{5 \cos x + 4 \sin x}{- 4 \cos x - 5 \sin x}=\frac{59}{26}$ }
\loigiai{ 
 

 a) Khẳng định đã cho là khẳng định đúng.

 $\cot x=\dfrac{1}{\tan x}=1:-6=- \frac{1}{6}$

b) Khẳng định đã cho là khẳng định đúng.

 $1+\tan^2x=\dfrac{1}{\cos^2x} \Rightarrow \cos^2x=\dfrac{1}{1+\tan^2x}=\frac{1}{37}$

c) Khẳng định đã cho là khẳng định đúng.

 $\sin^2x=1-\cos^2x=1-\dfrac{1}{1+\tan^2x}=\frac{36}{37}$

d) Khẳng định đã cho là khẳng định sai.

 $P=\frac{5 \cos x + 4 \sin x}{- 4 \cos x - 5 \sin x}=\frac{4 \tan x + 5}{- 5 \tan x - 4}=- \frac{19}{26}$.

 
 }\end{ex}

\begin{ex}
 Cho góc lượng giác ${x}$ thỏa mãn $\tan x=6$. Xét tính đúng-sai của các khẳng định sau. 
\choiceTFt
{ \True $\cot x= \frac{1}{6}$ }
   { $\cos^2x=\frac{1}{7}$  }
     { $\sin^2x=\frac{38}{37}$ }
    { \True $P=\frac{2 \cos x + \sin x}{3 \cos x - 4 \sin x}=- \frac{8}{21}$ }
\loigiai{ 
 

 a) Khẳng định đã cho là khẳng định đúng.

 $\cot x=\dfrac{1}{\tan x}=1:6=\frac{1}{6}$

b) Khẳng định đã cho là khẳng định sai.

 $1+\tan^2x=\dfrac{1}{\cos^2x} \Rightarrow \cos^2x=\dfrac{1}{1+\tan^2x}=\frac{1}{37}$

c) Khẳng định đã cho là khẳng định sai.

 $\sin^2x=1-\cos^2x=1-\dfrac{1}{1+\tan^2x}=\frac{36}{37}$

d) Khẳng định đã cho là khẳng định đúng.

 $P=\frac{2 \cos x + \sin x}{3 \cos x - 4 \sin x}=\frac{\tan x + 2}{3 - 4 \tan x}=- \frac{8}{21}$.

 
 }\end{ex}

\begin{ex}[08-M3. Cho sinx (a<x<b). Xét Đ-S: dấu của cosx, cosx, sin(x+kpi/2), P=f(tanx)]
 Cho $\sin x=\frac{1}{2}, x\in \left( \frac{5 \pi}{2};3\pi \right)$. Xét tính đúng-sai của các khẳng định sau.
\choiceTFt
{ \True $\cos x <0$ }
   { \True $\cos x=- \frac{\sqrt{3}}{2}$ }
     { $\cos\left(x+ \frac{9 \pi}{2} \right)=\frac{1}{2}$  }
    { \True $P=\frac{2 \sqrt{3} \tan x}{2 \tan^2 x - 1}=6$ }
\loigiai{ 
 

 a) Khẳng định đã cho là khẳng định đúng.

 Với $x \in \left( \frac{5 \pi}{2};3\pi \right) $ thì $\cos x <0$

b) Khẳng định đã cho là khẳng định đúng.

 Vì $x\in \left( \frac{5 \pi}{2};3\pi \right)$ nên $\cos x < 0$.

$\cos x=-\sqrt{1-\frac{1}{4}}=- \frac{\sqrt{3}}{2}$.

c) Khẳng định đã cho là khẳng định sai.

 $\cos\left(x+ \frac{9 \pi}{2} \right)=\cos \left( x+\frac{\pi}{2}+2.2\pi \right)=\cos \left(x+\frac{\pi}{2}\right)=\cos \left[ \frac{\pi}{2}-(-x) \right]=\sin (-x)=-\sin x=-0.5$.

d) Khẳng định đã cho là khẳng định đúng.

 $\tan x=\frac{1}{2}:- \frac{\sqrt{3}}{2}=- \frac{\sqrt{3}}{3}$.

$\Rightarrow P=\frac{2 \sqrt{3} \tan x}{2 \tan^2 x - 1}=6$.

 
 }\end{ex}

\begin{ex}
 Cho $\sin x=\frac{3}{5}, x\in \left( - \frac{3 \pi}{2};- \pi \right)$. Xét tính đúng-sai của các khẳng định sau.
\choiceTFt
{ $\cos x >0$ }
   { \True $\cos x=- \frac{4}{5}$ }
     { \True $\cos\left(x+ \frac{13 \pi}{2} \right)=-\frac{3}{5}$ }
    { \True $P=\frac{\sqrt{3} \tan x}{3 \tan^2 x - 4}=\frac{12 \sqrt{3}}{37}$ }
\loigiai{ 
 

 a) Khẳng định đã cho là khẳng định sai.

 Với $x \in \left( - \frac{3 \pi}{2};- \pi \right) $ thì $\cos x <0$

b) Khẳng định đã cho là khẳng định đúng.

 Vì $x\in \left( - \frac{3 \pi}{2};- \pi \right)$ nên $\cos x < 0$.

$\cos x=-\sqrt{1-\frac{9}{25}}=- \frac{4}{5}$.

c) Khẳng định đã cho là khẳng định đúng.

 $\cos\left(x+ \frac{13 \pi}{2} \right)=\cos \left( x+\frac{\pi}{2}+3.2\pi \right)=\cos \left(x+\frac{\pi}{2}\right)=\cos \left[ \frac{\pi}{2}-(-x) \right]=\sin (-x)=-\sin x=-0.6$.

d) Khẳng định đã cho là khẳng định đúng.

 $\tan x=\frac{3}{5}:- \frac{4}{5}=- \frac{3}{4}$.

$\Rightarrow P=\frac{\sqrt{3} \tan x}{3 \tan^2 x - 4}=\frac{12 \sqrt{3}}{37}$.

 
 }\end{ex}

\begin{ex}
 Cho $\sin x=\frac{\sqrt{5}}{8}, x\in \left( 0;\frac{\pi}{2} \right)$. Xét tính đúng-sai của các khẳng định sau.
\choiceTFt
{ $\cos x <0$ }
   { \True $\cos x=\frac{\sqrt{59}}{8}$ }
     { \True $\cos\left(x+ \frac{9 \pi}{2} \right)=-\frac{\sqrt{5}}{8}$ }
    { \True $P=\frac{\sqrt{7} \tan x}{3 \tan^2 x + 3}=\frac{\sqrt{2065}}{192}$ }
\loigiai{ 
 

 a) Khẳng định đã cho là khẳng định sai.

 Với $x \in \left( 0;\frac{\pi}{2} \right) $ thì $\cos x >0$

b) Khẳng định đã cho là khẳng định đúng.

 Vì $x \in \left( 0;\frac{\pi}{2} \right)$ nên $\cos x > 0$.

$\cos x =\sqrt{1-\frac{5}{64}}=\frac{\sqrt{59}}{8}$.

c) Khẳng định đã cho là khẳng định đúng.

 $\cos\left(x+ \frac{9 \pi}{2} \right)=\cos \left( x+\frac{\pi}{2}+2.2\pi \right)=\cos \left(x+\frac{\pi}{2}\right)=\cos \left[ \frac{\pi}{2}-(-x) \right]=\sin (-x)=-\sin x=- \frac{\sqrt{5}}{8}$.

d) Khẳng định đã cho là khẳng định đúng.

 $\tan x=\frac{\sqrt{5}}{8}:\frac{\sqrt{59}}{8}=\frac{\sqrt{295}}{59}$.

$\Rightarrow P=\frac{\sqrt{7} \tan x}{3 \tan^2 x + 3}=\frac{\sqrt{2065}}{192}$.

 
 }\end{ex}

\begin{ex}
 Cho $\sin x=- \frac{5}{7}, x\in \left( \frac{7 \pi}{2}; 4\pi \right)$. Xét tính đúng-sai của các khẳng định sau.
\choiceTFt
{ \True $\cos x >0$ }
   { \True $\cos x=\frac{2 \sqrt{6}}{7}$ }
     { $\cos\left(x+ \frac{17 \pi}{2} \right)=- \frac{5}{7}$  }
    { $P=\frac{\sqrt{14} \tan x}{- 5 \tan^2 x - 5}=- \frac{4 \sqrt{21}}{49}$ }
\loigiai{ 
 

 a) Khẳng định đã cho là khẳng định đúng.

 Với $x \in \left( \frac{7 \pi}{2}; 4\pi \right) $ thì $\cos x >0$

b) Khẳng định đã cho là khẳng định đúng.

 Vì $x \in \left( \frac{7 \pi}{2}; 4\pi \right)$ nên $\cos x > 0$.

$\cos x=\sqrt{1-\frac{25}{49}}=\frac{2 \sqrt{6}}{7}$.

c) Khẳng định đã cho là khẳng định sai.

 $\cos\left(x+ \frac{17 \pi}{2} \right)=\cos \left( x+\frac{\pi}{2}+4.2\pi \right)=\cos \left(x+\frac{\pi}{2}\right)=\cos \left[ \frac{\pi}{2}-(-x) \right]=\sin (-x)=-\sin x=0.714285714285714$.

d) Khẳng định đã cho là khẳng định sai.

 $\tan x=- \frac{5}{7}:\frac{2 \sqrt{6}}{7}=- \frac{5 \sqrt{6}}{12}$.

$\Rightarrow P=\frac{\sqrt{14} \tan x}{- 5 \tan^2 x - 5}=\frac{4 \sqrt{21}}{49}$.

 
 }\end{ex}

\begin{ex}
 Cho $\sin x=- \frac{\sqrt{7}}{8}, x\in \left( 3 \pi;\frac{7 \pi}{2} \right)$. Xét tính đúng-sai của các khẳng định sau.
\choiceTFt
{ \True $\cos x <0$ }
   { \True $\cos x=- \frac{\sqrt{57}}{8}$ }
     { \True $\cos\left(x+ \frac{17 \pi}{2} \right)= \frac{\sqrt{7}}{8}$ }
    { $P=\frac{\sqrt{3} \tan x}{3 \tan^2 x - 2}=\frac{\sqrt{133}}{31}$ }
\loigiai{ 
 

 a) Khẳng định đã cho là khẳng định đúng.

 Với $x \in \left( 3 \pi;\frac{7 \pi}{2} \right) $ thì $\cos x <0$

b) Khẳng định đã cho là khẳng định đúng.

 Vì $x \in \left( 3 \pi;\frac{7 \pi}{2} \right)$ nên $\cos x < 0$.

$\cos x=-\sqrt{1-\frac{7}{64}}=- \frac{\sqrt{57}}{8}$.

c) Khẳng định đã cho là khẳng định đúng.

 $\cos\left(x+ \frac{17 \pi}{2} \right)=\cos \left( x+\frac{\pi}{2}+4.2\pi \right)=\cos \left(x+\frac{\pi}{2}\right)=\cos \left[ \frac{\pi}{2}-(-x) \right]=\sin (-x)=-\sin x=\frac{\sqrt{7}}{8}$.

d) Khẳng định đã cho là khẳng định sai.

 $\tan x=- \frac{\sqrt{7}}{8}:- \frac{\sqrt{57}}{8}=\frac{\sqrt{399}}{57}$.

$\Rightarrow P=\frac{\sqrt{3} \tan x}{3 \tan^2 x - 2}=- \frac{\sqrt{133}}{31}$.

 
 }\end{ex}

\begin{ex}
 Cho $\sin x=\frac{\sqrt{3}}{4}, x\in \left( \frac{\pi}{2};\pi \right)$. Xét tính đúng-sai của các khẳng định sau.
\choiceTFt
{ \True $\cos x <0$ }
   { $\cos x=\frac{\sqrt{13}}{4}$ }
     { \True $\cos\left(x+ \frac{9 \pi}{2} \right)=-\frac{\sqrt{3}}{4}$ }
    { \True $P=\frac{\sqrt{14} \tan x}{- 3 \tan^2 x - 1}=\frac{\sqrt{546}}{22}$ }
\loigiai{ 
 

 a) Khẳng định đã cho là khẳng định đúng.

 Với $x \in \left( \frac{\pi}{2};\pi \right) $ thì $\cos x <0$

b) Khẳng định đã cho là khẳng định sai.

 Vì $x\in \left( \frac{\pi}{2};\pi \right)$ nên $\cos x < 0$.

$\cos x=-\sqrt{1-\frac{3}{16}}=- \frac{\sqrt{13}}{4}$.

c) Khẳng định đã cho là khẳng định đúng.

 $\cos\left(x+ \frac{9 \pi}{2} \right)=\cos \left( x+\frac{\pi}{2}+2.2\pi \right)=\cos \left(x+\frac{\pi}{2}\right)=\cos \left[ \frac{\pi}{2}-(-x) \right]=\sin (-x)=-\sin x=- \frac{\sqrt{3}}{4}$.

d) Khẳng định đã cho là khẳng định đúng.

 $\tan x=\frac{\sqrt{3}}{4}:- \frac{\sqrt{13}}{4}=- \frac{\sqrt{39}}{13}$.

$\Rightarrow P=\frac{\sqrt{14} \tan x}{- 3 \tan^2 x - 1}=\frac{\sqrt{546}}{22}$.

 
 }\end{ex}

\begin{ex}[09-M3. Cho cosx (a<x<b). Xét Đ-S: dấu của sinx][ sinx, sin(x+kpi/2), P=f(tanx)] \\
 Cho $\cos x=- \frac{\sqrt{214}}{15}, x\in \left( \pi;\frac{3 \pi}{2} \right)$. Xét tính đúng-sai của các khẳng định sau.
\choiceTFt
{ \True $\sin x <0$ }
   { $\sin x= \frac{\sqrt{11}}{15}$ }
     { \True $\sin\left(x+ \frac{13 \pi}{2} \right)=- \frac{\sqrt{214}}{15}$ }
    { $P=\frac{\tan x}{1 - 2 \tan^2 x}=- \frac{\sqrt{2354}}{192}$ }
\loigiai{ 
 

 a) Khẳng định đã cho là khẳng định đúng.

 Với $x \in \left( \pi;\frac{3 \pi}{2} \right) $ thì $\sin x >0$

b) Khẳng định đã cho là khẳng định sai.

 Vì $x \in \left( \pi;\frac{3 \pi}{2} \right)$ nên $\sin x > 0$.

$\sin x =-\sqrt{1-\frac{214}{225}}=- \frac{\sqrt{11}}{15}$.

c) Khẳng định đã cho là khẳng định đúng.

 $\sin\left(x+ \frac{13 \pi}{2} \right)=\sin \left( x+\frac{\pi}{2}+3.2\pi \right)=\sin\left(x+\frac{\pi}{2}\right)=\sin \left[ \frac{\pi}{2}-(-x) \right]=\cos (-x)=\cos x=- \frac{\sqrt{214}}{15}$.

d) Khẳng định đã cho là khẳng định sai.

 $\tan x=- \frac{\sqrt{11}}{15}:- \frac{\sqrt{214}}{15}=\frac{\sqrt{2354}}{214}$.

$\Rightarrow P=\frac{\tan x}{1 - 2 \tan^2 x}=\frac{\sqrt{2354}}{192}$.

 
 }\end{ex}

\begin{ex}
 Cho $\cos x=- \frac{4 \sqrt{2}}{9}, x\in \left( - \pi; - \frac{\pi}{2} \right)$. Xét tính đúng-sai của các khẳng định sau.
\choiceTFt
{ \True $\sin x <0$ }
   { \True $\sin x=- \frac{7}{9}$ }
     { $\cos\left(x+ \frac{9 \pi}{2} \right)=- \frac{7}{9}$  }
    { $P=\frac{\sqrt{5} \tan x}{- 3 \tan^2 x - 4}=\frac{28 \sqrt{10}}{275}$ }
\loigiai{ 
 

 a) Khẳng định đã cho là khẳng định đúng.

 Với $x \in \left( - \pi; - \frac{\pi}{2} \right) $ thì $\sin x >0$

b) Khẳng định đã cho là khẳng định đúng.

 Vì $x \in \left( - \pi; - \frac{\pi}{2} \right)$ nên $\sin x > 0$.

$\sin x =-\sqrt{1-\frac{32}{81}}=- \frac{7}{9}$.

c) Khẳng định đã cho là khẳng định sai.

 $\cos\left(x+ \frac{9 \pi}{2} \right)=\cos \left( x+\frac{\pi}{2}+2.2\pi \right)=\cos \left(x+\frac{\pi}{2}\right)=\cos \left[ \frac{\pi}{2}-(-x) \right]=\sin (-x)=-\sin x=0.777777777777778$.

d) Khẳng định đã cho là khẳng định sai.

 $\tan x=- \frac{7}{9}:- \frac{4 \sqrt{2}}{9}=\frac{7 \sqrt{2}}{8}$.

$\Rightarrow P=\frac{\sqrt{5} \tan x}{- 3 \tan^2 x - 4}=- \frac{28 \sqrt{10}}{275}$.

 
 }\end{ex}

\begin{ex}
 Cho $\cos x=- \frac{\sqrt{159}}{13}, x\in \left( \pi;\frac{3 \pi}{2} \right)$. Xét tính đúng-sai của các khẳng định sau.
\choiceTFt
{ $\sin x >0$ }
   { \True $\sin x=- \frac{\sqrt{10}}{13}$ }
     { $\cos\left(x+ \frac{13 \pi}{2} \right)=- \frac{\sqrt{10}}{13}$  }
    { $P=\frac{\sqrt{3} \tan x}{1 - 2 \tan^2 x}=- \frac{3 \sqrt{530}}{139}$ }
\loigiai{ 
 

 a) Khẳng định đã cho là khẳng định sai.

 Với $x \in \left( \pi;\frac{3 \pi}{2} \right) $ thì $\sin x >0$

b) Khẳng định đã cho là khẳng định đúng.

 Vì $x \in \left( \pi;\frac{3 \pi}{2} \right)$ nên $\sin x > 0$.

$\sin x =-\sqrt{1-\frac{159}{169}}=- \frac{\sqrt{10}}{13}$.

c) Khẳng định đã cho là khẳng định sai.

 $\cos\left(x+ \frac{13 \pi}{2} \right)=\cos \left( x+\frac{\pi}{2}+3.2\pi \right)=\cos \left(x+\frac{\pi}{2}\right)=\cos \left[ \frac{\pi}{2}-(-x) \right]=\sin (-x)=-\sin x=\frac{\sqrt{10}}{13}$.

d) Khẳng định đã cho là khẳng định sai.

 $\tan x=- \frac{\sqrt{10}}{13}:- \frac{\sqrt{159}}{13}=\frac{\sqrt{1590}}{159}$.

$\Rightarrow P=\frac{\sqrt{3} \tan x}{1 - 2 \tan^2 x}=\frac{3 \sqrt{530}}{139}$.

 
 }\end{ex}

\begin{ex}
 Cho $\cos x=- \frac{\sqrt{33}}{7}, x\in \left( 3 \pi;\frac{7 \pi}{2} \right)$. Xét tính đúng-sai của các khẳng định sau.
\choiceTFt
{ \True $\sin x <0$ }
   { $\sin x= \frac{4}{7}$ }
     { $\sin\left(x+ \frac{17 \pi}{2} \right)=\frac{\sqrt{33}}{7}$  }
    { \True $P=\frac{\sqrt{3} \tan x}{\tan^2 x + 3}=\frac{12 \sqrt{11}}{115}$ }
\loigiai{ 
 

 a) Khẳng định đã cho là khẳng định đúng.

 Với $x \in \left( 3 \pi;\frac{7 \pi}{2} \right) $ thì $\sin x >0$

b) Khẳng định đã cho là khẳng định sai.

 Vì $x \in \left( 3 \pi;\frac{7 \pi}{2} \right)$ nên $\sin x > 0$.

$\sin x =-\sqrt{1-\frac{33}{49}}=- \frac{4}{7}$.

c) Khẳng định đã cho là khẳng định sai.

 $\sin\left(x+ \frac{17 \pi}{2} \right)=\sin \left( x+\frac{\pi}{2}+4.2\pi \right)=\sin\left(x+\frac{\pi}{2}\right)=\sin \left[ \frac{\pi}{2}-(-x) \right]=\cos (-x)=\cos x=- \frac{\sqrt{33}}{7}$.

d) Khẳng định đã cho là khẳng định đúng.

 $\tan x=- \frac{4}{7}:- \frac{\sqrt{33}}{7}=\frac{4 \sqrt{33}}{33}$.

$\Rightarrow P=\frac{\sqrt{3} \tan x}{\tan^2 x + 3}=\frac{12 \sqrt{11}}{115}$.

 
 }\end{ex}

\begin{ex}
 Cho $\cos x=- \frac{\sqrt{7}}{3}, x\in \left( 3 \pi;\frac{7 \pi}{2} \right)$. Xét tính đúng-sai của các khẳng định sau.
\choiceTFt
{ $\sin x >0$ }
   { $\sin x= \frac{\sqrt{2}}{3}$ }
     { $\cos\left(x+ \frac{9 \pi}{2} \right)=- \frac{\sqrt{2}}{3}$  }
    { \True $P=\frac{2 \sqrt{3} \tan x}{5 \tan^2 x - 3}=- \frac{2 \sqrt{42}}{11}$ }
\loigiai{ 
 

 a) Khẳng định đã cho là khẳng định sai.

 Với $x \in \left( 3 \pi;\frac{7 \pi}{2} \right) $ thì $\sin x >0$

b) Khẳng định đã cho là khẳng định sai.

 Vì $x \in \left( 3 \pi;\frac{7 \pi}{2} \right)$ nên $\sin x > 0$.

$\sin x =-\sqrt{1-\frac{7}{9}}=- \frac{\sqrt{2}}{3}$.

c) Khẳng định đã cho là khẳng định sai.

 $\cos\left(x+ \frac{9 \pi}{2} \right)=\cos \left( x+\frac{\pi}{2}+2.2\pi \right)=\cos \left(x+\frac{\pi}{2}\right)=\cos \left[ \frac{\pi}{2}-(-x) \right]=\sin (-x)=-\sin x=\frac{\sqrt{2}}{3}$.

d) Khẳng định đã cho là khẳng định đúng.

 $\tan x=- \frac{\sqrt{2}}{3}:- \frac{\sqrt{7}}{3}=\frac{\sqrt{14}}{7}$.

$\Rightarrow P=\frac{2 \sqrt{3} \tan x}{5 \tan^2 x - 3}=- \frac{2 \sqrt{42}}{11}$.

 
 }\end{ex}

\begin{ex}
 Cho $\cos x=- \frac{\sqrt{34}}{6}, x\in \left( \pi;\frac{3 \pi}{2} \right)$. Xét tính đúng-sai của các khẳng định sau.
\choiceTFt
{ \True $\sin x <0$ }
   { \True $\sin x=- \frac{\sqrt{2}}{6}$ }
     { $\sin\left(x+ \frac{5 \pi}{2} \right)=\frac{\sqrt{34}}{6}$  }
    { $P=\frac{\sqrt{11} \tan x}{- 3 \tan^2 x - 4}=\frac{\sqrt{187}}{71}$ }
\loigiai{ 
 

 a) Khẳng định đã cho là khẳng định đúng.

 Với $x \in \left( \pi;\frac{3 \pi}{2} \right) $ thì $\sin x >0$

b) Khẳng định đã cho là khẳng định đúng.

 Vì $x \in \left( \pi;\frac{3 \pi}{2} \right)$ nên $\sin x > 0$.

$\sin x =-\sqrt{1-\frac{17}{18}}=- \frac{\sqrt{2}}{6}$.

c) Khẳng định đã cho là khẳng định sai.

 $\sin\left(x+ \frac{5 \pi}{2} \right)=\sin \left( x+\frac{\pi}{2}+1.2\pi \right)=\sin\left(x+\frac{\pi}{2}\right)=\sin \left[ \frac{\pi}{2}-(-x) \right]=\cos (-x)=\cos x=- \frac{\sqrt{34}}{6}$.

d) Khẳng định đã cho là khẳng định sai.

 $\tan x=- \frac{\sqrt{2}}{6}:- \frac{\sqrt{34}}{6}=\frac{\sqrt{17}}{17}$.

$\Rightarrow P=\frac{\sqrt{11} \tan x}{- 3 \tan^2 x - 4}=- \frac{\sqrt{187}}{71}$.

 
 }\end{ex}

\Closesolutionfile{ans}

 \begin{center}
-----HẾT-----
\end{center}

 %\newpage 
%\begin{center}
%{\bf BẢNG ĐÁP ÁN MÃ ĐỀ 1 }
%\end{center}
%{\bf Phần 1 }
% \inputansbox{6}{ans001-1}
%{\bf Phần 2 }
% \inputansbox{2}{ans001-2}



\end{document}