\documentclass[12pt,a4paper]{article}
\usepackage[top=1.5cm, bottom=1.5cm, left=2.0cm, right=1.5cm] {geometry}
\usepackage{amsmath,amssymb,fontawesome}
\usepackage{tkz-euclide}
\usepackage{setspace}
\usepackage{lastpage}

\usepackage{tikz,tkz-tab}
%\usepackage[solcolor]{ex_test}
%\usepackage[dethi]{ex_test} % Chỉ hiển thị đề thi
\usepackage[loigiai]{ex_test} % Hiển thị lời giải
%\usepackage[color]{ex_test} % Khoanh các đáp án
\everymath{\displaystyle}

\def\colorEX{\color{purple}}
%\def\colorEX{}%Không tô màu đáp án đúng trong tùy chọn loigiai
\renewtheorem{ex}{\color{violet}Câu}
\renewcommand{\FalseEX}{\stepcounter{dapan}{{\bf \textcolor{blue}{\Alph{dapan}.}}}}
\renewcommand{\TrueEX}{\stepcounter{dapan}{{\bf \textcolor{blue}{\Alph{dapan}.}}}}

%---------- Khai báo viết tắt, in đáp án
\newcommand{\hoac}[1]{ %hệ hoặc
    \left[\begin{aligned}#1\end{aligned}\right.}
\newcommand{\heva}[1]{ %hệ và
    \left\{\begin{aligned}#1\end{aligned}\right.}

%Tiêu đề
\newcommand{\tenso}{THÀNH PHỐ HỒ CHÍ MINH}
\newcommand{\tentruong}{HD-NQQ}
\newcommand{\tenkythi}{GÓC LƯỢNG GIÁC }
\newcommand{\tenmonthi}{Môn : D11-C1-B1 - BT 14-19}
\newcommand{\thoigian}{}
\newcommand{\tieude}[1]{
   \begin{tabular}{cm{1cm}cm{3cm}cm{3cm}}
    {\bf \tenso} & & {\bf \tenkythi} \\
    {\bf \tentruong} & & {\bf \tenmonthi}\\
    && {\bf Thời gian: \bf \thoigian \, phút}\\
    && { \fbox{\bf Mã đề: #1}}
   \end{tabular}\\\\
    
   {Họ tên HS: \dotfill Số báo danh \dotfill}\\
}
\newcommand{\chantrang}[2]{\rfoot{Trang \thepage $-$ Mã đề #2}}
\pagestyle{fancy}
\fancyhf{}
\renewcommand{\headrulewidth}{0pt} 
\renewcommand{\footrulewidth}{0pt}
\usetikzlibrary{shapes.geometric,arrows,calc,intersections,angles,quotes,patterns,snakes,positioning}

\begin{document}
%Thiết lập giãn dọng 1.5cm 
%\setlength{\lineskip}{1.5em}
%Nội dung trắc nghiệm bắt đầu ở đây


\tieude{001}
%\chantrang{\pageref{LastPage}}{001}
\setcounter{page}{1}
{\bf PHẦN I. Câu trắc nghiệm nhiều phương án lựa chọn.}
\setcounter{ex}{0}
\Opensolutionfile{ans}[ans/ans001-1]
\begin{ex}[14-M2. Tìm tất cả các góc có cùng điểm biểu diễn với x và -x]
 Trên đường tròn lượng giác, cho các góc lượng giác ${\frac{\pi}{4}}$ và ${- \frac{\pi}{4}}$ có điểm biểu diễn lần lượt là ${M}$ và ${N}$. Góc lượng giác nào sau đây cũng chỉ có đúng hai điểm biểu diễn là ${M}$ và ${N}$? 
\choice
{ \True $\frac{\pi}{4}+k\pi, k\in \mathbb{Z}$ }
   { $\frac{\pi}{4}+k4\pi, k\in \mathbb{Z}$ }
     { $\frac{\pi}{4}+\dfrac{k\pi }{3}, k\in \mathbb{Z}$ }
    { $\frac{\pi}{4}+\dfrac{k\pi }{2}, k\in \mathbb{Z}$ }
\loigiai{ 
 $\frac{\pi}{4}+k\pi, k\in \mathbb{Z}$ là khẳng định đúng. 
 }\end{ex}

\begin{ex}
 Trên đường tròn lượng giác, cho các góc lượng giác ${\frac{\pi}{2}}$ và ${- \frac{\pi}{2}}$ có điểm biểu diễn lần lượt là ${M}$ và ${N}$. Góc lượng giác nào sau đây cũng chỉ có đúng hai điểm biểu diễn là ${M}$ và ${N}$? 
\choice
{ $\frac{\pi}{2}+\dfrac{k\pi }{3}, k\in \mathbb{Z}$ }
   { $\frac{\pi}{2}+\dfrac{k\pi }{4}, k\in \mathbb{Z}$ }
     { \True $\frac{\pi}{2}+k\pi, k\in \mathbb{Z}$ }
    { $\frac{\pi}{2}+\dfrac{k\pi }{2}, k\in \mathbb{Z}$ }
\loigiai{ 
 $\frac{\pi}{2}+k\pi, k\in \mathbb{Z}$ là khẳng định đúng. 
 }\end{ex}

\begin{ex}
 Trên đường tròn lượng giác, cho các góc lượng giác ${6 \pi}$ và ${- 6 \pi}$ có điểm biểu diễn lần lượt là ${M}$ và ${N}$. Góc lượng giác nào sau đây cũng chỉ có đúng hai điểm biểu diễn là ${M}$ và ${N}$? 
\choice
{ $6 \pi+\dfrac{k\pi }{4}, k\in \mathbb{Z}$ }
   { \True $6 \pi+k\pi, k\in \mathbb{Z}$ }
     { $6 \pi+\dfrac{k\pi }{2}, k\in \mathbb{Z}$ }
    { $6 \pi+k2\pi, k\in \mathbb{Z}$ }
\loigiai{ 
 $6 \pi+k\pi, k\in \mathbb{Z}$ là khẳng định đúng. 
 }\end{ex}

\begin{ex}
 Trên đường tròn lượng giác, cho các góc lượng giác ${\frac{3 \pi}{7}}$ và ${- \frac{3 \pi}{7}}$ có điểm biểu diễn lần lượt là ${M}$ và ${N}$. Góc lượng giác nào sau đây cũng chỉ có đúng hai điểm biểu diễn là ${M}$ và ${N}$? 
\choice
{ $\frac{3 \pi}{7}+\dfrac{k\pi }{2}, k\in \mathbb{Z}$ }
   { $\frac{3 \pi}{7}+\dfrac{k\pi }{3}, k\in \mathbb{Z}$ }
     { \True $\frac{3 \pi}{7}+k\pi, k\in \mathbb{Z}$ }
    { $\frac{3 \pi}{7}+\dfrac{k\pi }{4}, k\in \mathbb{Z}$ }
\loigiai{ 
 $\frac{3 \pi}{7}+k\pi, k\in \mathbb{Z}$ là khẳng định đúng. 
 }\end{ex}

\begin{ex}
 Trên đường tròn lượng giác, cho các góc lượng giác ${\frac{9 \pi}{8}}$ và ${- \frac{9 \pi}{8}}$ có điểm biểu diễn lần lượt là ${M}$ và ${N}$. Góc lượng giác nào sau đây cũng chỉ có đúng hai điểm biểu diễn là ${M}$ và ${N}$? 
\choice
{ \True $\frac{9 \pi}{8}+k\pi, k\in \mathbb{Z}$ }
   { $\frac{9 \pi}{8}+\dfrac{k\pi }{2}, k\in \mathbb{Z}$ }
     { $\frac{9 \pi}{8}+k4\pi, k\in \mathbb{Z}$ }
    { $\frac{9 \pi}{8}+\dfrac{k\pi }{3}, k\in \mathbb{Z}$ }
\loigiai{ 
 $\frac{9 \pi}{8}+k\pi, k\in \mathbb{Z}$ là khẳng định đúng. 
 }\end{ex}

\begin{ex}
 Trên đường tròn lượng giác, cho các góc lượng giác ${\frac{2 \pi}{3}}$ và ${- \frac{2 \pi}{3}}$ có điểm biểu diễn lần lượt là ${M}$ và ${N}$. Góc lượng giác nào sau đây cũng chỉ có đúng hai điểm biểu diễn là ${M}$ và ${N}$? 
\choice
{ \True $\frac{2 \pi}{3}+k\pi, k\in \mathbb{Z}$ }
   { $\frac{2 \pi}{3}+\dfrac{k\pi }{3}, k\in \mathbb{Z}$ }
     { $\frac{2 \pi}{3}+\dfrac{k\pi }{2}, k\in \mathbb{Z}$ }
    { $\frac{2 \pi}{3}+\dfrac{k\pi }{4}, k\in \mathbb{Z}$ }
\loigiai{ 
 $\frac{2 \pi}{3}+k\pi, k\in \mathbb{Z}$ là khẳng định đúng. 
 }\end{ex}

\begin{ex}[15-M2. Tìm số điểm biểu diễn của góc dạng a+k2pi/n]
 Số điểm biểu diễn của góc $x=\frac{4 \pi}{5}+k\frac{\pi}{4}\, (k\in \mathbb{Z})$ trên đường tròn lượng giác là 
\choice
{ ${1}$ }
   { \True ${8}$ }
     { ${5}$ }
    { ${6}$ }
\loigiai{ 
 góc lượng giác $\frac{4 \pi}{5}+k\frac{\pi}{4}$ có ${8}$ điểm biểu diễn. 
 }\end{ex}

\begin{ex}
 Số điểm biểu diễn của góc $\alpha=\frac{3 \pi}{4}+k\frac{2 \pi}{7}\, (k\in \mathbb{Z})$ trên đường tròn lượng giác là 
\choice
{ ${6}$ }
   { ${8}$ }
     { ${5}$ }
    { \True ${7}$ }
\loigiai{ 
 góc lượng giác $\frac{3 \pi}{4}+k\frac{2 \pi}{7}$ có ${7}$ điểm biểu diễn. 
 }\end{ex}

\begin{ex}
 Số điểm biểu diễn của góc $a=\frac{2 \pi}{3}+k\frac{\pi}{3}\, (k\in \mathbb{Z})$ trên đường tròn lượng giác là 
\choice
{ ${9}$ }
   { ${7}$ }
     { ${2}$ }
    { \True ${6}$ }
\loigiai{ 
 góc lượng giác $\frac{2 \pi}{3}+k\frac{\pi}{3}$ có ${6}$ điểm biểu diễn. 
 }\end{ex}

\begin{ex}
 Số điểm biểu diễn của góc $\beta=\frac{5 \pi}{6}+k\frac{\pi}{2}\, (k\in \mathbb{Z})$ trên đường tròn lượng giác là 
\choice
{ \True ${4}$ }
   { ${2}$ }
     { ${7}$ }
    { ${8}$ }
\loigiai{ 
 góc lượng giác $\frac{5 \pi}{6}+k\frac{\pi}{2}$ có ${4}$ điểm biểu diễn. 
 }\end{ex}

\begin{ex}
 Số điểm biểu diễn của góc $x=\frac{7 \pi}{11}+k\frac{\pi}{4}\, (k\in \mathbb{Z})$ trên đường tròn lượng giác là 
\choice
{ ${7}$ }
   { \True ${8}$ }
     { ${5}$ }
    { ${6}$ }
\loigiai{ 
 góc lượng giác $\frac{7 \pi}{11}+k\frac{\pi}{4}$ có ${8}$ điểm biểu diễn. 
 }\end{ex}

\begin{ex}
 Số điểm biểu diễn của góc $x=\frac{3 \pi}{5}+k\frac{\pi}{3}\, (k\in \mathbb{Z})$ trên đường tròn lượng giác là 
\choice
{ ${2}$ }
   { ${9}$ }
     { ${5}$ }
    { \True ${6}$ }
\loigiai{ 
 góc lượng giác $\frac{3 \pi}{5}+k\frac{\pi}{3}$ có ${6}$ điểm biểu diễn. 
 }\end{ex}

\begin{ex}[16-M2. Tìm điểm biểu diễn trên hình ứng với 4 điểm A, A', B, B']
 Trong mặt phẳng tọa độ ${Oxy},$ cho đường tròn lượng giác như hình vẽ bên dưới. Tìm điểm biểu diễn của góc có số đo bằng $7 \pi$. 
\begin{center}
\begin{tikzpicture}[scale=.7,>=stealth, font=\footnotesize, line join=round, line cap=round]
            \draw[fill=black](0,0) coordinate (O) node[below left]{$O$} circle (1.5pt) (2,0) coordinate (A) node[above right] {$A$} circle (1.5pt) (180:2) coordinate (A2) node [above left] {$A'$} circle (1.5 pt) (90:2) coordinate (B1) node [above right] {$B$} circle (1.5pt) (270:2) coordinate (B2) node [below left] {$B'$} circle (1.5pt);
            \draw[very thick] (0,0) circle (2 cm);
            \draw[->] (-3,0)--(3,0) node [below]{$x$};
            \draw[->] (0,-3)--(0,3) node [left]{$y$};
            \clip (-5,-5) rectangle (5,5);
    \end{tikzpicture} 
\end{center}
\choice
{ Điểm ${{B'}}$ }
   { \True Điểm ${{A'}}$ }
     { Điểm ${{A}}$ }
    { Điểm ${{B}}$ }
\loigiai{ 
 Điểm biểu diễn của góc $7 \pi$ là Điểm ${{A'}}$. 
 }\end{ex}

\begin{ex}
 Trong mặt phẳng tọa độ ${Oxy},$ cho đường tròn lượng giác như hình vẽ bên dưới. Tìm điểm biểu diễn của góc có số đo bằng $- \pi$. 
\begin{center}
\begin{tikzpicture}[scale=.7,>=stealth, font=\footnotesize, line join=round, line cap=round]
            \draw[fill=black](0,0) coordinate (O) node[below left]{$O$} circle (1.5pt) (2,0) coordinate (A) node[above right] {$A$} circle (1.5pt) (180:2) coordinate (A2) node [above left] {$A'$} circle (1.5 pt) (90:2) coordinate (B1) node [above right] {$B$} circle (1.5pt) (270:2) coordinate (B2) node [below left] {$B'$} circle (1.5pt);
            \draw[very thick] (0,0) circle (2 cm);
            \draw[->] (-3,0)--(3,0) node [below]{$x$};
            \draw[->] (0,-3)--(0,3) node [left]{$y$};
            \clip (-5,-5) rectangle (5,5);
    \end{tikzpicture} 
\end{center}
\choice
{ \True Điểm ${{A'}}$ }
   { Điểm ${{B}}$ }
     { Điểm ${{A}}$ }
    { Điểm ${{B'}}$ }
\loigiai{ 
 Điểm biểu diễn của góc $- \pi$ là Điểm ${{A'}}$. 
 }\end{ex}

\begin{ex}
 Trong mặt phẳng tọa độ ${Oxy},$ cho đường tròn lượng giác như hình vẽ bên dưới. Tìm điểm biểu diễn của góc có số đo bằng $- \frac{\pi}{2}$. 
\begin{center}
\begin{tikzpicture}[scale=.7,>=stealth, font=\footnotesize, line join=round, line cap=round]
            \draw[fill=black](0,0) coordinate (O) node[below left]{$O$} circle (1.5pt) (2,0) coordinate (A) node[above right] {$A$} circle (1.5pt) (180:2) coordinate (A2) node [above left] {$A'$} circle (1.5 pt) (90:2) coordinate (B1) node [above right] {$B$} circle (1.5pt) (270:2) coordinate (B2) node [below left] {$B'$} circle (1.5pt);
            \draw[very thick] (0,0) circle (2 cm);
            \draw[->] (-3,0)--(3,0) node [below]{$x$};
            \draw[->] (0,-3)--(0,3) node [left]{$y$};
            \clip (-5,-5) rectangle (5,5);
    \end{tikzpicture} 
\end{center}
\choice
{ Điểm ${{A}}$ }
   { Điểm ${{B}}$ }
     { Điểm ${{A'}}$ }
    { \True ("Điểm ${{B'}}$",) }
\loigiai{ 
 Điểm biểu diễn của góc $- \frac{\pi}{2}$ là ("Điểm ${{B'}}$",). 
 }\end{ex}

\begin{ex}
 Trong mặt phẳng tọa độ ${Oxy},$ cho đường tròn lượng giác như hình vẽ bên dưới. Tìm điểm biểu diễn của góc có số đo bằng $2 \pi$. 
\begin{center}
\begin{tikzpicture}[scale=.7,>=stealth, font=\footnotesize, line join=round, line cap=round]
            \draw[fill=black](0,0) coordinate (O) node[below left]{$O$} circle (1.5pt) (2,0) coordinate (A) node[above right] {$A$} circle (1.5pt) (180:2) coordinate (A2) node [above left] {$A'$} circle (1.5 pt) (90:2) coordinate (B1) node [above right] {$B$} circle (1.5pt) (270:2) coordinate (B2) node [below left] {$B'$} circle (1.5pt);
            \draw[very thick] (0,0) circle (2 cm);
            \draw[->] (-3,0)--(3,0) node [below]{$x$};
            \draw[->] (0,-3)--(0,3) node [left]{$y$};
            \clip (-5,-5) rectangle (5,5);
    \end{tikzpicture} 
\end{center}
\choice
{ Điểm ${{B}}$ }
   { Điểm ${{A'}}$ }
     { \True Điểm ${{A}}$ }
    { Điểm ${{B'}}$ }
\loigiai{ 
 Điểm biểu diễn của góc $2 \pi$ là Điểm ${{A}}$. 
 }\end{ex}

\begin{ex}
 Trong mặt phẳng tọa độ ${Oxy},$ cho đường tròn lượng giác như hình vẽ bên dưới. Tìm điểm biểu diễn của góc có số đo bằng $7 \pi$. 
\begin{center}
\begin{tikzpicture}[scale=.7,>=stealth, font=\footnotesize, line join=round, line cap=round]
            \draw[fill=black](0,0) coordinate (O) node[below left]{$O$} circle (1.5pt) (2,0) coordinate (A) node[above right] {$A$} circle (1.5pt) (180:2) coordinate (A2) node [above left] {$A'$} circle (1.5 pt) (90:2) coordinate (B1) node [above right] {$B$} circle (1.5pt) (270:2) coordinate (B2) node [below left] {$B'$} circle (1.5pt);
            \draw[very thick] (0,0) circle (2 cm);
            \draw[->] (-3,0)--(3,0) node [below]{$x$};
            \draw[->] (0,-3)--(0,3) node [left]{$y$};
            \clip (-5,-5) rectangle (5,5);
    \end{tikzpicture} 
\end{center}
\choice
{ Điểm ${{A}}$ }
   { Điểm ${{B}}$ }
     { \True Điểm ${{A'}}$ }
    { Điểm ${{B'}}$ }
\loigiai{ 
 Điểm biểu diễn của góc $7 \pi$ là Điểm ${{A'}}$. 
 }\end{ex}

\begin{ex}
 Trong mặt phẳng tọa độ ${Oxy},$ cho đường tròn lượng giác như hình vẽ bên dưới. Tìm điểm biểu diễn của góc có số đo bằng $- \frac{7 \pi}{2}$. 
\begin{center}
\begin{tikzpicture}[scale=.7,>=stealth, font=\footnotesize, line join=round, line cap=round]
            \draw[fill=black](0,0) coordinate (O) node[below left]{$O$} circle (1.5pt) (2,0) coordinate (A) node[above right] {$A$} circle (1.5pt) (180:2) coordinate (A2) node [above left] {$A'$} circle (1.5 pt) (90:2) coordinate (B1) node [above right] {$B$} circle (1.5pt) (270:2) coordinate (B2) node [below left] {$B'$} circle (1.5pt);
            \draw[very thick] (0,0) circle (2 cm);
            \draw[->] (-3,0)--(3,0) node [below]{$x$};
            \draw[->] (0,-3)--(0,3) node [left]{$y$};
            \clip (-5,-5) rectangle (5,5);
    \end{tikzpicture} 
\end{center}
\choice
{ \True Điểm ${{B}}$ }
   { Điểm ${{A}}$ }
     { Điểm ${{A'}}$ }
    { Điểm ${{B'}}$ }
\loigiai{ 
 Điểm biểu diễn của góc $- \frac{7 \pi}{2}$ là Điểm ${{B}}$. 
 }\end{ex}

\begin{ex}[17-M2. Tìm điểm biểu diễn các góc pi/4, -pi/4, 3pi/4, -3pi/4]
 Trong mặt phẳng tọa độ ${Oxy},$ cho đường tròn lượng giác như hình vẽ bên dưới. Điểm biểu diễn của góc có số đo bằng $\frac{9 \pi}{4}$ là điểm nào trong các điểm sau? 
\begin{center}
 \begin{tikzpicture}[scale=.7,>=stealth, font=\footnotesize, line join=round, line cap=round]
            \draw[fill=black](0,0) coordinate (O) node[below left]{$O$} circle (1.5pt) (2,0) coordinate (A) (45:2) coordinate (M) node [above] {$M$} circle (1.5 pt) (135:2) coordinate (N) node [above] {$N$} circle (1.5pt) (225:2) coordinate (P) node [below left] {$P$} circle (1.5pt) (-45:2) coordinate (Q) node [below] {$Q$} circle (1.5pt);
            \draw[very thick] (0,0) circle (2 cm);
            \draw[->] (-3,0)--(3,0) node [below]{$x$};
            \draw[->] (0,-3)--(0,3) node [left]{$y$};
            \clip (-5,-5) rectangle (5,5);
    \end{tikzpicture}
\end{center}
\choice
{ Điểm ${{Q}}$ }
   { Điểm ${{N}}$ }
     { Điểm ${{B}}$ }
    { \True Điểm ${{M}}$ }
\loigiai{ 
 Điểm biểu diễn của góc $\frac{9 \pi}{4}$ trùng với điểm biểu diễn của góc $\frac{\pi}{4}$ là Điểm ${{M}}$. 
 }\end{ex}

\begin{ex}
 Trong mặt phẳng tọa độ ${Oxy},$ cho đường tròn lượng giác như hình vẽ bên dưới. Điểm biểu diễn của góc có số đo bằng $- \frac{13 \pi}{4}$ là điểm nào trong các điểm sau? 
\begin{center}
 \begin{tikzpicture}[scale=.7,>=stealth, font=\footnotesize, line join=round, line cap=round]
            \draw[fill=black](0,0) coordinate (O) node[below left]{$O$} circle (1.5pt) (2,0) coordinate (A) (45:2) coordinate (M) node [above] {$M$} circle (1.5 pt) (135:2) coordinate (N) node [above] {$N$} circle (1.5pt) (225:2) coordinate (P) node [below left] {$P$} circle (1.5pt) (-45:2) coordinate (Q) node [below] {$Q$} circle (1.5pt);
            \draw[very thick] (0,0) circle (2 cm);
            \draw[->] (-3,0)--(3,0) node [below]{$x$};
            \draw[->] (0,-3)--(0,3) node [left]{$y$};
            \clip (-5,-5) rectangle (5,5);
    \end{tikzpicture}
\end{center}
\choice
{ \True Điểm ${{N}}$ }
   { Điểm ${{A'}}$ }
     { Điểm ${{A}}$ }
    { Điểm ${{P}}$ }
\loigiai{ 
 Điểm biểu diễn của góc $- \frac{13 \pi}{4}$ trùng với điểm biểu diễn của góc $\frac{3 \pi}{4}$ là Điểm ${{N}}$. 
 }\end{ex}

\begin{ex}
 Trong mặt phẳng tọa độ ${Oxy},$ cho đường tròn lượng giác như hình vẽ bên dưới. Điểm biểu diễn của góc có số đo bằng $\frac{15 \pi}{4}$ là điểm nào trong các điểm sau? 
\begin{center}
 \begin{tikzpicture}[scale=.7,>=stealth, font=\footnotesize, line join=round, line cap=round]
            \draw[fill=black](0,0) coordinate (O) node[below left]{$O$} circle (1.5pt) (2,0) coordinate (A) (45:2) coordinate (M) node [above] {$M$} circle (1.5 pt) (135:2) coordinate (N) node [above] {$N$} circle (1.5pt) (225:2) coordinate (P) node [below left] {$P$} circle (1.5pt) (-45:2) coordinate (Q) node [below] {$Q$} circle (1.5pt);
            \draw[very thick] (0,0) circle (2 cm);
            \draw[->] (-3,0)--(3,0) node [below]{$x$};
            \draw[->] (0,-3)--(0,3) node [left]{$y$};
            \clip (-5,-5) rectangle (5,5);
    \end{tikzpicture}
\end{center}
\choice
{ Điểm ${{P}}$ }
   { Điểm ${{B}}$ }
     { Điểm ${{A'}}$ }
    { \True Điểm ${{Q}}$ }
\loigiai{ 
 Điểm biểu diễn của góc $\frac{15 \pi}{4}$ trùng với điểm biểu diễn của góc $- \frac{\pi}{4}$ là Điểm ${{Q}}$. 
 }\end{ex}

\begin{ex}
 Trong mặt phẳng tọa độ ${Oxy},$ cho đường tròn lượng giác như hình vẽ bên dưới. Điểm biểu diễn của góc có số đo bằng $- \frac{15 \pi}{4}$ là điểm nào trong các điểm sau? 
\begin{center}
 \begin{tikzpicture}[scale=.7,>=stealth, font=\footnotesize, line join=round, line cap=round]
            \draw[fill=black](0,0) coordinate (O) node[below left]{$O$} circle (1.5pt) (2,0) coordinate (A) (45:2) coordinate (M) node [above] {$M$} circle (1.5 pt) (135:2) coordinate (N) node [above] {$N$} circle (1.5pt) (225:2) coordinate (P) node [below left] {$P$} circle (1.5pt) (-45:2) coordinate (Q) node [below] {$Q$} circle (1.5pt);
            \draw[very thick] (0,0) circle (2 cm);
            \draw[->] (-3,0)--(3,0) node [below]{$x$};
            \draw[->] (0,-3)--(0,3) node [left]{$y$};
            \clip (-5,-5) rectangle (5,5);
    \end{tikzpicture}
\end{center}
\choice
{ Điểm ${{Q}}$ }
   { Điểm ${{P}}$ }
     { \True Điểm ${{M}}$ }
    { Điểm ${{B'}}$ }
\loigiai{ 
 Điểm biểu diễn của góc $- \frac{15 \pi}{4}$ trùng với điểm biểu diễn của góc $\frac{\pi}{4}$ là Điểm ${{M}}$. 
 }\end{ex}

\begin{ex}
 Trong mặt phẳng tọa độ ${Oxy},$ cho đường tròn lượng giác như hình vẽ bên dưới. Điểm biểu diễn của góc có số đo bằng $\frac{3 \pi}{4}$ là điểm nào trong các điểm sau? 
\begin{center}
 \begin{tikzpicture}[scale=.7,>=stealth, font=\footnotesize, line join=round, line cap=round]
            \draw[fill=black](0,0) coordinate (O) node[below left]{$O$} circle (1.5pt) (2,0) coordinate (A) (45:2) coordinate (M) node [above] {$M$} circle (1.5 pt) (135:2) coordinate (N) node [above] {$N$} circle (1.5pt) (225:2) coordinate (P) node [below left] {$P$} circle (1.5pt) (-45:2) coordinate (Q) node [below] {$Q$} circle (1.5pt);
            \draw[very thick] (0,0) circle (2 cm);
            \draw[->] (-3,0)--(3,0) node [below]{$x$};
            \draw[->] (0,-3)--(0,3) node [left]{$y$};
            \clip (-5,-5) rectangle (5,5);
    \end{tikzpicture}
\end{center}
\choice
{ Điểm ${{Q}}$ }
   { Điểm ${{P}}$ }
     { Điểm ${{A}}$ }
    { \True Điểm ${{N}}$ }
\loigiai{ 
 Điểm biểu diễn của góc $\frac{3 \pi}{4}$ trùng với điểm biểu diễn của góc $\frac{3 \pi}{4}$ là Điểm ${{N}}$. 
 }\end{ex}

\begin{ex}
 Trong mặt phẳng tọa độ ${Oxy},$ cho đường tròn lượng giác như hình vẽ bên dưới. Điểm biểu diễn của góc có số đo bằng $\frac{11 \pi}{4}$ là điểm nào trong các điểm sau? 
\begin{center}
 \begin{tikzpicture}[scale=.7,>=stealth, font=\footnotesize, line join=round, line cap=round]
            \draw[fill=black](0,0) coordinate (O) node[below left]{$O$} circle (1.5pt) (2,0) coordinate (A) (45:2) coordinate (M) node [above] {$M$} circle (1.5 pt) (135:2) coordinate (N) node [above] {$N$} circle (1.5pt) (225:2) coordinate (P) node [below left] {$P$} circle (1.5pt) (-45:2) coordinate (Q) node [below] {$Q$} circle (1.5pt);
            \draw[very thick] (0,0) circle (2 cm);
            \draw[->] (-3,0)--(3,0) node [below]{$x$};
            \draw[->] (0,-3)--(0,3) node [left]{$y$};
            \clip (-5,-5) rectangle (5,5);
    \end{tikzpicture}
\end{center}
\choice
{ Điểm ${{P}}$ }
   { Điểm ${{B'}}$ }
     { Điểm ${{M}}$ }
    { \True Điểm ${{N}}$ }
\loigiai{ 
 Điểm biểu diễn của góc $\frac{11 \pi}{4}$ trùng với điểm biểu diễn của góc $\frac{3 \pi}{4}$ là Điểm ${{N}}$. 
 }\end{ex}

\begin{ex}[18-M2. Tìm góc có điểm biểu diễn đối xứng với M qua O]
   Trong mặt phẳng tọa độ ${Oxy}$, trên đường tròn lượng giác gọi điểm ${P}$ là điểm biểu diễn của góc $\frac{5 \pi}{6}$. Lấy điểm ${Q}$ đối xứng với ${P}$ qua gốc tọa độ. Hỏi ${Q}$ là điểm biểu diễn của góc có số đo bằng bao nhiêu? 
\choice
{ $\frac{\pi}{6}$ }
   { $- \frac{5 \pi}{6}$ }
     { \True $\frac{11 \pi}{6}$ }
    { $\frac{4 \pi}{3}$ }
\loigiai{ 
 ${Q}$ là điểm biểu diễn của góc $\frac{5 \pi}{6}+\pi=\frac{11 \pi}{6}$. 
 }\end{ex}

\begin{ex}
   Trong mặt phẳng tọa độ ${Oxy}$, trên đường tròn lượng giác gọi điểm ${P}$ là điểm biểu diễn của góc $\frac{\pi}{5}$. Lấy điểm ${Q}$ đối xứng với ${P}$ qua gốc tọa độ. Hỏi ${Q}$ là điểm biểu diễn của góc có số đo bằng bao nhiêu? 
\choice
{ $\frac{4 \pi}{5}$ }
   { $\frac{9 \pi}{20}$ }
     { $\frac{7 \pi}{10}$ }
    { \True $\frac{6 \pi}{5}$ }
\loigiai{ 
 ${Q}$ là điểm biểu diễn của góc $\frac{\pi}{5}+\pi=\frac{6 \pi}{5}$. 
 }\end{ex}

\begin{ex}
   Trong mặt phẳng tọa độ ${Oxy}$, trên đường tròn lượng giác gọi điểm ${N}$ là điểm biểu diễn của góc $\frac{\pi}{5}$. Lấy điểm ${P}$ đối xứng với ${N}$ qua gốc tọa độ. Hỏi ${P}$ là điểm biểu diễn của góc có số đo bằng bao nhiêu? 
\choice
{ $- \frac{\pi}{5}$ }
   { \True $\frac{6 \pi}{5}$ }
     { $\frac{7 \pi}{10}$ }
    { $- \frac{3 \pi}{10}$ }
\loigiai{ 
 ${P}$ là điểm biểu diễn của góc $\frac{\pi}{5}+\pi=\frac{6 \pi}{5}$. 
 }\end{ex}

\begin{ex}
   Trong mặt phẳng tọa độ ${Oxy}$, trên đường tròn lượng giác gọi điểm ${M}$ là điểm biểu diễn của góc $\frac{4 \pi}{7}$. Lấy điểm ${Q}$ đối xứng với ${M}$ qua gốc tọa độ. Hỏi ${Q}$ là điểm biểu diễn của góc có số đo bằng bao nhiêu? 
\choice
{ $- \frac{4 \pi}{7}$ }
   { $\frac{15 \pi}{14}$ }
     { $\frac{23 \pi}{28}$ }
    { \True $\frac{11 \pi}{7}$ }
\loigiai{ 
 ${Q}$ là điểm biểu diễn của góc $\frac{4 \pi}{7}+\pi=\frac{11 \pi}{7}$. 
 }\end{ex}

\begin{ex}
   Trong mặt phẳng tọa độ ${Oxy}$, trên đường tròn lượng giác gọi điểm ${Q}$ là điểm biểu diễn của góc $\frac{2 \pi}{3}$. Lấy điểm ${P}$ đối xứng với ${Q}$ qua gốc tọa độ. Hỏi ${P}$ là điểm biểu diễn của góc có số đo bằng bao nhiêu? 
\choice
{ \True $\frac{5 \pi}{3}$ }
   { $\frac{\pi}{6}$ }
     { $\pi$ }
    { $\frac{7 \pi}{6}$ }
\loigiai{ 
 ${P}$ là điểm biểu diễn của góc $\frac{2 \pi}{3}+\pi=\frac{5 \pi}{3}$. 
 }\end{ex}

\begin{ex}
   Trong mặt phẳng tọa độ ${Oxy}$, trên đường tròn lượng giác gọi điểm ${M}$ là điểm biểu diễn của góc $\frac{\pi}{4}$. Lấy điểm ${N}$ đối xứng với ${M}$ qua gốc tọa độ. Hỏi ${N}$ là điểm biểu diễn của góc có số đo bằng bao nhiêu? 
\choice
{ $- \frac{\pi}{4}$ }
   { \True $\frac{5 \pi}{4}$ }
     { $\frac{9 \pi}{4}$ }
    { $\frac{7 \pi}{12}$ }
\loigiai{ 
 ${N}$ là điểm biểu diễn của góc $\frac{\pi}{4}+\pi=\frac{5 \pi}{4}$. 
 }\end{ex}

\begin{ex}[19-M2. Tìm góc có điểm biểu diễn đối xứng với M qua Ox]
   Trong mặt phẳng tọa độ ${Oxy}$, trên đường tròn lượng giác gọi điểm ${P}$ là điểm biểu diễn của góc $\frac{5 \pi}{9}$. Lấy điểm ${N}$ đối xứng với ${P}$ qua trục ${Ox}$. Hỏi ${N}$ là điểm biểu diễn của góc có số đo bằng bao nhiêu? 
\choice
{ $\frac{4 \pi}{9}$ }
   { \True $- \frac{5 \pi}{9}$ }
     { $\frac{23 \pi}{9}$ }
    { $\frac{8 \pi}{9}$ }
\loigiai{ 
 ${N}$ là điểm biểu diễn của góc $- \frac{5 \pi}{9}$. 
 }\end{ex}

\begin{ex}
   Trong mặt phẳng tọa độ ${Oxy}$, trên đường tròn lượng giác gọi điểm ${M}$ là điểm biểu diễn của góc $\frac{5 \pi}{8}$. Lấy điểm ${E}$ đối xứng với ${M}$ qua trục ${Ox}$. Hỏi ${E}$ là điểm biểu diễn của góc có số đo bằng bao nhiêu? 
\choice
{ $\frac{21 \pi}{8}$ }
   { \True $- \frac{5 \pi}{8}$ }
     { $\frac{3 \pi}{8}$ }
    { $\frac{13 \pi}{8}$ }
\loigiai{ 
 ${E}$ là điểm biểu diễn của góc $- \frac{5 \pi}{8}$. 
 }\end{ex}

\begin{ex}
   Trong mặt phẳng tọa độ ${Oxy}$, trên đường tròn lượng giác gọi điểm ${N}$ là điểm biểu diễn của góc $\frac{2 \pi}{5}$. Lấy điểm ${Q}$ đối xứng với ${N}$ qua trục ${Ox}$. Hỏi ${Q}$ là điểm biểu diễn của góc có số đo bằng bao nhiêu? 
\choice
{ $\frac{12 \pi}{5}$ }
   { \True $- \frac{2 \pi}{5}$ }
     { $\frac{9 \pi}{10}$ }
    { $\frac{11 \pi}{15}$ }
\loigiai{ 
 ${Q}$ là điểm biểu diễn của góc $- \frac{2 \pi}{5}$. 
 }\end{ex}

\begin{ex}
   Trong mặt phẳng tọa độ ${Oxy}$, trên đường tròn lượng giác gọi điểm ${E}$ là điểm biểu diễn của góc $\frac{3 \pi}{5}$. Lấy điểm ${P}$ đối xứng với ${E}$ qua trục ${Ox}$. Hỏi ${P}$ là điểm biểu diễn của góc có số đo bằng bao nhiêu? 
\choice
{ \True $- \frac{3 \pi}{5}$ }
   { $\frac{11 \pi}{10}$ }
     { $\frac{17 \pi}{20}$ }
    { $\frac{\pi}{10}$ }
\loigiai{ 
 ${P}$ là điểm biểu diễn của góc $- \frac{3 \pi}{5}$. 
 }\end{ex}

\begin{ex}
   Trong mặt phẳng tọa độ ${Oxy}$, trên đường tròn lượng giác gọi điểm ${Q}$ là điểm biểu diễn của góc $\frac{2 \pi}{3}$. Lấy điểm ${E}$ đối xứng với ${Q}$ qua trục ${Ox}$. Hỏi ${E}$ là điểm biểu diễn của góc có số đo bằng bao nhiêu? 
\choice
{ $\frac{\pi}{3}$ }
   { $\frac{11 \pi}{12}$ }
     { $\frac{7 \pi}{6}$ }
    { \True $- \frac{2 \pi}{3}$ }
\loigiai{ 
 ${E}$ là điểm biểu diễn của góc $- \frac{2 \pi}{3}$. 
 }\end{ex}

\begin{ex}
   Trong mặt phẳng tọa độ ${Oxy}$, trên đường tròn lượng giác gọi điểm ${N}$ là điểm biểu diễn của góc $\frac{3 \pi}{5}$. Lấy điểm ${M}$ đối xứng với ${N}$ qua trục ${Ox}$. Hỏi ${M}$ là điểm biểu diễn của góc có số đo bằng bao nhiêu? 
\choice
{ $\frac{17 \pi}{20}$ }
   { \True $- \frac{3 \pi}{5}$ }
     { $\frac{13 \pi}{5}$ }
    { $\frac{8 \pi}{5}$ }
\loigiai{ 
 ${M}$ là điểm biểu diễn của góc $- \frac{3 \pi}{5}$. 
 }\end{ex}

\Closesolutionfile{ans}

 \begin{center}
-----HẾT-----
\end{center}

 %\newpage 
%\begin{center}
%{\bf BẢNG ĐÁP ÁN MÃ ĐỀ 1 }
%\end{center}
%{\bf Phần 1 }
% \inputansbox{6}{ans001-1}



\end{document}